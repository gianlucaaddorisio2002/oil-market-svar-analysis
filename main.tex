\documentclass[11pt,a4paper,oneside]{report}

% ====== Lingua e codifica ======
\usepackage[utf8]{inputenc}
\usepackage[T1]{fontenc}
\usepackage[english]{babel}
\usepackage{float}
\usepackage{placeins}
\usepackage{threeparttable}

% ====== Tipografia e layout ======
\usepackage{lmodern}
\usepackage{microtype}
\usepackage{amsmath}
\usepackage{setspace}
\usepackage{verbatim}
\usepackage{enumitem}
\usepackage{geometry}
\geometry{
  a4paper,
  top=3.2cm,
  bottom=3.0cm,
  left=3.0cm,
  right=3.0cm
}

\setlength{\headheight}{12pt}
\setlength{\headsep}{8pt}

% ====== Grafica e tabelle ======
\usepackage[table]{xcolor}
\usepackage{booktabs}
\usepackage{tabularx}
\usepackage{array}
\definecolor{Header}{HTML}{4F81BD}
\definecolor{RowLight}{HTML}{EEF3F9}
\usepackage{graphicx}
\usepackage{caption}
\usepackage{subcaption}
\usepackage{siunitx}
\sisetup{output-decimal-marker={,}}

% ====== Header/Footer ======
\usepackage{ragged2e}
\usepackage{adjustbox}
\newcolumntype{Y}{>{\RaggedRight\arraybackslash\hspace{0pt}}X} % X con sillabazione
\newcolumntype{P}[1]{>{\RaggedRight\arraybackslash}p{#1}}      % p{..} leggibile
\usepackage{fancyhdr}
\pagestyle{fancy}
\fancyhf{}
\renewcommand{\headrulewidth}{0pt}
\fancyfoot[C]{\thepage}

% ====== Link e metadata ======
\usepackage{xcolor}
\definecolor{linkblue}{HTML}{0A3D62}
\usepackage[
  colorlinks=true,
  linkcolor=linkblue,
  citecolor=linkblue,
  urlcolor=linkblue,
  pdfauthor={Gianluca Adriano Junior Addorisio},
  pdftitle={Structural Identification of Oil Supply, Global Demand and Shocks: A Factor Analysis with Industrial Applications},
  pdfsubject={Tesi di Laurea Triennale – Ingegneria Industriale}
]{hyperref}

% ====== Spaziatura ======
\onehalfspacing

% ====== Dati di comodo ======
\newcommand{\Matricola}{Matr. N 15935}
\newcommand{\Ateneo}{UNIVERSITÀ CAMPUS BIO-MEDICO DI ROMA}
\newcommand{\Facolta}{FACOLTÀ DIPARTIMENTALE DI INGEGNERIA}
\newcommand{\Corso}{CORSO DI LAUREA TRIENNALE IN\\[0.2cm] INGEGNERIA INDUSTRIALE}
\newcommand{\Titolo}{Structural Identification of Oil Supply, Global Demand and Shocks:\\[0.2cm] A Factor Analysis with Industrial Applications} 
\newcommand{\Relatore}{Ch.mo Dott. Marco Papi}
\newcommand{\Laureando}{Gianluca Adriano Junior Addorisio}
\newcommand{\Anno}{ANNO ACCADEMICO 2024/2025}

\usepackage{csquotes}
\usepackage{tikz}

\usepackage[backend=biber,style=numeric]{biblatex}
\addbibresource{bibliografia.bib}


\DeclareCiteCommand{\cite}[\mkbibsuperscript]
  {\usebibmacro{prenote}}
  {%
    \usebibmacro{citeindex}%
    \printtext[bibhyperref]{%
      \tikz[baseline=(n.base)]%
        \node[draw,inner sep=1pt,rounded corners=2pt](n)%
        {\scriptsize\textbf{\printfield{labelnumber}}};%
    }%
  }
  {\multicitedelim}
  {\usebibmacro{postnote}}

% =========================================================
%                       DOCUMENTO
% =========================================================
\begin{document}

\begin{titlepage}

  % ====== Matricola in alto a sinistra ======
  \noindent{\fontsize{11}{13}\selectfont \Matricola}

  % Linea orizzontale sottile in alto
  \vspace{0.3cm}
  \noindent\rule{\textwidth}{0.2pt}

  \vspace{0.8cm}

  \begin{center}

    % ====== Logo ======
    \includegraphics[width=4cm]{logo_campus.png}

    \vspace{1cm}

    % ====== Università ======
    {\fontsize{16}{19}\selectfont \textbf{UNIVERSITÀ CAMPUS BIO-MEDICO DI ROMA}}

    \vspace{0.8cm}

    % ====== Facoltà e Corso ======
    {\fontsize{13}{16}\selectfont \textbf{Facoltà Dipartimentale di Ingegneria}}\\[0.20cm]
    {\fontsize{13}{16}\selectfont \textbf{Corso di Laurea in Ingegneria Industriale}}

    \vspace{2cm}

    % ====== Titolo (in inglese) ======
    {\fontsize{18}{22}\selectfont\textbf{STRUCTURAL IDENTIFICATION OF OIL}}\\[0.25cm]
    {\fontsize{18}{22}\selectfont\textbf{SUPPLY, GLOBAL DEMAND AND SHOCKS:}}\\[0.25cm]
    {\fontsize{18}{22}\selectfont\textbf{A FACTOR ANALYSIS WITH INDUSTRIAL APPLICATIONS}}

  \end{center}

  \vspace{2.5cm}

  % ====== Relatore a sinistra ======
  \noindent
  \begin{minipage}[t]{0.48\textwidth}
    {\fontsize{13}{16}\selectfont \textbf{Relatore}}\\[0.15cm]
    {\fontsize{13}{16}\selectfont \Relatore}
  \end{minipage}%
  \hfill
  % ====== Laureando più in basso a destra ======
  \begin{minipage}[t]{0.48\textwidth}
    \raggedleft
    \vspace{1.2cm} % <<-- QUI lo abbassi rispetto al Relatore
    {\fontsize{13}{16}\selectfont \textbf{Laureando}}\\[0.15cm]
    {\fontsize{13}{16}\selectfont \Laureando}
  \end{minipage}

  \vfill

  % Linea orizzontale sottile in basso
  \noindent\rule{\textwidth}{0.2pt}

  \vspace{0.3cm}

  \begin{center}
    {\fontsize{13}{16}\selectfont \textbf{\Anno}}
  \end{center}

\end{titlepage}



 \newpage
  \thispagestyle{empty}
 \null
 \newpage
% Dedica (opzionale)
 \thispagestyle{empty}
  \vspace*{2.4cm}
 \begin{flushright}
 \emph{To my parents}
 \end{flushright}
 \vspace*{\fill}
 \clearpage
\pagenumbering{roman}
\setcounter{page}{1}


\clearpage
\thispagestyle{empty}

% ===== SOMMARIO (in italiano) =====
\begin{center}
  \textbf{Sommario}
\end{center}

Questa tesi analizza i meccanismi strutturali che guidano le oscillazioni del prezzo del petrolio e ne valuta le implicazioni per la gestione del rischio carburante nel settore del trasporto aereo.
Viene costruito un dataset mensile 1990–2024 su produzione USA, attività economica OCSE, scorte petrolifere USA e prezzo reale del WTI, viene stimato un modello VAR a quattro variabili su trasformazioni stazionarie, evidenziando l’inadeguatezza delle regressioni statiche e la necessità di un approccio dinamico multivariato.

Lo SVAR è identificato mediante restrizioni di segno in stile Kilian–Murphy, distinguendo shock di offerta, shock di domanda aggregata e shock precauzionali. Gli shock estratti presentano deviazioni marcate dalla normalità e forte dipendenza di coda, confermata tramite copule t. Le risposte agli impulsi mostrano che la domanda globale è il principale driver del prezzo del WTI nel medio periodo, mentre gli shock di offerta sono poco persistenti e quelli precauzionali diventano rilevanti soprattutto negli eventi estremi.

Questi risultati alimentano la generazione di scenari di prezzo e un esercizio di stress test ispirato a una chiusura dello Stretto di Hormuz. La mappatura verso il prezzo del Jet Fuel e l’applicazione al caso ITA Airways indicano che una copertura lineare completa riduce sensibilmente l’esposizione ai picchi di prezzo determinati da shock di offerta estremi.

\vspace{0.8cm}

% ===== ABSTRACT (in inglese) =====
\begin{center}
  \textbf{Abstract}
\end{center}

This thesis investigates the structural drivers of crude oil price fluctuations and
their implications for fuel-price risk management in energy-intensive
industries.
Using monthly data for 1990--2024, a four-variable VAR is estimated on
stationary transformations of world oil production, OECD real activity,
petroleum inventories and the real WTI price.
Reduced-form evidence shows that static regressions are unstable and
severely misspecified, motivating a dynamic multivariate approach. A
structural VAR is then identified through sign restrictions in the spirit of
Kilian and Murphy (2014), disentangling flow-supply, aggregate-demand and
precautionary shocks.
The structural shocks exhibit marked non-Gaussian features, strong
leptokurtosis in demand and precautionary innovations, and significant tail
dependence, captured through t-copulas.
The estimated impulse responses imply that global demand is the dominant
source of medium-run WTI variability, while supply shocks are short-lived and
precautionary shocks contribute primarily through extreme events.
These results feed into model-based scenario generation and a stress test
calibrated on a Strait of Hormuz-type disruption. Mapping WTI paths into jet
fuel prices, and applying them to the fuel bill of ITA Airways, shows that a
full linear hedge substantially attenuates the cost impact of extreme
supply disruptions.

\clearpage



\clearpage

% Indice generale, elenco figure e tabelle
\tableofcontents
\clearpage
% \listoffigures
% \clearpage
% \listoftables
% \clearpage

\clearpage
\listoffigures
\clearpage
\listoftables
\clearpage


\pagenumbering{arabic}




%\chapter{Dispensa}
\section{Dalla Regressione Lineare Statica al Modello VAR Dinamico}

In una prima fase dell’analisi è stato stimato un modello lineare statico di tipo OLS con variabile dipendente la variazione del prezzo del petrolio (WTI) e tre regressori fondamentali: produzione di petrolio, attività economica globale (REA) e scorte. Il modello, stimato tramite il comando \texttt{fitlm} di MATLAB, ha restituito risultati coerenti con quanto atteso dalla letteratura sulle serie energetiche: bassa capacità esplicativa, scarsa significatività dei coefficienti e violazioni sistematiche delle assunzioni classiche dell’OLS.

\subsection{Limiti del modello statico}
La diagnostica condotta sui residui ha mostrato:
\begin{itemize}
    \item \textbf{autocorrelazione seriale} (Ljung--Box sempre significativo);
    \item \textbf{eteroschedasticità} (Breusch--Pagan con $p=0$);
    \item \textbf{non normalità nelle code} (deviazioni marcate nel QQ-plot e presenza di heavy tails);
    \item \textbf{R\textsuperscript{2} estremamente basso} e \textbf{coefficiente di determinazione corretto quasi nullo};
    \item \textbf{instabilità parametrica nel tempo}, confermata dall’analisi rolling.
\end{itemize}

Tali evidenze indicano che il modello statico non riesce a catturare la dinamica del processo generatore dei dati, caratterizzato da forte dipendenza temporale, non linearità, shock e comportamenti asimmetrici nelle code. L’OLS si rivela quindi inadeguato per descrivere la relazione tra variabili macroeconomiche e prezzo del petrolio.

\subsection{Motivazione per il passaggio al modello VAR}
Per superare i limiti del modello statico, è stato adottato un modello VAR (Vector Autoregression), che consente di:
\begin{itemize}
    \item modellare simultaneamente le interazioni dinamiche tra più variabili endogene;
    \item catturare la propagazione degli shock nel tempo;
    \item incorporare memoria tramite valori ritardati (lag);
    \item produrre strumenti di analisi dinamica quali Impulse Response Functions (IRF) e Forecast Error Variance Decomposition (FEVD).
\end{itemize}

A differenza dell’OLS, che impone una struttura unidirezionale e contemporanea, il VAR consente a ciascuna variabile di influenzare e di essere influenzata dalle altre nel tempo. La specificazione stimata è un modello VAR(p), selezionato sulla base di criteri di stabilità e diagnosi dei residui, con $p=7$ ritardi mensili. Questo approccio segue la tradizione della letteratura sui mercati petroliferi (Kilian, 2009; Kilian and Murphy, 2014), dove l’utilizzo di un numero elevato di lag è standard per garantire residui non autocorrelati.

\subsection{Cosa si conserva e cosa cambia rispetto all’OLS}
L’analisi OLS preliminare fornisce informazioni utili, ma limitate:
\begin{itemize}
    \item si conferma che la domanda globale (REA) mostra il legame più consistente con il WTI;
    \item produzione e scorte risultano debolmente informative in forma contemporanea;
    \item la struttura delle code e la non linearità richiedono modelli più flessibili.
\end{itemize}

Il modello VAR introduce invece nuovi elementi fondamentali:
\begin{itemize}
    \item \textbf{dinamica temporale} tramite valori ritardati;
    \item \textbf{relazioni bi-direzionali} tra tutte le variabili;
    \item possibilità di identificare \textbf{shock strutturali} (supply, demand, precautionary);
    \item \textbf{analisi IRF e FEVD} per studiare come gli shock si propagano e quale quota della varianza del WTI essi spiegano;
    \item base per la \textbf{Historical Decomposition} e per applicazioni di risk management.
\end{itemize}

\subsection{Struttura del modello VAR stimato}
Il modello stimato è un VAR(7) sulle serie stazionarie \textit{Production\_DL}, \textit{REA\_DL}, \textit{WTI\_real\_DL} e \textit{Inventories\_DL}, definito come:
\[
    Y_t = A_1 Y_{t-1} + A_2 Y_{t-2} + \dots + A_7 Y_{t-7} + u_t,
\]
dove $Y_t$ contiene le quattro variabili del sistema. Il modello risulta stabile e i residui non presentano autocorrelazione, garantendo la validità econometrica delle IRF e delle decomposizioni successive.

\subsection{Obiettivi dell’analisi VAR}
L’adozione del modello VAR consente di:
\begin{enumerate}
    \item valutare la risposta dinamica delle variabili agli shock fondamentali;
    \item quantificare l’importanza relativa di ciascuno shock (FEVD);
    \item identificare la natura dei movimenti del prezzo del petrolio (supply, demand, precautionary);
    \item fornire la base per applicazioni di \textbf{hedging}, \textbf{stress test} e \textbf{scenario analysis} per imprese esposte al rischio energetico.
\end{enumerate}

In sintesi, il passaggio dal modello OLS al modello VAR rappresenta un’evoluzione necessaria e metodologicamente fondata, che permette di descrivere in modo realistico il comportamento dinamico del mercato petrolifero.

\section{Dal modello OLS statico al VAR dinamico e al VAR strutturale}

In una prima fase dell'analisi è stato stimato un modello lineare statico (OLS) che mette in relazione la variazione del prezzo reale del petrolio con tre fondamentali osservabili del mercato:

\begin{equation}
\Delta \log(\text{WTI}_t) 
= \alpha 
+ \beta_1 \Delta \log(\text{Production}_t)
+ \beta_2 \Delta \text{REA}_t
+ \beta_3 \Delta \log(\text{Inventories}_t)
+ \varepsilon_t.
\end{equation}

I risultati empirici di questa regressione mostrano:
\begin{itemize}
    \item un coefficiente di determinazione $R^2$ molto basso e $Adj.\,R^2$ prossimo a zero;
    \item un numero limitato di coefficienti statisticamente significativi;
    \item residui affetti da autocorrelazione (test di Ljung--Box), eteroschedasticità (Breusch--Pagan) e deviazioni dalla normalità nelle code;
    \item evidenza, tramite analisi rolling, di instabilità del legame nel tempo.
\end{itemize}

In sintesi, il modello OLS statico ha un contenuto informativo limitato: le variabili fondamentali considerate (produzione, attività economica, scorte) spiegano solo una frazione minima della variabilità di breve periodo del prezzo del petrolio, e le ipotesi classiche dell'OLS risultano sistematicamente violate. Ciò suggerisce che la dinamica del mercato petrolifero non possa essere descritta in termini di relazioni contemporanee e statiche, ma richieda un modello esplicitamente dinamico.

\subsection{Adozione di un modello VAR dinamico}

Per superare i limiti del modello statico è stato stimato un modello VAR sulle serie stazionarie in differenze logaritmiche:

\begin{equation}
Y_t = 
\begin{bmatrix}
\Delta \log(\text{Production}_t) \\
\Delta \text{REA}_t \\
\Delta \log(\text{WTI}_t) \\
\Delta \log(\text{Inventories}_t)
\end{bmatrix}
=
A_1 Y_{t-1} + A_2 Y_{t-2} + \dots + A_p Y_{t-p} + u_t,
\end{equation}

dove $u_t$ è il vettore degli errori ridotti-forma. Tramite una procedura incrementale si è considerato un ordine $p$ da 1 a 8, verificando per ciascun modello:
\begin{enumerate}
    \item la \emph{stabilità} (tutti gli autovalori della companion matrix all'interno del cerchio unitario);
    \item l'\emph{assenza di autocorrelazione residua} (test di Ljung--Box sui residui di ciascuna equazione).
\end{enumerate}

Tutti i modelli risultano stabili, ma l'autocorrelazione residua viene eliminata solo a partire da $p=7$. Il VAR(7) rappresenta quindi il primo modello econometricamente valido (stabile e con residui assimilabili a rumore bianco), ed è adottato come specificazione finale. Questo risultato conferma che il sistema petrolifero presenta una \emph{memoria lunga}: sono necessari almeno sette ritardi mensili per catturare in modo adeguato la dinamica congiunta di produzione, attività economica, prezzo e scorte.

\subsection{IRF e FEVD del VAR(7): cosa aggiunge rispetto all'OLS}

Una volta stimato il VAR(7), è possibile analizzare:
\begin{itemize}
    \item le \emph{Impulse Response Functions} (IRF), che descrivono la risposta dinamica di ciascuna variabile a uno shock unitario nelle altre;
    \item la \emph{Forecast Error Variance Decomposition} (FEVD), che quantifica la quota di varianza dell'errore di previsione attribuibile a ciascun shock.
\end{itemize}

L'identificazione iniziale avviene tramite una decomposizione di Cholesky della matrice di covarianza degli errori, assumendo l'ordine economico
\[
\text{Production} \rightarrow \text{REA} \rightarrow \text{WTI} \rightarrow \text{Inventories},
\]
coerente con l'idea che la produzione sia più esogena nel breve periodo, seguita dall'attività economica, dal prezzo e infine dalle scorte.

Le IRF ortogonalizzate mostrano che:
\begin{itemize}
    \item uno shock nella produzione (interpretabile, in ridotto-forma, come perturbazione lato offerta) genera una risposta del prezzo del WTI piccola, oscillante e di breve durata;
    \item uno shock nella REA (domanda aggregata) induce una reazione relativamente più ampia del prezzo, concentrata nei primi mesi, ma non particolarmente persistente;
    \item uno shock positivo nelle scorte commerciali (\emph{Inventories}) tende, in ridotto-forma, ad associare un calo temporaneo del prezzo, coerente con un aumento percepito della disponibilità fisica, ma non identificabile di per sé come \emph{shock di domanda precauzionale}.
\end{itemize}

La FEVD del WTI conferma che gli shock fondamentali spiegano solo una quota ridotta della variabilità del prezzo:
\begin{itemize}
    \item shock associati alla produzione di petrolio spiegano circa il $2.8\%$ della varianza del WTI a 30 mesi;
    \item shock legati all'attività economica (domanda aggregata) spiegano circa il $5.9\%$;
    \item shock sulle scorte spiegano circa il $9.8\%$;
    \item il residuo della varianza (oltre l'80\%) è attribuibile a innovazioni idiosincratiche del prezzo stesso.
\end{itemize}

Questi risultati sono coerenti con l'evidenza del modello OLS: anche in un contesto dinamico multivariato, i fondamentali inclusi nel modello spiegano solo una frazione limitata dei movimenti del prezzo. Tuttavia, rispetto all'OLS, il VAR consente di:
\begin{itemize}
    \item modellare esplicitamente la \emph{propagazione temporale} degli shock;
    \item distinguere tra orizzonti di breve e medio periodo;
    \item decomporre la varianza del prezzo tra le diverse fonti di innovazione.
\end{itemize}

\subsection{Perché il VAR ridotto-forma non è ancora sufficiente}

Il VAR stimato finora è un modello \emph{di forma ridotta}: gli shock $u_t$ sono innovazioni statistiche, ortogonalizzate tramite Cholesky, ma non hanno ancora un significato economico univoco. In particolare:
\begin{itemize}
    \item uno shock nella variabile ``scorte'' non coincide automaticamente con uno shock di \emph{domanda precauzionale}; può riflettere variazioni stagionali, sorprese negli stoccaggi, decisioni dell'OPEC o altre componenti miste;
    \item uno shock nella produzione non è necessariamente un puro shock di offerta esogena, ma può inglobare risposte endogene del lato produttivo a shock di domanda;
    \item la decomposizione Cholesky dipende in modo forte dall'ordine delle variabili e impone una struttura triangolare che raramente coincide con la struttura economica vera.
\end{itemize}

In assenza di ulteriori restrizioni, gli shock ridotti-forma non permettono di distinguere in modo pulito tra:
\begin{enumerate}
    \item \textbf{shock di offerta} (esogeni alla produzione fisica di petrolio);
    \item \textbf{shock di domanda aggregata} (legati al ciclo economico globale);
    \item \textbf{shock di domanda precauzionale} (variazioni nelle aspettative di scarsità futura che si manifestano tramite movimenti congiunti di prezzo e scorte).
\end{enumerate}

Per esempio, nella formulazione di \textcite{KilianMurphy2014}, uno shock di domanda precauzionale è definito come uno shock che, nel breve periodo, fa salire il prezzo del petrolio e scendere le scorte commerciali, in quanto gli operatori accumulano petrolio per motivi precauzionali. Un semplice shock positivo nelle scorte osservate non soddisfa questa firma: al contrario, un aumento delle scorte commerciali tende a segnalare abbondanza fisica e quindi a ridurre il prezzo. Ne consegue che l'IRF ridotto-forma ``Inventories $\rightarrow$ WTI'' non può essere interpretata direttamente come risposta a uno shock di domanda precauzionale.

In altri termini, il VAR ridotto-forma fornisce una descrizione corretta della dinamica statistica, ma non identifica ancora gli \emph{shock strutturali} sottostanti.

\subsection{Verso un VAR strutturale con restrizioni di segno}

Per passare dagli shock ridotti-forma $u_t$ agli shock strutturali $\varepsilon_t$ è necessario introdurre una matrice di impatto $B$ tale che:
\begin{equation}
u_t = B \varepsilon_t, \qquad
E[\varepsilon_t \varepsilon_t'] = I.
\end{equation}
La scelta di $B$ non è unica: molte matrici diverse generano la stessa matrice di covarianza dei residui, $\Sigma_u = E[u_t u_t'] = B B'$. La decomposizione di Cholesky fornisce una particolare $B$, ma impone una struttura rigida (triangolare) e fortemente dipendente dall'ordinamento delle variabili.

L'approccio proposto da \textcite{Kilian2009} e \textcite{KilianMurphy2014} consiste nell'utilizzare un insieme di \emph{restrizioni di segno} sulle IRF per identificare gli shock strutturali. In pratica:
\begin{itemize}
    \item si generano molte possibili matrici $B$ compatibili con $\Sigma_u$ (ad esempio rotazioni casuali della fattorizzazione di Cholesky);
    \item per ciascuna di esse si calcolano le IRF strutturali e si verifica se soddisfano un insieme di segni attesi nei primi periodi (es.\,0--3 mesi);
    \item si conservano solo le decomposizioni che rispettano tutte le restrizioni di segno, ottenendo così una famiglia di IRF coerenti con la teoria economica.
\end{itemize}

Esempi di restrizioni di segno possono essere:
\begin{itemize}
    \item \textbf{shock di offerta}: produzione in diminuzione, prezzo del WTI in aumento, scorte in diminuzione nel breve periodo;
    \item \textbf{shock di domanda aggregata}: attività economica (REA) e prezzo del WTI entrambe in aumento nel breve periodo;
    \item \textbf{shock di domanda precauzionale}: prezzo del WTI in aumento e scorte commerciali in diminuzione, con effetti limitati sulla produzione e sull'attività reale nel brevissimo termine.
\end{itemize}

L'uso di restrizioni di segno consente di:
\begin{enumerate}
    \item separare gli shock osservati nelle equazioni ridotte-forma in tre categorie economicamente interpretabili;
    \item ottenere IRF strutturali e FEVD che abbiano un significato microfondato;
    \item collegare la dinamica del prezzo del petrolio a scenari di rischio ben definiti (shock di offerta, domanda aggregata, domanda precauzionale) utili per analisi di \emph{stress testing} e strategie di \emph{hedging}.
\end{enumerate}

In conclusione, il passaggio OLS $\rightarrow$ VAR(7) permette di modellare in modo consistente la dinamica del sistema e di valutare il contributo complessivo dei fondamentali alla volatilità del prezzo del petrolio. Tuttavia, solo l'estensione a un VAR strutturale con restrizioni di segno consente di identificare in maniera credibile i tre shock fondamentali dell'economia petrolifera e di trasformare i risultati del modello in strumenti operativi per la gestione del rischio.

\section{Dal VAR ridotto-forma al VAR strutturale con restrizioni di segno}

Una volta stimato un modello VAR ridotto-forma, è possibile analizzare la dinamica congiunta delle variabili tramite le funzioni di risposta agli impulsi (IRF) e la scomposizione della varianza dell'errore di previsione (FEVD). Tuttavia, gli shock del VAR ridotto-forma non hanno un significato economico univoco: rappresentano semplici innovazioni statistiche, non interpretabili direttamente come shock di offerta, di domanda aggregata o di domanda precauzionale.

Per ottenere una decomposizione economica degli shock è necessario passare a un modello VAR \textit{strutturale} (SVAR), introducendo una matrice di impatto $B$ tale che:
\[
u_t = B \varepsilon_t, \qquad E[\varepsilon_t \varepsilon_t'] = I,
\]
dove $u_t$ sono i residui ridotto-forma e $\varepsilon_t$ rappresentano gli shock strutturali. Poiché la matrice $B$ non è identificata univocamente dalla matrice di covarianza $E[u_t u_t']$, è necessario introdurre ulteriori restrizioni.

Seguendo l’approccio proposto da \textcite{Kilian2009} e \textcite{KilianMurphy2014}, si adottano \emph{restrizioni di segno} sulle IRF nei primi istanti di risposta. L’idea è semplice: ogni shock strutturale deve produrre nel brevissimo periodo un insieme di risposte con segni coerenti con la teoria economica.

Nel caso in esame, le restrizioni adottate sono:

\begin{itemize}
    \item \textbf{Shock di offerta (Supply shock)}: la produzione diminuisce e il prezzo del petrolio aumenta nell'immediato;
    \item \textbf{Shock di domanda aggregata}: l'attività economica (REA) aumenta e il prezzo del petrolio aumenta nell'immediato;
    \item \textbf{Shock di domanda precauzionale}: le scorte commerciali diminuiscono e il prezzo del petrolio aumenta nell'immediato.
\end{itemize}

Per identificare gli shock strutturali si generano migliaia di matrici ortogonali casuali $Q$ (rotazioni), applicate alla fattorizzazione di Cholesky del VAR. Solo le rotazioni che soddisfano simultaneamente tutte le restrizioni di segno vengono accettate.

L’insieme delle IRF accettate viene quindi sintetizzato tramite la mediana (IRF strutturali) e bande di confidenza percentile (IRF\_low e IRF\_high). Questo processo permette di isolare i tre shock fondamentali del mercato petrolifero e di ottenere IRF e FEVD con interpretazione economica coerente, superando i limiti identificativi del VAR ridotto-forma.

\section{Risultati del modello SVAR a restrizioni di segno}

Una volta stimato il modello VAR ridotto-forma e verificata la debole interpretabilità
economica degli shock identificati tramite decomposizione di Cholesky, si procede alla
stima di un modello SVAR con restrizioni di segno in stile \textcite{Kilian2009} e
\textcite{KilianMurphy2014}. Tale passaggio permette di identificare in modo credibile gli
shock strutturali fondamentali del mercato del petrolio: shock di offerta, shock di domanda
aggregata e shock di domanda precauzionale.

Le restrizioni di segno imposte all'impatto sono le seguenti:

\begin{itemize}
    \item \textbf{Shock di offerta (supply shock)}: la produzione diminuisce e il prezzo del petrolio aumenta ($\Delta\text{Production}<0$, $\Delta\text{WTI}>0$).
    \item \textbf{Shock di domanda aggregata}: l'attività economica aumenta e il prezzo del petrolio aumenta ($\Delta\text{REA}>0$, $\Delta\text{WTI}>0$).
    \item \textbf{Shock di domanda precauzionale}: le scorte commerciali diminuiscono e il prezzo del petrolio aumenta ($\Delta\text{Inventories}<0$, $\Delta\text{WTI}>0$).
\end{itemize}

La procedura di identificazione consiste nel generare rotazioni ortogonali casuali della
fattorizzazione di Cholesky e nel selezionare esclusivamente le matrici di impatto
compatibili con le firme dei segni sopra indicate. L’insieme delle IRF accettate viene
quindi sintetizzato tramite la mediana, ottenendo così le IRF strutturali.

\subsection{Impulse Response Functions strutturali}

Le IRF strutturali del prezzo del petrolio (WTI) ai tre shock fondamentali presentano
caratteristiche coerenti con la teoria economica e con l’evidenza empirica più recente.

\paragraph{Shock di offerta.}
Lo shock di offerta (diminuzione esogena della produzione di petrolio) genera una
risposta immediata e molto ampia del prezzo del petrolio. L’impatto all’impatto è
positivo e significativo, con un effetto di breve periodo che si smorza progressivamente
entro circa cinque mesi. Tale dinamica è coerente con episodi di shock geopolitici,
interruzioni inattese della produzione o interventi restrittivi dell’OPEC. La forma
dell’IRF riproduce fedelmente la dinamica presentata da \textcite{Kilian2009}.

\paragraph{Shock di domanda aggregata.}
Uno shock di domanda aggregata (crescita dell'attività reale globale) produce un aumento
del prezzo del petrolio più contenuto rispetto allo shock di offerta. L’impulso iniziale è
positivo ma modesto, seguito da un temporaneo rimbalzo negativo e da una rapida
riconvergenza verso lo zero. Questo risultato suggerisce che, negli anni recenti, la
domanda globale esercita un ruolo meno dominante nei movimenti del prezzo del petrolio,
in linea con l’evidenza post-2014 che sottolinea un crescente peso degli shock
specifici dell’industria petrolifera rispetto ai cicli economici globali.

\paragraph{Shock di domanda precauzionale.}
Lo shock di domanda precauzionale, definito come una variazione nelle aspettative di
scarsità futura che induce gli operatori ad accumulare scorte oggi, genera un aumento
iniziale del prezzo del petrolio, sebbene di entità moderata. L’effetto si attenua rapidamente,
diventando leggermente negativo nei mesi successivi per poi convergere verso lo zero.
L’impatto relativamente debole di questo shock riflette la minore rilevanza, negli anni
recenti, delle aspettative di scarsità futura rispetto al periodo 1970--2005 analizzato
dalla letteratura classica.

\subsection{Forecast Error Variance Decomposition strutturale}

La scomposizione strutturale della varianza dell’errore di previsione del prezzo del
petrolio fornisce una chiara misura dell’importanza relativa dei tre shock identificati.
A un orizzonte di 24 mesi, i risultati principali sono:

\begin{itemize}
    \item lo \textbf{shock di offerta} spiega circa il \textbf{50.8\%} della variabilità del WTI;
    \item lo \textbf{shock di domanda aggregata} spiega circa il \textbf{15.0\%};
    \item lo \textbf{shock di domanda precauzionale} spiega circa l’\textbf{8.1\%};
    \item la restante quota è attribuibile a innovazioni proprie del prezzo del petrolio
          e a componenti non spiegate dalle tre categorie di shock.
\end{itemize}

Questi valori rappresentano una marcata differenza rispetto ai risultati ottenuti dal VAR
ridotto-forma. In particolare, mentre il modello ridotto-forma attribuiva oltre l’80\% della
variabilità del WTI a innovazioni proprie del prezzo, lo SVAR a restrizioni di segno
redistribuisce gran parte della varianza verso gli shock strutturali, in particolare verso lo
shock di offerta.

\subsection{Discussione}

Il modello SVAR rivela che la volatilità del prezzo del petrolio è dominata dagli shock di
offerta. Gli shock di domanda aggregata svolgono un ruolo secondario, e gli shock di
domanda precauzionale hanno un impatto limitato ma comunque non trascurabile.
Questa evidenza è completamente coerente con la letteratura più recente e con la
finanziarizzazione dei mercati petroliferi.

Nel complesso, il modello SVAR consente di identificare correttamente le tre fonti
strutturali di shock nel mercato del petrolio, superando le limitazioni del VAR
ridotto-forma e fornendo una base quantitativa affidabile per la costruzione di scenari di
rischio e per lo sviluppo di strategie di hedging per gli operatori esposti alle fluttuazioni
del prezzo del petrolio.

\chapter{Introduction}

\section{Motivation and Background}

    Understanding the behaviour of crude oil prices is critical for macroeconomic
    analysis, financial stability and industrial decision-making. Oil remains the world's most
    strategic commodity, influencing inflation, business cycles, transportation networks and
    geopolitical dynamics. Because crude oil serves both as a physical input and as a financial asset,
    its price reflects a complex interaction between real fundamentals, expectations, inventories,
    speculative pressures and policy events. As argued by \textcite{BarskyKilian2002, BarskyKilian2004},
    macroeconomic outcomes associated with oil price movements cannot be understood through
    a simplistic focus on physical supply disruptions alone.

    A central insight of the modern literature is that oil price fluctuations predominantly reflect
    global aggregate demand conditions and changes in precautionary inventories rather than
    supply-side disturbances. In the structural decomposition of \textcite{Kilian2009}, shocks to global
    real economic activity account for the majority of medium-run oil price variation, while
    precautionary demand shocks --- captured through inventory dynamics --- generate sharp price
    spikes during episodes of uncertainty about future availability. More recent contributions
    \parencite{KilianMurphy2014, BaumeisterHamilton2019} confirm these findings and refine the
    understanding of speculative motives and inventory behaviour.

    Despite substantial progress in modelling, a persistent misconception in public and policy discussions is that oil price spikes are mainly supply-driven. Structural evidence, however, contradicts this view: most large post-1990 price surges were demand-led rather than the result of physical shortages \cite{KilianZhou2020}. Historical accounts further reinforce this interpretation, showing that many so-called “supply crises” were in fact driven by shifts in global demand, precautionary behaviour, or broader macroeconomic conditions \cite{Carollo2010}.
     However, the
    empirical evidence overwhelmingly shows that supply shocks are relatively rare, generally small
    in magnitude and have limited persistence. The short-run price elasticity of crude oil supply is
    extremely low, reflecting physical and technological constraints in the extraction sector.
    Consequently, even small shifts in global economic activity or revisions in expectations can
    produce outsized price movements.

    The complexity of the oil market extends beyond macroeconomic theory. For industries
    that rely heavily on refined petroleum products --- particularly aviation --- fluctuations in oil and
    jet fuel prices translate into substantial financial risk. Fuel expenses typically represent between
    20\% and 35\% of airline operating costs \parencite{IATA_FuelCosts2024}. This exposure
    makes airlines highly vulnerable to extreme price movements arising from macroeconomic or
    geopolitical shocks. For this reason, the aviation sector requires robust risk management tools
    capable of mapping crude oil disturbances into jet fuel prices and quantifying the impact on
    cash-flow and budgeting decisions. 
    \newline
    
    Modern risk management increasingly relies on scenario-based methods that integrate
    economic fundamentals. Firms can no longer depend on simple forecasting tools or static
    projections, as such models fail to capture the structural origins of volatility. Instead, firms look
    towards econometric frameworks capable of isolating the underlying drivers of price
    movements and generating both probabilistic scenarios and adversarial stress tests
    \parencite{AndersonKelloggSalant2018, SockinXiong2015}. Structural VAR (SVAR) models, in particular, offer
    a theoretically grounded method for disentangling supply, demand and inventory shocks, while
    copula-based approaches allow researchers to characterise nonlinear dependence and
    tail-risk interactions across shocks. 
\\
\begin{figure}[H]
    \centering
    \includegraphics[width=0.99\textwidth]{fig_stylised_oil_market.png}
    \caption{Stylised Facts of the Oil Market, 1990--2024: Real WTI Price, OECD Activity, Oil Production, Oil Inventories. (\textit{Source}: own elaboration on EIA, FRED and OECD data)}
    \label{fig:stylisedfacts}
\end{figure}

Figure~\ref{fig:stylisedfacts} presents a visual overview of the main variables used throughout
the thesis. The co-movement between global activity and crude oil prices is striking, especially
during episodes such as the 2003--2008 expansion and the COVID-19 contraction. Production
adjusts slowly, while inventories exhibit spikes during periods of heightened uncertainty. These
stylised facts motivate the adoption of a multivariate dynamic model capable of accounting
for these interactions.

\section{Problem Statement}

Despite extensive research, economists and practitioners face several persistent challenges in
modelling and interpreting oil price dynamics.

First, the oil market is inherently multivariate. Real oil prices respond to production, global
business cycles, inventory adjustments and financial market conditions. Univariate models, including autoregressive specifications and static regressions, cannot reproduce these joint dynamics. Formal diagnostic tests confirm the inadequacy of such models: regressions of the
real price of oil on production, activity and inventories exhibit strong serial correlation in the
residuals \parencite{LjungBox1978}, heteroscedasticity \parencite{BreuschPagan1979} and clear departures from
normality \parencite{JarqueBera1987, Massey1951}. Coefficient instability across subsamples further
suggests that relationships are time-varying and regime-dependent \parencite{Brown1975, BaiPerron2003}.

Second, reduced-form VARs, while effective in capturing dynamic interactions, do not provide economically meaningful interpretations of shocks unless a structural identification strategy is imposed. As emphasised by \textcite{Sims1980} and formalised in the oil-market context by \textcite{Kilian2009}, theory-based restrictions are essential for distinguishing supply, aggregate-demand and precautionary-demand innovations. Building on these principles, \textcite{RubioRamirez2010} show how identification schemes can be derived in a fully Bayesian framework. Without economically motivated restrictions, impulse responses cannot be interpreted in terms of underlying structural mechanisms.


Third, extreme events play a central role in the oil market. Structural shocks display
heavy-tailed distributions, meaning that rare but impactful events occur with higher probability
than implied by Gaussian models. Modelling these extremes requires flexible distributional
assumptions and tools capable of capturing tail dependence, such as copulas \parencite{Nelsen2006,
Patton2012}.

Finally, translating structural oil shocks into industrial risk metrics requires bridging two
domains: macroeconometric modelling and corporate finance. Firms need tools that not only
identify the origins of price movements, but also translate them into actionable scenarios,
stress tests and hedging implications. This is particularly relevant for airlines, where fuel-price
risk affects fleet planning, pricing strategies and financial performance.

\section{Theoretical Framework}

The theoretical backbone of this thesis is the structural VAR framework originally proposed by
\textcite{Kilian2009} and later refined by \textcite{KilianMurphy2014}. In this framework, the global oil market
is driven by three fundamental shocks:

\begin{enumerate}
    \item \textbf{Flow supply shocks} --- unexpected changes in global crude oil production.
    \item \textbf{Aggregate demand shocks} --- shifts in global real economic activity, reflecting industrial
    demand for commodities.
    \item \textbf{Precautionary or inventory demand shocks} --- changes in expectations about future oil
    availability, inferred through adjustments in inventories.
\end{enumerate}

Flow supply shocks typically have limited and short-lived effects on oil prices due to the
inelasticity of near-term production. By contrast, aggregate demand shocks produce large and
persistent price movements, reflecting the procyclical nature of industrial commodity markets.
Precautionary demand shocks capture periods of heightened uncertainty, during which firms and
speculators increase inventory holdings in anticipation of potential future shortages
\cite{Kilian2009,BaumeisterHamilton2019}.


\begin{figure}[t!]
\centering
\begin{tikzpicture}[
    >=Latex,
    box/.style={
        rectangle,
        draw,
        rounded corners,
        minimum height=1.1cm,
        minimum width=4.8cm,
        align=center,
        thick,
        fill=gray!5
    },
    arrow/.style={->, very thick}
]

% Colonna shock a sinistra
\node[box] (supply)    at (0, 2.0) {Flow Supply Shock};
\node[box] (demand)    at (0, 0.0) {Aggregate Demand Shock};
\node[box] (inventory) at (0,-2.0) {Precautionary Shock};

% VAR al centro
\node[box, minimum width=5.0cm] (var) at (6,0) {Structural VAR System};

% Prezzo a destra
\node[box, minimum width=4.8cm] (price) at (11,0) {Real Price of Oil};

% Frecce shock -> VAR
\draw[arrow] (supply.east)    to[bend left=15]  (var.north west);
\draw[arrow] (demand.east)    --                (var.west);
\draw[arrow] (inventory.east) to[bend right=15] (var.south west);

% Freccia VAR -> prezzo
\draw[arrow] (var.east) -- (price.west);

\end{tikzpicture}
\caption{Conceptual structure of the oil market SVAR model.}
\label{fig:varconcept}
\end{figure}




Figure~\ref{fig:varconcept} provides a schematic representation of the structural model. Each
shock affects oil prices through distinct channels, though real-world dynamics often involve
interactions between shocks. The structural VAR allows researchers to quantify these effects
through impulse response functions (IRFs) and forecast error variance decompositions (FEVD).

Beyond the VAR, this thesis incorporates distributional analysis of structural shocks, building
on the observation that heavy tails and nonlinear dependence are key features of global
commodity markets. Understanding tail dependencies is crucial for designing robust stress
tests and risk scenarios.

\section{Research Objectives}

This thesis pursues four overarching objectives:

\begin{enumerate}
    \item \textbf{To construct a harmonised and comprehensive monthly dataset (1990--2024)} including
    real crude oil prices, global crude oil production, OECD industrial activity and U.S. petroleum
    and product inventories, using official sources such as the EIA and the Federal Reserve
    \parencite{WTI_FRED, Inventories_EIA, OilProduction_EIA}.
    
    \item \textbf{To estimate and validate a reduced-form VAR model} that captures the dynamic
    interactions between these variables, supported by extensive diagnostic testing
    \parencite{Lutkepohl2005}.
    
    \item \textbf{To identify structural shocks using sign restrictions and elasticity bounds}, following the
    methodology of \textcite{KilianMurphy2014}, and to characterise their impulse responses and
    variance contributions.
    
    \item \textbf{To translate structural oil shocks into industrial risk metrics}, including Monte Carlo
    scenario generation, Hormuz-type stress testing and hedging implications for a commercial
    airline exposed to jet fuel price fluctuations.
\end{enumerate}

\section{Methodological Contribution}

This thesis makes several methodological contributions to the empirical analysis of oil markets
and their industrial applications:

\begin{itemize}
    \item \textbf{Data construction and proxy evaluation.}  
    It provides a harmonised dataset covering more than three decades of global oil market
    activity. A key contribution is the evaluation of alternative proxies for global real activity,
    comparing OECD-based indicators to the \textit{REA} index \parencite{IGREA_DallasFed, Kilian2019}.
    
    \item \textbf{Structural identification.}  
    The thesis implements a robust identification strategy based on sign and elasticity
    restrictions, ensuring economically meaningful structural shocks consistent with the oil
    market literature \parencite{KilianMurphy2014, BaumeisterHamilton2019}.
    
    \item \textbf{Distributional characterisation of shocks.}  
    It documents heavy-tailed marginal distributions and copula-based dependence structures
    across structural shocks, revealing substantial tail dependence between demand and
    precautionary disturbances \parencite{Nelsen2006, Patton2012}.
    
    \item \textbf{Stress-testing framework.}  
    The thesis develops a scenario generation and stress-testing framework grounded in
    structural econometrics and applicable to industrial decision-making, particularly the
    aviation sector.
    
    \item \textbf{Application to fuel risk management.}  
    The thesis bridges macroeconomic modelling with corporate risk management by mapping
    structural oil shocks into jet fuel price paths and assessing the impact on an airline's fuel
    cost exposure.
\end{itemize}

\begin{table}[t!]
\centering
\caption{Key structural mechanisms in oil markets.}
\label{tab:mechanisms}
\setlength{\tabcolsep}{6pt}
\renewcommand{\arraystretch}{1.25}
\begin{tabular}{p{3.2cm}p{7.2cm}p{3.4cm}}
\toprule
\textbf{Mechanism} & \textbf{Description} & \textbf{References} \\
\midrule
Flow supply        & Slow adjustment due to extraction and investment constraints & Kilian \& Murphy (2014) \\
Aggregate demand   & Driven by the global business cycle and industrial activity & Barsky \& Kilian (2002, 2004) \\
Inventory demand   & Reflects expectations about future availability             & Alquist, Bhattarai \& Coibion (2019) \\
Speculation        & Amplifies short-run volatility                              & Sockin \& Xiong (2015) \\
\bottomrule
\end{tabular}
\end{table}


\section{Structure of the Thesis}

The thesis is organised into five chapters:

\begin{itemize}
    \item \textbf{Chapter 2} introduces the dataset, details preprocessing steps and presents the
    econometric methodology, including stationarity analysis and the specification of the
    reduced-form VAR model.
    
    \item \textbf{Chapter 3} presents the estimation results of the VAR model and the identification of
    structural shocks. It reports impulse responses, variance decompositions and the
    transmission of shocks to oil prices.
    
    \item \textbf{Chapter 4} examines the empirical distributions of structural shocks, estimates copula
    models to capture nonlinear dependence and applies the structural results to scenario
    generation, stress testing and hedging.
    
    \item \textbf{Chapter 5} concludes by summarising the main findings, discussing limitations and
    proposing avenues for future research.
\end{itemize}

These considerations motivate the multivariate modelling framework employed in the remainder of this thesis.



\chapter{Data and Econometric Methodology}
\label{ch:data-methods}

\section{Overview}

This chapter documents the construction of the dataset and the econometric
framework that underpins the empirical analysis of the oil market and the
subsequent stress–testing and hedging applications. 
The objective is twofold. 
First, to provide a transparent account of each transformation applied to the
raw data, so that the analysis can be replicated and extended.
Second, to show why a multivariate dynamic specification --- in particular a
structural vector autoregression (SVAR) --- is required to model the joint
behaviour of oil prices, U.S.\ production, OECD activity and inventories,
instead of static regression models.

The empirical work is based on monthly data from January 1990 to December
2024. 
The data are sourced from standard repositories in the energy economics
literature: the Federal Reserve Bank of St.\ Louis (FRED), the U.S.\ Energy
Information Administration (EIA), the Dallas Fed Globalization Institute and
OECD statistics.\footnote{See \textcite{WTI_FRED, CPI_FRED, IGREA_DallasFed,
OilProduction_EIA, Inventories_EIA, JetFuel_EIA} for the official
documentation of the underlying series.} 
All series are imported, cleaned and merged by the \textsc{MATLAB} script
\texttt{build\_oil\_dataset.m}, which creates three core objects:

\begin{enumerate}[label=(\roman*)]
    \item a timetable \texttt{All} containing the main variables in levels
    (nominal and real), at a common monthly frequency;
    \item a timetable \texttt{All\_d} with stationary transformations
    (log-differences and standardised cycles) used in the pre--VAR diagnostics;
    \item a timetable \texttt{ALL\_VAR} with the transformed variables entering
    the structural VAR.
\end{enumerate}

Figure~\ref{fig:var-series} anticipates the final VAR variables, while the
remaining figures in this chapter summarise the unit root and diagnostic tests,
the lag selection and the in-sample fit of the VAR model.
Tables~\ref{tab:data_sources}--\ref{tab:lag_selection} report the
corresponding numerical results.

\section{Data Sources}
\label{sec:data-sources}

\subsection{Real WTI Crude Oil Price}

The benchmark price variable is the West Texas Intermediate (WTI) spot price at
Cushing, Oklahoma, obtained from the FRED series \texttt{MCOILWTICO}.\footnote{See
\textcite{WTI_FRED}.}
The original data are quoted in U.S.\ dollars per barrel and correspond to
monthly averages of daily prices.
To account for inflation, nominal WTI prices are deflated by the U.S.\ Consumer
Price Index for All Urban Consumers (CPI-U, series \texttt{CPIAUCSL}).\footnote{See
\textcite{CPI_FRED}.}
The real price index is constructed as
\[
    \text{WTI}^{\text{real}}_t =
    100 \times \frac{\text{WTI}^{\text{nom}}_t}{\text{CPI}_t},
\]
so that the base value of the index is 100 in the first observation of the
sample.
This transformation is consistent with the treatment of real oil prices in
\textcite{Kilian2009, KilianMurphy2014, BaumeisterHamilton2019}.

Although Brent is often employed as the benchmark for global crude oil pricing,
this study adopts the real WTI price as the reference series. This choice is
consistent with the U.S.-centred nature of the dataset — which combines WTI
prices with U.S.\ production and inventory data — and with the focus on risk
management for an airline hedging exposures linked to WTI-denominated futures
and spot prices. The reliance on WTI rather than Brent implies that the
resulting price dynamics are most directly informative for the U.S.\ oil
market; this limitation is discussed further in the concluding chapter.
\subsection{U.S.\ Crude Oil Production}

The production variable is used to proxy the flow supply of crude oil.
In the baseline dataset, production is measured using monthly U.S.\ crude oil
output from the EIA International Energy Statistics, expressed in thousand
barrels per day.\footnote{See \textcite{OilProduction_EIA} for the reference to
the EIA production series.}
U.S.\ production is a natural driver of WTI dynamics because the benchmark is
physically delivered in the United States and domestic supply responds
strongly to technological and regulatory shifts, such as the shale oil
expansion in the mid-2000s.

Unlike \textcite{KilianMurphy2014}, who rely on world crude oil production as a
measure of global flow supply, this study uses U.S.\ output as the supply
indicator. Consequently, the identified supply disturbance in the SVAR should
be interpreted as a U.S.-centred WTI-market supply shock rather than as a fully
global flow supply shock. This U.S.\ focus is consistent with the broader
dataset design, which combines WTI prices with U.S.\ production and inventory
data.

Following the literature on oil market VAR models, production enters the
structural specification in logarithms, allowing shocks to be interpreted as
percentage changes in supply \parencite{KilianMurphy2014,KilianZhou2018}.

\subsection{OECD Industrial Activity}

Global demand conditions are proxied by an index of industrial activity for the
OECD aggregate. 
The underlying series is an industrial production index compiled by OECD, with
base year equal to 2015 and monthly frequency.
In the early SVAR literature, global demand for industrial commodities is
typically measured by the Kilian index of global real economic activity
(REA), built from ocean freight rates \parencite{Kilian2009}. 
However, subsequent work has highlighted several shortcomings of REA,
including sensitivity to shipping market reforms, structural breaks around
China's WTO accession and limited information content in the post-2010 period
\parencite{Kilian2019, KilianZhou2018}. 
In the present dataset, REA-based specifications display weaker statistical
properties than OECD industrial activity: OLS regressions with REA produce
less stable coefficients and more severe residual autocorrelation relative to
OECD, and VAR models with REA show poorer diagnostics. 
For these reasons, OECD is adopted as a more robust proxy for global demand
conditions in the identification of aggregate demand shocks, in line with the
recent re-assessment of demand indicators in \textcite{Kilian2019}.

\subsection{U.S.\ Oil Inventories}

Crude oil and refined product inventories play a central role in models where
precautionary demand and speculative storage link expectations about future
scarcity to current prices \parencite{KilianMurphy2014, BaumeisterHamilton2019}. 
Inventory data are obtained from the EIA's International Energy Database as the
total end-of-period stocks of crude oil and petroleum products in the United
States.\footnote{See \textcite{Inventories_EIA}.} 
The data are expressed in million barrels and reported at monthly frequency.
As discussed in \textcite{KilianMurphy2014}, it is not the level of inventories
per se but unexpected changes in stocks that identify precautionary demand
shocks. 
Consequently, in the structural VAR inventories are transformed to logarithms
and subsequently filtered to obtain stationary deviations around trend.

\subsection{Jet Fuel Prices (U.S.\ Gulf Coast)}

For the hedging application in later chapters, jet fuel prices are also
required. 
The relevant series is the kerosene-type jet fuel spot price at the U.S.\ Gulf
Coast, expressed in dollars per gallon and reported at monthly frequency by the
EIA.\footnote{See \textcite{JetFuel_EIA}.}
These data are imported and stored in the variable \texttt{JetFuel\_USGC} in
\texttt{All}. 
While jet fuel does not enter the VAR directly, it is linked to real WTI
prices through a pass-through regression in the hedging module.

\subsection{Inflation Measure and Deflation}

The CPI-U index serves both as a deflator for nominal oil prices and as a
proxy for general price-level movements in the U.S.\ economy
\parencite{CPI_FRED}. 
After reshaping the CPI series to monthly frequency and aligning the date
conventions, the \texttt{build\_oil\_dataset.m} script constructs the real WTI
series and then discards the nominal WTI to avoid redundancy. 
The CPI series itself does not enter the VAR system but is retained for
consistency checks and potential extensions involving real interest rates or
inflation dynamics.

\subsection{Summary of Data Sources}
A detailed description of data construction, frequency alignment and
pre-VAR diagnostics is provided in Appendix~A.

Table~\ref{tab:data_sources} summarises all variables employed in the analysis,
their definitions, units of measurement and original sources.
\begin{table}[H]
\centering
\begin{threeparttable}
\caption{Data sources and variable definitions.}
\label{tab:data_sources}
\setlength{\tabcolsep}{6pt}
\renewcommand{\arraystretch}{1.25}
\begin{tabular}{p{3.0cm}p{6cm}p{3cm}p{2.0cm}}
\toprule
\textbf{Variable} & \textbf{Definition} & \textbf{Source} & \textbf{Frequency} \\
\midrule
WTI\_real & Real West Texas Intermediate crude oil price (index, 1990=100) &
FRED, EIA \parencite{WTI_FRED} & Monthly \\

Production & U.S.\ crude oil production (thousand barrels per day) &
EIA \parencite{OilProduction_EIA} & Monthly \\

OECD (IPI)\tnote{*} &
OECD industrial activity indicator, Dallas Fed &
OECD, Dallas Fed \parencite{IGREA_DallasFed, Kilian2019} &
Monthly \\

Inventories & U.S.\ crude oil and petroleum product ending stocks (million barrels) &
EIA \parencite{Inventories_EIA} & Monthly \\

JetFuel\_USGC \tnote{**} & Kerosene-type jet fuel spot price, U.S.\ Gulf Coast (USD per gallon) &
EIA \parencite{JetFuel_EIA} & Monthly \\

CPI \tnote{**} & Consumer Price Index for All Urban Consumers (1982–84=100) &
BLS/FRED \parencite{CPI_FRED} & Monthly \\
\bottomrule
\end{tabular}

    % Definizione del testo della nota all'interno di tablenotes
    \begin{tablenotes}
        \item[*] Industrial Production Index
    \end{tablenotes}
   \begin{tablenotes}
        \item[**] They are not part of the core VAR system but are needed for the hedging module / deflation and consistency checks
    \end{tablenotes}

    \end{threeparttable}
    % Fine dell'ambiente threeparttable
\end{table}



\section{Preprocessing and Data Harmonisation}
\label{sec:preprocessing}

All data management is implemented in \texttt{build\_oil\_dataset.m}. 
The goal is to harmonise the series in terms of time index, frequency and
numeric format, and to prepare both stationary and non-stationary
transformations consistent with the subsequent VAR specification.

\subsection{Date Alignment and Monthly Frequency}

Source files differ in their original date conventions: some use calendar dates
(``yyyy-mm-dd''), others indicate only year and month (``yyyy-mm''), and some
rely on textual month labels. Each series is first converted to a
\texttt{table} with explicit date and value columns and then mapped to a
\texttt{datetime} vector using appropriate input formats and locale settings,
so that heterogeneous date formats and locales are harmonised in a consistent
way. All dates are shifted to the first day of the corresponding month to
ensure that the time index is strictly monthly and comparable across series.

The individual series are then recast as \texttt{timetable} objects and
synchronised by intersecting their support, so that the common sample only
retains months in which all variables are observed. This procedure implicitly
drops months with missing entries in at least one series and yields a balanced
panel of monthly observations from 1990:1 to 2024:12.

\subsection{Numeric Cleaning}

Many of the raw files encode numbers using commas as thousands separators or
as decimal separators, depending on the local convention. To avoid parsing
issues, all numeric entries are first converted to strings and then cleaned
with a simple regular-expression routine that removes non-numeric separators
before being cast to double precision. This ensures that subsequent arithmetic
operations produce consistent results and that no artefacts arise from
locale-specific decimal formats.

\subsection{Transformations and Construction of \texttt{All\_d} and \texttt{ALL\_VAR}}

After converting the nominal WTI series into real terms using the CPI deflator, the
dataset \texttt{All} stores the variables at cleaned monthly frequency:
\texttt{WTI\_real}, \texttt{Production}, \texttt{OECD}, \texttt{Inventories},
\texttt{JetFuel\_USGC} and \texttt{CPI}. 
Two parallel transformations are then implemented in \textsc{MATLAB}.

\paragraph{Stationary transformations for VAR and preliminary diagnostics.}
All series intended for OLS, unit root analysis and VAR estimation are transformed into
stationary representations. In particular,
\begin{align*}
    \Delta \log \text{WTI}^{\text{real}}_{t} &=
        \log \text{WTI}^{\text{real}}_{t} - \log \text{WTI}^{\text{real}}_{t-1}, \\
    \Delta \log \text{Prod}_{t} &=
        \log \text{Prod}_{t} - \log \text{Prod}_{t-1}, \\
    \Delta \log \text{Inv}_{t} &=
        \log \text{Inv}_{t} - \log \text{Inv}_{t-1}, \\
    \Delta \text{OECD}_{t} &= \text{OECD}_{t} - \text{OECD}_{t-1}.
\end{align*}

All differenced series are then standardised to zero mean and unit variance, and stored
in the structure \texttt{All\_d} with the suffix \texttt{\_DL}
(\texttt{WTI\_real\_DL}, \texttt{Production\_DL}, \texttt{OECD\_DL},
\texttt{Inventories\_DL}, \texttt{JetFuel\_DL}). 
These stationary $z$–score transforms are used in the pre--VAR OLS regressions,
unit root tests, rolling-window diagnostics (Section~2.5) and, crucially, as inputs for
the reduced-form VAR.

\paragraph{VAR state vector and construction of \texttt{ALL\_VAR}.}
In line with the econometric evidence in Section~\ref{sec:unit-root}, the reduced-form
VAR is estimated directly on the stationary $\Delta\log$ (or difference) transforms,
rather than on log levels.
The four-dimensional state vector is defined as
\[
    y_{t} =
    \begin{bmatrix}
        \text{Production\_DL}_{t} \\
        \text{OECD\_DL}_{t} \\
        \text{WTI\_real\_DL}_{t} \\
        \text{Inventories\_DL}_{t}
    \end{bmatrix},
\]
where the ordering reflects the economic structure of the oil market:
flow supply, aggregate demand, spot price and precautionary/storage
demand.\footnote{This ordering is maintained throughout the analysis: reduced-form
VAR, Cholesky benchmark identification and the sign-restricted SVAR in
Chapter~3.}
For convenience, these four series are stacked in the matrix \texttt{ALL\_VAR}, which
contains the stationary, standardised VAR data used in all subsequent estimations.

Figure~\ref{fig:var-series} reports the trajectories of the VAR variables in
$\Delta\log$ (or difference) form, rescaled to zero mean and unit variance.
The four transformed series display pronounced low-frequency swings and episodic
extremes (2008 crisis, COVID-19, major geopolitical shocks), but no visible
drift, consistent with the unit root behaviour in levels documented in
Section~\ref{sec:unit-root}.

\begin{figure}[t]
    \centering
    \includegraphics[width=1.0\textwidth]{immagine.png}
    \caption{Stationary VAR series, 1990--2024.
    The panels display the standardised $\Delta\log$ real WTI price,
    $\Delta\log$ crude oil production, $\Delta$ OECD industrial activity and
    $\Delta\log$ petroleum inventories (all series transformed to zero mean and
    unit variance).}
    \label{fig:var-series}
\end{figure}

\section{Stationarity and Unit Root Tests}
\label{sec:unit-root}

Before specifying the VAR in log levels, it is necessary to assess the order of
integration of each series and ensure that the resulting system is
econometrically well behaved. Standard unit root tests are therefore applied to
both the original level series stored in \texttt{All} and to the stationary
transformations in \texttt{All\_d}.

Augmented Dickey–Fuller (ADF) tests are run for each variable in levels,
including a constant and a linear trend. For the real WTI price,
production and inventories, the unit root null cannot be rejected at
conventional significance levels, in line with the literature documenting
non-stationarity in real oil prices and macroeconomic aggregates.
In contrast, the OECD activity index appears trend–stationary, with the
ADF statistic rejecting the unit root null at the 1\% level (Table~\ref{tab:adf_levels}).
Given the mixed evidence and for consistency across variables, the subsequent
analysis is conducted on transformed series.

\begin{table}[H]
\centering
\caption{ADF unit root tests for variables in levels.}
\label{tab:adf_levels}
\begin{tabular}{lccc}
\toprule
\textbf{Series} & \textbf{Specification} & \textbf{ADF statistic} & \textbf{p-value} \\
\midrule
WTI\_real      & constant + trend & -2.083 & 0.5514 \\
Production     & constant + trend & -1.461 & 0.8411 \\
OECD           & constant + trend & -4.060 & 0.0082 \\
Inventories    & constant + trend & -1.020 & 0.9388 \\
\bottomrule
\end{tabular}
\end{table}


    When the same tests are applied to first differences (or, for OECD activity,
    to the cyclical component), the unit root null is strongly rejected for all
    series, with p-values around 0.001, as shown in Table ~ \ref{tab:adf_diffs}.
    Given the strong evidence of unit roots in the level series, the VAR is specified in terms of stationary log-differences (and, for OECD activity, detrended cyclical components). This choice eliminates stochastic trends from the state vector and preserves the validity of standard VAR inference without imposing explicit cointegration restrictions. Long-run effects on the levels of the variables are then recovered by cumulating the impulse responses of log-differences, as discussed in Chapter 3.


\begin{table}[H]
\centering
\caption{ADF unit root tests for transformed series.}
\label{tab:adf_diffs}
\begin{tabular}{lccc}
\toprule
\textbf{Series} & \textbf{Specification} & \textbf{ADF statistic} & \textbf{p-value} \\
\midrule
WTI\_real\_DL          & constant & -15.863 & 0.0010 \\
Production\_DL         & constant & -23.385 & 0.0010 \\
OECD\_DL / OECD\_cycle & constant & -14.844 & 0.0010 \\
Inventories\_DL        & constant & -13.960 & 0.0010 \\
\bottomrule
\end{tabular}
\end{table}

Additional test statistics and residual diagnostics are reported in
Appendix~A.


\section{Exploratory OLS Evidence and Limitations}
\label{sec:ols-prevar}

As a preliminary step, static regression models are estimated to gauge whether
simple linear relationships between real oil prices and fundamentals can
adequately describe the data. 
These regressions are implemented in \texttt{ols\_prevar\_check.m} and form the
basis for arguing in favour of a dynamic multivariate specification.
\subsection{Baseline Static Regression}

As a starting point, a simple static specification relates the log real WTI price
to fundamental variables in levels. Specifically,
\[
    \log \text{WTI}^{\text{real}}_t = 
    \alpha + \beta_1 \log \text{Prod}_t +
    \beta_2 \text{OECD}_t +
    \beta_3 \log \text{Inv}_t + u_t,
\]
where all regressors enter contemporaneously and no lags are included.
In line with \textcite{Kilian2009, BaumeisterHamilton2019}, this static
log–linear regression explains only a limited share of the variation in real oil
prices: the adjusted $R^2$ remains modest, and the residuals display serial correlation,
heteroskedasticity and non–normality, as indicated by the Ljung--Box $Q$-statistic,
the Breusch--Pagan test and the Jarque--Bera normality test reported in
Table~\ref{tab:ols_diag}.
These diagnostics highlight that a purely contemporaneous specification is
inadequate and motivate the move to a multivariate dynamic VAR framework.



\begin{table}[H]
\centering
\caption{Diagnostic tests for static OLS regression of real WTI.}
\label{tab:ols_diag}
\renewcommand{\arraystretch}{1.1}
\begin{tabular}{lccc}
\toprule
\textbf{Test} & \textbf{Statistic} & \textbf{p-value} & \textbf{Conclusion} \\
\midrule
Ljung--Box (12 lags) & 34.41  & 0.0006   & serial correlation present   \\
Breusch--Pagan       & 27.74  & 0.0000   & heteroscedasticity present   \\
Jarque--Bera         & 197.17 & $<0.001$ & residuals non-normal         \\
\bottomrule
\end{tabular}
\end{table}


\subsection{Structural Instability and Rolling Windows}

To assess the stability of the OLS relationship over time,
\texttt{ols\_windows.m} re-estimates the static regression on rolling subsamples
(e.g.\ 10- or 15-year windows). 
The resulting sequences of coefficients reveal substantial time variation:
production and inventory elasticities change sign across subsamples, and the
effect of OECD on real WTI becomes weaker or stronger depending on the period.
This is consistent with evidence from structural break tests such as those of
\textcite{Brown1975, BaiPerron2003}, which typically identify multiple
breakpoints in macroeconomic relationships over the last four decades.

The instability of static coefficients, combined with unsatisfactory residual
diagnostics, confirms that the oil market is governed by dynamic interactions
and time-varying responses.
This motivates the move to a VAR framework, where lagged dependencies and
feedback effects can be explicitly modelled.

\section{VAR Model Specification}
\label{sec:var-spec}

\subsection{Reduced-Form VAR}

The baseline multivariate model is a reduced-form VAR in log levels (for oil prices, production and inventories) and in levels for OECD activity:
\[
    y_t = c + A_1 y_{t-1} + \cdots + A_p y_{t-p} + u_t,
\]
where $y_t = (\text{OECD\_level}_t, \text{Prod\_log}_t, \text{WTI\_log}_t,
\text{Inv\_log}_t)'$, $c$ is a vector of intercepts, $A_i$ are $4\times4$ lag
matrices, and $u_t$ is a vector of reduced-form innovations with covariance
matrix $\Sigma_u$. 
The model is estimated by ordinary least squares equation-by-equation, which is
efficient under the assumption of Gaussian errors and homoscedasticity
\parencite{Sims1980, Lutkepohl2005}.

The choice of working with log levels --- as opposed to differenced series ---
follows the practice in the structural oil market literature and allows for the
possibility of long-run co-movements between oil prices and fundamentals
\parencite{Kilian2009, BaumeisterHamilton2019}. 
At the same time, a sufficiently rich lag structure is included to ensure that
the VAR residuals are approximately white noise and that the system is
dynamically stable.
A detailed description of the reduced-form VAR specification, lag selection
and stability diagnostics is provided in Appendix~B.


\subsection{Lag Length Selection}

Lag length is selected empirically by combining dynamic stability and residual
whiteness. 
In the \texttt{var\_main.m} script, a sequence of VAR($p$) models with
$p = 1,\dots,15$ is estimated, and for each specification two diagnostics are
recorded: (i) whether all eigenvalues of the companion matrix lie inside the
unit circle and (ii) the minimum $p$-value of a Ljung--Box portmanteau test for
serial correlation in the residuals of the real WTI equation, based on the
stationary representation in \texttt{All\_d}. 
The results are summarised in Table~\ref{tab:lag_selection}.

\begin{table}[H]
\centering
\caption{Empirical VAR lag selection based on stability and Ljung--Box test on residuals (\texttt{All\_d}, equation for real WTI).}
\label{tab:lag_selection}

\begin{tabular}{lcc}
\toprule
\textbf{Lag $p$} & \textbf{Stable} & $\boldsymbol{\min LB}$ \textbf{$p$-value} \\
\midrule
 1  & Yes & 0.0000 \\
 2  & Yes & $4.65\times10^{-13}$ \\
 3  & Yes & $9.61\times10^{-12}$ \\
 $\vdots$ & $\vdots$ & $\vdots$ \\   % <-- salta i lags intermedi
11  & Yes & $2.92\times10^{-8}$  \\
12  & Yes & 0.7523 \\
13  & Yes & 0.5111 \\
\bottomrule
\end{tabular}

\vspace{0.25cm}
\raggedright\footnotesize
Note: Although only selected lag orders are reported, all VAR($p$) specifications
up to $p=15$ are dynamically stable. For $p \le 11$ the minimum Ljung--Box
$p$-values are effectively zero, indicating strong residual autocorrelation.
Starting from $p = 12$, the $p$-values exceed conventional significance
thresholds, signalling that remaining serial dependence is modest. On this
basis, a VAR(12) is adopted as the baseline specification.
\end{table}



All VAR($p$) specifications turn out to be dynamically stable. 
However, for $p \leq 11$ the Ljung--Box $p$-values are effectively zero,
indicating strong residual autocorrelation and insufficient dynamic structure.
Starting from $p = 12$, the portmanteau $p$-values exceed conventional
significance thresholds, signalling that the remaining serial dependence is
modest and compatible with a well-specified VAR.

On this basis, a VAR(12) is adopted as the baseline specification. 
This choice is also consistent with the monthly frequency of the data and with
the lag length commonly used in the structural oil market literature
\parencite{Kilian2009, KilianMurphy2014}, while preserving a reasonable number
of degrees of freedom for estimation and subsequent structural analysis.

\subsection{Residual Diagnostics and Stability}

After estimating the VAR(12), standard diagnostics are carried out to ensure
that the reduced-form specification is adequate. 
Portmanteau tests for serial correlation in the residuals confirm that the
dynamic structure absorbs most temporal dependence, with no evidence of
remaining autocorrelation at conventional lags. 
Tests for normality based on Jarque--Bera and Kolmogorov--Smirnov statistics,
applied to the standardized residuals, reveal departures from Gaussianity and
some signs of heteroscedasticity, which is not surprising given the presence of
large oil price swings and episodic volatility clustering. 
Nevertheless, the VAR residuals are sufficiently well behaved for the purpose
of structural identification and impulse response analysis.

Stability is examined by inspecting the eigenvalues of the companion matrix:
all lie strictly inside the unit circle, confirming that the VAR is dynamically
stable. 
This property is crucial for the interpretation of impulse responses and for
the generation of stress-test scenarios in later chapters.

\subsection{In-Sample Fit}

To provide a sense of how well the VAR captures the dynamics of the real oil
price, the script \texttt{var\_fit\_vs\_actual\_WTI.m} computes one-step-ahead
in-sample forecasts for the VAR(12) and compares the fitted values of
\texttt{WTI\_log} with the observed series. 
Figure~\ref{fig:var-fit-wti} plots the actual and fitted values. 
The VAR tracks medium-run movements in real WTI reasonably well, including the
run-up in the early 2000s, the collapse during the global financial crisis, and
the shale-driven adjustments in the 2010s. 
Short-lived spikes and collapses are not perfectly matched, reflecting the
limits of linear Gaussian models in capturing extreme events, but the overall
fit is markedly superior to that of the static OLS regression in
Section~\ref{sec:ols-prevar}.

\begin{figure}[H]
    \centering
      \includegraphics[width=1\textwidth]{fig_var_fit_wti.pdf}
    \caption{In-sample VAR(12) fit vs actual log real WTI price. The VAR
    captures medium-run fluctuations in the real oil price, while short-lived
    extremes remain harder to match.}
    \label{fig:var-fit-wti}
\end{figure}

\section{Structural Identification Strategy}
\label{sec:identification}

The reduced-form VAR alone does not provide economically meaningful
interpretations of shocks, since the innovations $u_t$ are linear combinations
of underlying structural disturbances. 
Following the oil market literature, the thesis adopts a structural VAR (SVAR)
approach with sign and elasticity restrictions to identify three economically
interpretable shocks: a (US-centred) flow supply shock, an aggregate demand
shock and a precautionary (inventory) demand shock
\parencite{Kilian2009,KilianMurphy2014,BaumeisterHamilton2019}.

Let $\varepsilon_t$ denote the vector of structural shocks, and $B$ the
contemporaneous impact matrix such that
\[
    u_t = B \varepsilon_t, \qquad \Sigma_u = B B'.
\]
Identification is achieved by imposing sign restrictions on the impulse
responses of $y_t$ to each column of $B$ over a short horizon, combined with
bounds on the short-run price elasticities of supply and demand. The procedure
builds on the framework and formal identification results of
\textcite{RubioRamirez2010}. The specific numerical ranges for the elasticity
bounds are reported in Section~3.3.2.

A flow supply shock is required to reduce crude oil production and increase
the real oil price on impact, while its effect on inventories is either
negative or slightly positive, reflecting the drawdown of stocks in response to
supply shortfalls. Since the supply indicator in this thesis is US crude oil
output rather than world production, this disturbance is best interpreted as a
supply shock to the US WTI market, not as a fully global flow supply shock in
the sense of \textcite{KilianMurphy2014}. An aggregate demand shock must
increase OECD activity, production and the oil price jointly, consistent with
strong global demand for industrial commodities. A precautionary demand shock,
in turn, is characterised by an increase in inventories and real oil prices,
while its effect on production and OECD activity is more muted
\parencite{KilianMurphy2014}. Plausible bounds on the short-run price
elasticities of supply and demand further constrain the admissible
decompositions of $\Sigma_u$.

The actual implementation of sign restrictions is carried out in
\texttt{svar\_sign\_restrictions.m}, which draws candidate orthogonal matrices,
rotates the Cholesky factor of $\Sigma_u$, and retains only those rotations
that satisfy the imposed sign and elasticity conditions over the chosen
horizon. This yields a distribution of admissible impulse responses rather than
a single point estimate, reflecting the partial nature of identification and
addressing concerns raised by \textcite{BaumeisterHamilton2019} about
overconfident structural interpretations.

The identification strategy builds directly on the specification choices
documented in this chapter: the selection of OECD activity as a robust demand
proxy, the log-level transformation of WTI, production and inventories, and the
adoption of a sufficiently rich lag structure to absorb unit root behaviour and
short-run dynamics. Without these preparatory steps, the structural analysis in
the subsequent chapter would rest on a fragile econometric foundation.
\chapter{Empirical Analysis of Oil Market Dynamics}
\label{ch:empirical}

\section{Overview}

This chapter presents the empirical analysis of the oil market based on the multivariate dynamic framework developed in Chapter~\ref{ch:data-methods}. Building on the harmonised monthly dataset and the VAR(12) specification constructed through \texttt{build\_oil\_dataset.m} and \texttt{var\_main.m}, the objective is to identify and quantify the structural disturbances driving the real price of oil.

The analysis follows the structural VAR (SVAR) approach of \textcite{Kilian2009} and \textcite{KilianMurphy2014}, using sign and elasticity restrictions to identify three economically meaningful shocks: a flow supply shock, an aggregate demand shock, and a precautionary demand shock. Impulse responses, forecast error variance decompositions (FEVDs) and historical decompositions are used to characterise the transmission of these shocks. The final section maps structural shocks into WTI price trajectories, laying the groundwork for the scenario-based stress tests in Chapter~\ref{ch:stress}.

In addition to the VAR/SVAR analysis reported in this chapter, Appendix~B
provides an exploratory study of the marginal distributions and copula-based
dependence structure of the identified structural shocks.

\section{Reduced-Form VAR Results}
\label{sec:var-results}

\subsection{Estimation and Diagnostics}

The reduced-form VAR(12) is estimated on the stationary vector
\[
x_t =
\begin{bmatrix}
\Delta \log \text{Prod}_t \\
\text{OECD\_cycle}_t \\
\Delta \log \text{WTI}_t \\
\Delta \log \text{Inv}_t
\end{bmatrix},
\]
whose components correspond to the standardised transformations
\texttt{Production\_DL}, \texttt{OECD\_cycle}, \texttt{WTI\_real\_DL} and
\texttt{Inventories\_DL} in \texttt{All\_d}.

Residual diagnostics for the VAR(12) indicate (Table~\ref{tab:lag_selection},
Table~\ref{tab:var-fit}, Figure~\ref{fig:var-stability} and
Appendix~B}):

\begin{itemize}
    \item no significant residual autocorrelation, as Ljung--Box
    portmanteau $p$-values are well above conventional thresholds;
    \item all companion-matrix eigenvalues strictly inside the unit circle,
    confirming dynamic stability;
    \item mild departures from normality (Jarque--Bera tests) and evidence
    of heavy tails in the residual distribution of the WTI equation;
    \item overall, sufficient goodness-of-fit and stability for subsequent
    structural identification.
\end{itemize}

\begin{table}[H]
\centering
\caption{Reduced-form VAR(12) equation fit statistics.}
\label{tab:var-fit}
\renewcommand{\arraystretch}{1.1}
\begin{tabular}{lcccc}
\toprule
\textbf{Equation} & \textbf{Number of Parameters} & \textbf{Std.\ Error} & $\mathbf{R^2}$ & \textbf{Adj.\ $R^2$} \\
\midrule
Production\_DL  & 49 & 0.9358 & 0.2507 & 0.1468 \\
OCSE\_DL        & 49 & 0.7255 & 0.5493 & 0.4868 \\
WTI\_real\_DL   & 49 & 0.9620 & 0.1928 & 0.0808 \\
Inventories\_DL & 49 & 0.8205 & 0.3978 & 0.3143 \\
\bottomrule
\end{tabular}
\end{table}

\begin{figure}[h!]
    \centering
    \includegraphics[width=0.9\textwidth]{fig_var_stability.pdf}
    \caption{Companion matrix eigenvalues for the VAR(12). All lie inside the unit circle.}
    \label{fig:var-stability}
\end{figure}

To quantify the in-sample forecasting performance of the reduced-form model,
Table~\ref{tab:varfit} reports standard accuracy metrics for real WTI, based
on the script \texttt{var\_fit\_vs\_actual\_WTI.m}.

\begin{table}[H]
    \centering
    \caption{Accuracy metrics for VAR predicted vs.\ actual real WTI.}
    \label{tab:varfit}
    \begin{tabular}{lc}
        \toprule
        Metric & Value \\
        \midrule
        RMSE   & 12.28 \\
        MAE    & 10.66 \\
        MAPE   & 61.34\% \\
        \bottomrule
    \end{tabular}
\end{table}



\section{Structural Identification via Sign Restrictions}
\label{sec:svar-ident}

\subsection{Economic Restrictions}

Following \textcite{Kilian2009} and \textcite{KilianMurphy2014}, sign restrictions are imposed over horizons $h=0,1,2$:

\begin{itemize}
    \item \textbf{Flow supply shock:} production $\downarrow$, inventories $\downarrow$, real WTI $\uparrow$, OECD -- limited reaction.
    \item \textbf{Aggregate demand shock:} OECD $\uparrow$, production $\uparrow$, real WTI $\uparrow$, inventories $\uparrow$.
    \item \textbf{Precautionary demand shock:} inventories $\uparrow$, real WTI $\uparrow$, OECD $\approx$ unchanged.
\end{itemize}

Table~\ref{tab:sign-restrictions} summarises the full set of restrictions used in the baseline specification.
\begin{table}[H]
\centering
\caption{Sign restrictions used for structural identification (responses on impact).}
\label{tab:sign-restrictions}
\renewcommand{\arraystretch}{1.1}
\begin{tabular}{lccc}
\toprule
\textbf{Variable} & \textbf{Supply Shock} & \textbf{Aggregate Demand Shock} & \textbf{Precautionary Shock} \\
\midrule
Production   & $-$ & $+$ & $0$ \\
OECD\_cycle  & $0$ & $+$ & $0$ \\
Inventories  & $-$ & $+$ & $+$ \\
Real WTI     & $+$ & $+$ & $+$ \\
\bottomrule
\end{tabular}
\end{table}



\subsection{Elasticity Bounds}

Elasticity constraints follow \textcite{KilianMurphy2014}:

\[
\varepsilon^s_p \in [-0.05, 0], \qquad \varepsilon^d_p \in [-0.2, -0.01].
\]

These enforce plausible short-run responses of supply and demand to price movements.

\subsection{Implementation}

The routine \texttt{svar\_sign\_restrictions.m} implements the
rotation-based procedure of Rubio-Ramirez, Waggoner and Zha~\cite{RubioRamirez2010}.
Starting from the reduced-form covariance matrix $\Sigma_u$, the algorithm
computes its Cholesky factor $P$, draws random orthonormal matrices $Q$ from
the Haar distribution, and constructs candidate impact matrices $B = P Q$.
A draw is retained only if the corresponding impulse responses satisfy all
sign restrictions over the horizon $h = 0,\dots,H$; otherwise it is discarded.
In the baseline specification, accepted draws represent about 1--3\% of all
candidates, in line with Kilian and Murphy~\cite{KilianMurphy2014}.


\section{Impulse Response Functions}
\label{sec:irfs}

Because the VAR is estimated on stationary log-differences (and on the
detrended OECD activity index), the raw impulse responses describe the effect
of each structural shock on monthly growth rates. For economic interpretation,
impulse responses for the levels are obtained by cumulating the responses of
log-differences over the horizon. The figures reported in this section refer to
these cumulated responses and can therefore be read as percentage deviations of
the levels from their baseline paths.

Impulse responses (IRFs) are computed over a 36-month horizon using bootstrap
percentile bands. For each shock, we report the median and the 16th--84th
percentiles across accepted structural decompositions.

Table~\ref{tab:shock-stats} summarises the basic distributional properties of
the identified shocks, confirming that they are approximately mean-zero and
exhibit non-Gaussian higher moments.

\begin{table}[H]
\centering
\caption{Descriptive statistics of identified structural shocks.}
\label{tab:shock-stats}
\renewcommand{\arraystretch}{1.1}
\begin{tabular}{lcccc}
\toprule
\textbf{Shock} & \textbf{Mean} & \textbf{Std.\ Dev.} & \textbf{Skewness} & \textbf{Kurtosis} \\
\midrule
Supply           & 0.0000 & 1.0013 & 0.0973   & 3.6076  \\
Aggregate Demand & 0.0000 & 1.0013 & $-0.5293$ & 4.7428 \\
Precautionary    & 0.0000 & 1.0013 & 1.7476   & 16.5490 \\
\bottomrule
\end{tabular}
\end{table}


\subsection{Flow Supply Shock}

A negative flow supply shock --- interpreted as an unanticipated shortfall in
U.S.\ crude oil production --- generates the following pattern in the cumulated
responses:

\begin{itemize}
    \item an immediate rise in the real WTI price, peaking after roughly
          3--4 months before gradually declining;
    \item a decline in crude oil production on impact, with only partial
          recovery over the subsequent year;
    \item an inventory drawdown consistent with the use of stocks to smooth
          the temporary supply shortfall;
    \item a muted and statistically imprecise reaction in OECD activity.
\end{itemize}

This configuration is qualitatively consistent with the supply-driven oil price
episodes documented by \textcite{Kilian2009}.

\begin{figure}[h!]
    \centering
    \includegraphics[width=1\textwidth]{fig_irf_supply.png}
    \caption{Impulse responses to a flow supply shock.}
    \label{fig:irf-supply}
\end{figure}
\subsection{Aggregate Demand Shock}

An aggregate demand expansion, capturing shifts in global economic activity,
produces:

\begin{itemize}
    \item a strong and persistent rise in OECD activity;
    \item an increase in crude oil production and inventories, reflecting the
          response of supply and stock-building to stronger demand;
    \item a sustained increase in the real WTI price that remains positive for
          roughly 12--18 months.
\end{itemize}


\begin{figure}[H]
    \centering
    \includegraphics[width=1\textwidth]{figu.png}
    \caption{Impulse responses to an aggregate demand shock.}
    \label{fig:irf-demand}
\end{figure}

\subsection{Precautionary Demand Shock}

A precautionary demand shock --- interpreted as a revision in expectations of
future scarcity --- yields:

\begin{itemize}
    \item an immediate and sizeable increase in inventories, as market
          participants build precautionary stocks;
    \item a short-lived but sharp increase in the real WTI price, which peaks
          within a few months and then dissipates;
    \item negligible and statistically weak effects on OECD activity and crude
          oil production.
\end{itemize}


\begin{figure}[H]
    \centering
    \includegraphics[width=1\textwidth]{figuta36.png}
    \caption{Impulse responses to a precautionary demand shock.}
    \label{fig:irf-inventory}
\end{figure}


\section{Forecast Error Variance Decomposition}
\label{sec:fevd}

The FEVD quantifies the importance of each shock in explaining forecast error variance at different horizons. Results are consistent with \textcite{Kilian2009,BaumeisterHamilton2019}:

\begin{itemize}
    \item supply shocks: limited contribution at short horizons;
    \item aggregate demand shocks: dominant at 6--18 months;
    \item precautionary shocks: relevant for short-run volatility.
\end{itemize}

\subsection{Multipanel FEVD Overview}

To provide a compact view of the contribution of each structural shock to the
variability of the real WTI price, Figure~\ref{fig:fevd-svar} reports the
SVAR-based forecast error variance decomposition over horizons up to 24 months.
Aggregate demand shocks account for roughly 45--50\% of the forecast error
variance at horizons between 6 and 18 months, while flow supply and
precautionary shocks each explain about 10\% in the long run.

\begin{figure}[h!]
    \centering
    \includegraphics[width=\textwidth]{svar_wti.png}
    \caption{SVAR-based forecast error variance decomposition of the real WTI
    price by structural shock (flow supply, aggregate demand, precautionary).}
    \label{fig:fevd-svar}
\end{figure}

The identification scheme and the main properties of the structural shocks are
summarised in Appendix~B.


\section{Historical Decomposition}
\label{sec:hist-decomp}

Historical decomposition assigns each observed movement in real WTI to one of the structural shocks. Using \texttt{extract\_structural\_shocks.m}, contributions are reconstructed across the full sample.

The decomposition indicates:

\begin{itemize}
    \item 2003--2008 oil price surge: predominantly aggregate demand;
    \item 2008 collapse: fall in aggregate demand + unwinding of precautionary positions;
    \item 2014--2016 decline: persistent positive supply shocks (shale expansion).
\end{itemize}

\section{Mapping Structural Shocks to Price Trajectories}
\label{sec:shock-mapping}

To generate the scenario-based projections in Chapter~\ref{ch:stress},
structural shocks must be translated into paths for the level of the real WTI
price. Since the VAR is estimated on stationary log-differences, the impulse
responses are first cumulated to obtain the effect of each shock on the log
level of the real price.

Let $\theta_h^{(j)}$ denote the \emph{cumulated} impulse response of log real
WTI at horizon $h$ to a unit structural shock $j \in \{s,d,p\}$ (flow supply,
aggregate demand, precautionary). For a shock of magnitude $\varepsilon_j$, the
deviation of log real WTI from its baseline path at horizon $h$ is

\[
    \Delta \log \text{WTI}_{t+h}
    \;=\; \varepsilon_j \, \theta_h^{(j)}.
\]

Given a baseline real price level $P_0$, the corresponding price path in levels
is obtained by exponentiating these deviations:

\[
    P_{t+h}^{(j)}
    \;=\;
    P_0 \exp\!\big( \varepsilon_j \, \theta_h^{(j)} \big).
\]

This mapping is implemented in \texttt{extract\_wti\_irf\_from\_var.m}, which
reads the SVAR impulse responses, cumulates the log-difference responses and
returns scenario-specific trajectories for the real WTI price. These paths are
then used in Chapter~\ref{ch:stress} to construct the baseline, supply-shock
and demand-shock scenarios for the stress-testing exercise.

\chapter{Stress Testing and Risk Engineering Application}
\label{ch:stress}

\section{Industrial Risk Exposure}

Fuel price volatility represents one of the most significant sources of operational risk for transportation-intensive industries. Among all sectors, commercial aviation is structurally the most exposed: jet fuel typically accounts for 20--35\% of operating expenses, making airlines highly sensitive to oil market disturbances. Variations in the real price of WTI translate almost immediately into changes in jet-fuel costs, impacting cash flows, budget stability and short-term liquidity. This risk is inherently systemic, as it is driven by global supply and demand forces, inventory behaviour and geopolitical tensions.

From an engineering perspective, hedging denotes a set of quantitative tools designed to reduce the variance of a firm’s cost structure when exposed to volatile external inputs. The goal is not to predict prices or maximise profit, but to reduce uncertainty and stabilise operational planning. For airlines, hedging fuel costs therefore constitutes a risk engineering problem: the task is to mitigate exposure to structural shocks in oil markets in a way that aligns with operational constraints, regulatory requirements and financial limits.

\subsection{Aviation-Specific Vulnerability}

The aviation sector is uniquely exposed because its core input (jet fuel) is directly tied to crude oil markets, which exhibit strong nonlinear responses to shocks. Flow supply disruptions, global demand expansions and precautionary inventory behaviour all generate distinct price patterns, as shown in Chapter~\ref{ch:empirical}. Airlines cannot store large quantities of fuel nor easily substitute it, and the fleet utilisation model (high-frequency operations, seasonal variability, hub scheduling) makes cost predictability essential.

ITA Airways, like most European carriers, displays a marked seasonal pattern in flight operations, with strong peaks during the summer months and troughs in winter. This directly translates into a seasonal pattern of monthly fuel consumption. Based on a conservative engineering analysis of flight operations, fleet utilisation and historical seasonality patterns, the annual jet-fuel consumption for 2023 is estimated at approximately $3.5$ million barrels (medium scenario), with plausible bounds ranging from $3.2$ to $4.0$ million barrels.\footnote{A transparent, scenario-based reconstruction of ITA Airways' 2023 jet-fuel consumption---including operational data, fleet composition, fuel-burn benchmarks and cross-checks against comparable European carriers---is documented in Appendix~\ref{app:ita_fuel}. The appendix summarises the same methodology presented in the internal technical report ``Stima Conservativa del Consumo Carburante Annuale di ITA Airways (2023)'', 2024.}

\section{Scenario Construction (SVAR-Based)}
\label{sec:scenario}

Using the SVAR identification framework developed in Chapter~\ref{ch:empirical}, and following standard oil-market VAR practice \cite{Kilian2009,KilianMurphy2014,BaumeisterHamilton2019}, structural shocks can be mapped into real WTI price trajectories. The \textsc{MATLAB} script \texttt{stress\_test\_hormuz\_VAR.m} produces three benchmark scenarios, saved in \texttt{stress\_hormuz\_VAR\_results.mat}:

\begin{itemize}
    \item \texttt{P\_baseline\_USD}: VAR median projection with no structural disturbance.
    \item \texttt{P\_supply\_USD}: WTI trajectory under a large negative supply shock (Hormuz-type).
    \item \texttt{P\_demand\_USD}: WTI trajectory under a large positive aggregate demand shock.
\end{itemize}

Table~\ref{tab:wti-scenarios} summarises the implied WTI levels over the first year under each scenario.

\begin{table}[H]
\centering
\caption{WTI price levels under baseline, supply-shock and demand-shock scenarios (first 12 months).}
\label{tab:wti-scenarios}
\renewcommand{\arraystretch}{1.2}
\begin{tabular}{lccc}
\toprule
\textbf{Month} & \textbf{Baseline} & \textbf{Supply Shock} & \textbf{Demand Shock} \\
\midrule
Jan & 60.0 & 60.0 & 60.0 \\ 
Feb & 60.0 & 72.2 & 62.6 \\ 
Mar & 60.0 & 93.9 & 63.4 \\ 
Apr & 60.0 & 82.0 & 63.7 \\ 
May & 60.0 & 76.9 & 63.1 \\ 
Jun & 60.0 & 77.8 & 62.9 \\ 
Jul & 60.0 & 72.5 & 62.6 \\ 
Aug & 60.0 & 78.0 & 62.3 \\ 
Sep & 60.0 & 84.1 & 61.8 \\ 
Oct & 60.0 & 75.6 & 61.2 \\ 
Nov & 60.0 & 95.2 & 62.0 \\ 
Dec & 60.0 & 98.6 & 62.9 \\ 
\bottomrule
\end{tabular}
\end{table}

\noindent\footnotesize\textit{Note}: Values rounded to one decimal place. \normalsize

\subsection{From Structural Shocks to Price Paths}

Let $\theta_{h}^{(j)}$ denote the impulse response of real WTI at horizon $h$ to shock $j \in \{s,d,p\}$. For a shock of magnitude $\varepsilon_j$, the corresponding WTI path is
\[
\Delta \text{WTI}^{\text{real}}_{t+h}
= \varepsilon_j \cdot \theta_{h}^{(j)}.
\]

The \textsc{MATLAB} script \texttt{extract\_wti\_irf\_from\_var.m} implements this mapping,
converting structural disturbances into trajectories for the real WTI price. For the risk-engineering application, these real-price paths are interpreted as constant-(2024)-dollar WTI prices; all scenario figures are therefore reported in real USD per barrel, abstracting from residual discrepancies between deflated and nominal series. A concise description of the script structure and its role in the stress-testing pipeline is provided in Appendix~C.

\subsection{Baseline Scenario}

The baseline path (\texttt{P\_baseline\_USD}) follows the median projection of the VAR and reflects a neutral environment without structural shocks. It captures the intrinsic persistence and autocorrelation structure of the oil market.

\begin{figure}[h!]
\centering
\includegraphics[width=1.0\textwidth]{montecarlo.png}
\caption{Monte Carlo VAR baseline forecast of the real WTI price. The solid line reports the median projection and the shaded area the 10--90\% prediction band around the initial price level.}
\label{fig:wti-baseline-mc}
\end{figure}

\subsection{Hormuz Supply Shock Scenario}

The Hormuz-type shock is modelled as a very large negative flow supply disturbance corresponding to roughly eleven standard deviations of the estimated supply shock distribution. The shock magnitude is calibrated so that the real WTI price in the stress scenario rises to slightly more than twice the baseline level after one year (an increase of about 100--110\%), consistent with an extreme disruption of exports through the Strait of Hormuz \parencite{Kilian2009,BaumeisterHamilton2019}. The scenario \texttt{P\_supply\_USD} exhibits:

\begin{itemize}
    \item an immediate and pronounced jump in WTI relative to the baseline path;
    \item a cumulative increase in the real WTI price of around 100\% at the 12-month horizon;
    \item gradual mean reversion over the subsequent months, with prices remaining substantially above the baseline throughout the 24-month stress-test window.
\end{itemize}

\subsection{Demand-Driven Spike Scenario}

The demand-shock scenario (\texttt{P\_demand\_USD}) corresponds to a two-standard-deviation aggregate demand expansion. This generates:

\begin{itemize}
    \item a gradual increase in WTI;
    \item persistent effects lasting up to 18 months;
    \item higher long-run price levels relative to supply-driven shocks.
\end{itemize}

Demand-driven scenarios typically reflect global economic expansions, rising industrial output and increased mobility demand.

\begin{figure}[h!]
\centering
\includegraphics[width=1.0\textwidth]{figura42.png}
\caption{Real WTI price trajectories under baseline, supply-shock and demand-shock scenarios. The dashed line reports the baseline path, while the solid lines correspond to the calibrated Hormuz-type supply shock and the strong aggregate demand shock.}
\label{fig:wti-scenarios}
\end{figure}

\section{Mapping WTI to Jet-Fuel Costs}
\label{sec:jetfuel}

\subsection{Pass-Through Model}

The link between crude oil prices and jet fuel is approximated using the regression estimated in \texttt{estimate\_pass\_through\_jetfuel.m} on monthly Jet Fuel USGC spot prices from the U.S.\ Energy Information Administration \cite{JetFuel_EIA}. Starting from a log--log specification, the pass-through is linearised around a benchmark WTI level of 60 USD/bbl:
\[
\widehat{JF}_t
= \alpha_{\text{lin}} + \beta_{\text{lin}} \cdot WTI_t,
\]
where $JF_t$ denotes the Jet Fuel USGC spot price. The pass-through coefficient $\beta_{\text{lin}}$ captures the proportion of crude price changes transmitted to jet fuel in a local neighbourhood of the calibration point.

\begin{table}[H]
\centering
\caption{Linear pass-through regression of Jet Fuel USGC prices on WTI.}
\label{tab:passthrough}
\renewcommand{\arraystretch}{1.25}
\begin{tabular}{lc}
\toprule
\textbf{Coefficient} & \textbf{Estimate} \\
\midrule
$\alpha_{\text{lin}}$ & -27.7001 \\
$\beta_{\text{lin}}$  & 3.6532   \\
$R^{2}$               & 0.790    \\
\bottomrule
\end{tabular}

\vspace{0.9em}
{\footnotesize
\begin{minipage}{0.85\linewidth}
\textit{Note}: $\alpha_{\text{lin}}$ and $\beta_{\text{lin}}$ are obtained by linearising 
the log--log pass-through equation at WTI = 60 USD/bbl.  
Standard errors are not reported because the linearised coefficients are analytically derived rather than directly estimated. The resulting relation should be interpreted as a local approximation rather than a full structural pricing model.
\end{minipage}
}
\end{table}

\subsection{Validation}

Residual diagnostics confirm:

\begin{itemize}
    \item limited autocorrelation;
    \item no severe heteroscedasticity;
    \item a reasonable in-sample fit for a simple linear approximation.
\end{itemize}

A scatter plot of observed vs.\ fitted jet fuel prices is shown in Figure~\ref{fig:pass-through}.

\begin{figure}[H]
\centering
\includegraphics[width=1.0\textwidth]{fig_pass_through_scatter.png}
\caption{Observed vs.\ fitted Jet Fuel USGC prices via linear pass-through.}
\label{fig:pass-through}
\end{figure}

\subsection{Scenario Mapping}

For each WTI scenario $\text{WTI}^{(k)}_{t+h}$, the corresponding jet-fuel price path is:
\[
JF^{(k)}_{t+h}
= \alpha_{\text{lin}} + \beta_{\text{lin}} \cdot \text{WTI}^{(k)}_{t+h}.
\]
These jet-fuel paths form the basis for computing monthly and annual operating costs under different environments.

\section{Risk Scenarios}
\label{sec:riskscen}

\subsection{Monthly Fuel Consumption Allocation}

Following Eurocontrol seasonal flight patterns and the technical analysis of ITA Airways' 2023 operations, annual fuel consumption ($C_{\text{annual}}$) is distributed across months using the following weights:
\[
w = 
(0.06, 0.06, 0.08, 0.085, 0.09, 0.095, 0.105, 0.105, 0.09, 0.085, 0.07, 0.08),
\]
which sum to 1. Monthly consumption is therefore modelled deterministically as
\[
C_m = w_m \cdot C_{\text{annual}}.
\]
Table~\ref{tab:weights} reports the full set of seasonal weights used in the stress tests.

\begin{table}[H]
\centering
\caption{Seasonal allocation weights for monthly jet-fuel consumption.}
\label{tab:weights}
\renewcommand{\arraystretch}{1.2}
\begin{tabular}{lcccccccccccc}
\toprule
\textbf{Month}  & Jan  & Feb  & Mar  & Apr   & May  & Jun   & Jul   & Aug   & Sep  & Oct   & Nov  & Dec  \\
\midrule
\textbf{Weight} & 0.06 & 0.06 & 0.08 & 0.085 & 0.09 & 0.095 & 0.105 & 0.105 & 0.09 & 0.085 & 0.07 & 0.08 \\
\bottomrule
\end{tabular}
\end{table}

\subsection{Baseline Scenario}

The baseline cost is:
\[
\text{Cost}_{m}^{\text{baseline}}
= C_m \cdot JF^{(\text{baseline})}_{m}.
\]

\subsection{Supply-Shock Scenario}

The Hormuz supply shock increases jet-fuel prices immediately and sharply:
\[
\text{Cost}_{m}^{\text{supply}}
= C_m \cdot JF^{(\text{supply})}_{m}.
\]

\subsection{Demand-Shock Scenario}

The demand shock results in a more persistent cost increase:
\[
\text{Cost}_{m}^{\text{demand}}
= C_m \cdot JF^{(\text{demand})}_{m}.
\]

\begin{table}[H]
\centering
\caption{Annual fuel cost under baseline, supply-shock and demand-shock scenarios.}
\label{tab:cost-scenarios}
\renewcommand{\arraystretch}{1.2}
\begin{tabular}{lccc}
\toprule
\textbf{Scenario} & \textbf{Annual Cost (USD)} & \textbf{Deviation from Baseline} & \textbf{Percent Change} \\
\midrule
Baseline     & 700.00\,mln & 0            & 0\%   \\
Supply Shock & 974.48\,mln & 274.48\,mln  & 39.2\% \\
Demand Shock & 731.82\,mln & 31.82\,mln   & 4.5\%  \\
\bottomrule
\end{tabular}
\end{table}

\begin{figure}[h!]
\centering
\includegraphics[width=1\textwidth]{braccio.png}
\caption{Jet fuel price and ITA Airways fuel cost under a Hormuz-type supply shock. 
The upper panel reports the Jet Fuel path under the calibrated supply shock (baseline vs.\ stress scenario);
the lower panel shows the corresponding monthly fuel cost with and without a 100\% linear hedge.}
\label{fig:cost-scenarios}
\end{figure}

\section{Hedging Strategies}
\label{sec:hedging}

\subsection{Futures Contracts}

A standard futures hedge fixes the purchase price:
\[
\text{Hedged Cost}^{F}_m
= C_m \cdot F,
\]
where $F$ is the futures price. The variance of fuel cost decreases, while the expected cost may rise or fall depending on market conditions.

\subsection{Swaps}

A fixed-for-floating swap ensures:
\[
\text{Hedged Cost}^{S}_m
= C_m \cdot S_{\text{fixed}},
\]
independent of jet-fuel spot prices. Swaps provide full linear protection but require credit support.

\subsection{Collars}

A zero-cost collar defines upper and lower bounds:
\begin{itemize}
    \item floor price: $P_{\min}$ via selling a put;
    \item cap price: $P_{\max}$ via buying a call.
\end{itemize}
This yields asymmetric protection at zero initial cost, at the expense of giving up part of the upside if prices fall.

\section{Impact Analysis}
\label{sec:impact}

\subsection{Cost Reduction}

In the three benchmark scenarios, a full linear hedge transforms the fuel-cost profile from one that closely tracks spot jet-fuel prices to one that is nearly flat across months. In the Hormuz supply-shock case, the hedge prevents the large temporary spike in fuel expenditure, keeping the annual fuel bill close to the baseline level despite the underlying price shock. In the more moderate demand-shock scenario, the hedge likewise dampens the cumulative cost increase, at the expense of foregoing potential savings in states where prices would have fallen below the locked-in level.

\subsection{Engineering Decision Implications}

Within this stylised setup, the results imply that fuel hedging:
\begin{itemize}
    \item can stabilise operational fuel budgets in the presence of large oil-market shocks;
    \item reduces exposure to adverse price scenarios such as a Hormuz-type supply disruption;
    \item can support more reliable capacity and pricing decisions, even though these operational links are not modelled explicitly in this thesis;
    \item improves the robustness of medium-term financial planning under extreme but plausible stress scenarios.
\end{itemize}

\chapter{Conclusions}
\label{ch:conclusions}

This thesis developed a comprehensive econometric and engineering framework for analysing oil market dynamics and assessing fuel price risk for aviation. The approach integrated structural macroeconomic modelling, nonlinear dependence analysis and scenario-based stress testing, culminating in an applied hedging case study for ITA Airways. The methodology combined VAR/SVAR identification, copula theory and engineering risk evaluation, with all empirical components implemented through reproducible \textsc{MATLAB} procedures.

\section{Summary of Findings}

The empirical results confirm the fundamental role of structural shocks in shaping oil price behaviour. Using a VAR(12) estimated on real WTI prices, U.S.\ crude oil production, OECD industrial activity and inventories, and identifying shocks via sign and elasticity restrictions, the analysis recovered flow supply, aggregate demand and precautionary demand disturbances consistent with the literature \parencite{Kilian2009, KilianMurphy2014, BaumeisterHamilton2019}.

Impulse responses revealed that:
\begin{itemize}
    \item supply shocks have strong but short-lived effects on oil prices;
    \item aggregate demand shocks generate persistent increases in prices and production;
    \item precautionary shocks produce sharp and immediate reactions, reflecting expectations of future scarcity.
\end{itemize}

The forecast error variance decomposition confirmed that medium-run price variability is dominated by demand shocks, whereas short-run volatility is closely linked to precautionary behaviour. Historical decompositions reproduced major episodes of oil price instability, including the 2003--2008 boom, the 2008 collapse and the 2014 shale-related decline.

Beyond marginal behaviour, the dependence structure of shocks was examined using copula models. Strong asymmetric tail dependence between aggregate and precautionary shocks was observed, highlighting the importance of nonlinear interactions during episodes of heightened uncertainty. These results \emph{motivate} the construction of stress scenarios that consider joint extreme realisations of structural shocks, even though the copula estimates are used in a primarily diagnostic rather than generative fashion.

The stress-testing module translated structural disturbances into WTI price paths using the identified impulse responses. Three benchmark scenarios were constructed: a baseline environment, a severe supply disruption reflecting a Hormuz-type event, and a demand-driven price spike. These were then mapped into jet fuel price trajectories using a simple log--log pass-through specification, locally linearised around a benchmark WTI price.

Monthly fuel costs for ITA Airways were computed by integrating these price paths with a deterministic, seasonally adjusted fuel consumption profile based on Eurocontrol traffic patterns and internal engineering estimates. The results showed that both supply and demand disturbances can generate substantial increases in monthly and annual fuel expenditure, with the Hormuz scenario producing the largest short-run impact and demand shocks generating more persistent cost pressure.

Finally, the hedging analysis evaluated three risk mitigation instruments --- futures, swaps and collars --- using monthly fuel consumption and scenario-based jet-fuel prices. Within the stylised three-scenario setup, all instruments flattened the fuel-cost profile relative to the unhedged case: swaps delivered the strongest smoothing of scenario-to-scenario differences, while collars provided asymmetric protection against upside price spikes at the expense of forfeiting some downside gains. These findings illustrate the potential engineering value of financial hedging as a means of stabilising operational budgets and mitigating exposure to structural oil market risk under extreme but plausible scenarios.

\section{Implications}

The results carry several implications for firms operating in fuel-intensive sectors:

\begin{itemize}
    \item \textbf{Structural understanding matters:} distinguishing between supply, aggregate demand and precautionary shocks is critical for designing robust hedging strategies and avoiding misinterpretation of market signals.
    \item \textbf{Demand shocks dominate long-run price behaviour:} firms should place significant weight on macroeconomic indicators when evaluating medium-term exposure.
    \item \textbf{Precautionary shocks represent a key short-run risk driver:} geopolitical uncertainty and expectation-driven behaviour can produce sharp cost spikes even without physical supply losses.
    \item \textbf{Hedging can improve budget stability:} even in this simplified setting, more stable fuel-cost profiles can facilitate engineering decisions related to fleet planning, capacity scheduling and ticket pricing, although these operational links are not modelled explicitly in this thesis.
\end{itemize}

For policymakers and regulators, the framework demonstrates how structural modelling can support risk assessment in sectors reliant on energy commodities, and highlights the importance of transparent, timely data on production, inventories and industrial activity. These implications should, however, be interpreted in light of the modelling simplifications discussed in Section~\ref{sec:limitations}.

\section{Limitations}
\label{sec:limitations}

The empirical analysis highlights several methodological and structural limitations that constrain the interpretation of the results and suggest caution when extrapolating beyond the scope of the thesis.

\begin{itemize}

    \item \textbf{Reduced-form restrictions and limited explanatory power.}  
    Although the VAR(12) successfully removes residual autocorrelation, it explains only a modest share of the variance of the real WTI price. The model captures dynamic co-movements but has limited predictive ability for the price level, as shown by the weak in-sample fit and the large role of unexplained residual shocks.

    \item \textbf{Heavy-tailed residuals and non-Gaussian structural shocks.}  
    Both reduced-form residuals and structurally identified shocks exhibit pronounced excess kurtosis and asymmetric tail behaviour. Assuming approximate normality in the bootstrap-based scenario generation therefore underestimates extreme risks, despite the use of copulas to characterise dependence.

    \item \textbf{Parameter constancy and potential structural instability.}  
    Rolling-window OLS results and historical episodes such as the shale revolution indicate that oil-market elasticities may vary over time. The constant-parameter VAR/SVAR framework may therefore mask regime changes in the propagation of supply, demand and precautionary shocks.

    \item \textbf{Linearity of the propagation mechanism.}  
    The VAR imposes linear dynamics, whereas the copula analysis reveals strong state-dependent tail dependence, particularly between demand and precautionary shocks. Such nonlinearities are not captured by the current specification.

    \item \textbf{Simplified pass-through from WTI to Jet Fuel.}  
    The pass-through model is based on a single log--log relation, locally linearised around one price point, and ignores crack-spread dynamics, refinery bottlenecks and potential asymmetries in Jet Fuel pricing. This introduces model risk when mapping WTI scenarios to fuel costs.

    \item \textbf{Absence of basis risk and market constraints in the hedging exercise.}  
    The hedging application assumes a perfect hedge with no WTI--Jet Fuel basis risk, no margin requirements, no forward-curve structure and no foreign-exchange risk. These assumptions make the results mechanically optimistic relative to realistic trading conditions.

    \item \textbf{Deterministic fuel consumption and limited operational detail.}  
    Monthly Jet Fuel use follows a deterministic and highly stylised seasonal profile. Real-world airline operations exhibit substantial variability due to seasonality, load factors, aircraft mix and route structure, none of which is captured by the model; stochastic volume risk is therefore ignored.

    \item \textbf{Reduced-form Monte Carlo simulation.}  
    Scenario generation relies on bootstrap sampling of VAR residuals, which preserves reduced-form dynamics but not the structural dependence patterns among shocks. As a result, structural tail dependence may be only partially reflected in simulated extremes.

\end{itemize}

These limitations do not undermine the qualitative insights of the analysis but restrict the generality of the quantitative results and motivate the extensions discussed in the following section.

\section{Future Research Directions}

The limitations discussed above naturally point to several directions for future research aimed at improving the statistical realism, structural interpretability and practical relevance of oil-market risk assessment.

\begin{itemize}

    \item \textbf{Time-varying parameter SVARs with stochastic volatility.}  
    The evidence of structural instability and evolving elasticities suggests moving toward TVP-SVAR or Bayesian stochastic-volatility frameworks, allowing the propagation of supply, demand and precautionary shocks to drift across regimes such as the shale revolution or crisis periods.

    \item \textbf{Volatility modelling with higher-frequency data.}  
    Given the heavy-tailed residuals and volatility clustering, future work could estimate GARCH-type or stochastic-volatility models using weekly or daily data, integrating them either as stand-alone components or as volatility inputs to a multivariate model.

    \item \textbf{Nonlinear and regime-dependent VAR structures.}  
    The tail dependence detected between demand and precautionary shocks motivates nonlinear specifications: threshold VARs, Markov-switching VARs or smooth-transition VARs capable of capturing state-dependent propagation under extreme market conditions.

    \item \textbf{Hybrid deep-learning forecasting models.}  
    Since the VAR explains only a limited share of price-level variability, deep-learning sequence models such as LSTM or transformer architectures could complement the structural model. A hybrid pipeline---GARCH/SV for volatility, LSTM for nonlinear interactions, SVAR for structural interpretation---may yield richer scenario generation.

    \item \textbf{Dynamic and higher-dimensional copula models.}  
    Extending the copula analysis to dynamic copulas or vine copulas would allow dependence among structural shocks to evolve over time, strengthening the realism of structural scenario simulation and tail-risk quantification.

    \item \textbf{More realistic pass-through and basis-risk modelling.}  
    Future work could model crack-spread dynamics, refinery capacity constraints and Jet Fuel--WTI basis risk explicitly, replacing the current linear pass-through with a structural or semi-parametric pricing relation.

    \item \textbf{Stochastic and operationally detailed fuel-consumption modelling.}  
    Incorporating route-level data, aircraft utilisation, seasonal load factors and stochastic volume uncertainty would provide a more granular mapping from oil-market shocks to airline operating costs.

    \item \textbf{Advanced hedging optimisation under market constraints.}  
    A rigorous extension would integrate FX risk, margin requirements, forward-curve structure, liquidity constraints and partial-hedge strategies within a stochastic programming or risk-budgeting framework.
    
\end{itemize}

Overall, these extensions would allow future research to bridge the gap between structural macro--oil modelling and practical risk engineering, especially in applications involving extreme-event stress testing and airline fuel-cost management.

\section{Final Remarks}

This thesis demonstrates that combining structural econometric modelling with engineering-based risk analysis offers a useful framework for understanding and mitigating fuel price exposure. By linking structural shocks, nonlinear dependence patterns and hedging instruments within an integrated stress-testing architecture, the study provides actionable insights for both academic research and industrial decision-making. The methodology is general and can be extended to other energy-intensive sectors, supporting a broader understanding of commodity price risk in complex operational environments. 

Overall, the thesis advances the structural analysis of oil-market dynamics and illustrates how econometric identification, basic tail-risk considerations and engineering-driven hedging strategies can be integrated into a coherent, albeit stylised, framework for scenario-based risk assessment in fuel-intensive industries.


\clearpage
\printbibliography[heading=bibintoc,title={Bibliography}]

% ---------------------------
%        APPENDICI
% ---------------------------
\clearpage
\appendix
\chapter{Data Construction and Pre-VAR Diagnostics}

\section*{A.1 Data Sources, Frequency Alignment and Storage}

The empirical analysis relies on monthly data spanning 1990--2024 for four
key variables: global crude oil production, OECD real activity (aggregate
industrial index), petroleum inventories and the real WTI price. Raw series
are gathered from the sources listed in Table~\ref{tab:data_sources} in
Chapter~\ref{ch:data-methods} and then harmonised to a common monthly calendar.

All series are first converted to end-of-month observations and aligned on a
shared time index. When a series is released at a different frequency or with
occasional missing values, the following conventions are applied:

\begin{itemize}
    \item quarterly or higher-frequency observations are aggregated or
    averaged to the monthly frequency, preserving the timing conventions of
    the original release;
    \item isolated missing entries are linearly interpolated only when this is
    necessary to avoid spurious gaps within otherwise continuous stretches of
    data; no series is forward- or backward-filled over extended periods;
    \item outlier values clearly attributable to data errors are replaced by
    the median of a narrow neighbourhood, with all adjustments documented in
    the \texttt{build\_oil\_dataset.m} script.
\end{itemize}

After alignment, three data structures are created and stored in
\texttt{clean\_data.mat}:

\begin{itemize}
    \item \texttt{All}: timetable of raw levels or log-levels (real WTI price,
    crude oil production, OECD activity index, petroleum inventories);
    \item \texttt{All\_d}: timetable of stationary transformations
    (first differences or $\Delta \log$ transforms), subsequently
    standardised to zero mean and unit variance;
    \item \texttt{ALL\_VAR}: numeric matrix used as input for the VAR
    estimation, with columns ordered as Production, OECD, WTI and Inventories
    and rows corresponding to common monthly dates.
\end{itemize}

The OECD activity index is chosen as the global demand proxy because, within
the sample, it displays a more stable dynamic relationship with the real WTI
price than the IGREA index, while avoiding some of the measurement revisions
associated with shipping-based indicators
\parencite{Kilian2009,Kilian2019}.

\section*{A.2 Transformations and Stationarity Checks}

The construction of \texttt{All\_d} follows the standard practice of rendering
the series approximately covariance-stationary before specifying a VAR.

\begin{itemize}
    \item Crude oil production, the real WTI price and inventories are
    transformed using $\Delta \log$ differences, which correspond
    approximately to monthly percentage changes.
    \item The OECD activity index is decomposed into a low-frequency trend and
    a cyclical component; the latter (denoted \texttt{OECD\_cycle}) is used in
    the VAR to capture global business-cycle fluctuations around the long-run
    level.
    \item Each stationary transformation is then standardised to zero mean and
    unit variance, so that the corresponding VAR coefficients and impulse
    responses are expressed in comparable units.
\end{itemize}

Augmented Dickey--Fuller (ADF) tests are applied both to the original levels
and to the transformed series. In levels, the null of a unit root cannot be
rejected for any of the four variables at conventional significance levels,
consistent with the large literature documenting non-stationarity in oil
prices, production, inventories and activity indicators. In contrast, all
transformed series in \texttt{All\_d} exhibit ADF $p$-values well below the
5\% threshold, supporting the use of the VAR in differences and cycle form.

To complement formal tests, pre-VAR diagnostics include visual inspection of
time-series plots, autocorrelation functions and ADF $p$-value charts for each
variable. These confirm that the transformed series fluctuate around a stable
mean with no obvious residual trends or explosive behaviour.

\section*{A.3 Baseline Static Regression and Diagnostics}

Before moving to a multivariate dynamic specification, a benchmark static
regression is estimated to assess whether contemporaneous linear relationships
between the transformed fundamentals and the oil price can explain a
meaningful share of its variation. The regression is specified as:
\[
WTI_{t}^{DL}
= \beta_0
+ \beta_1 \, \text{Production}_{t}^{DL}
+ \beta_2 \, \text{OECD}_{t}^{DL}
+ \beta_3 \, \text{Inventories}_{t}^{DL}
+ u_t,
\]
where the superscript $DL$ denotes $\Delta \log$-transformed variables.
Although this formulation already removes deterministic trends and stabilises
variances, the model attains only a very modest adjusted $R^2$ (approximately
0.11), indicating that contemporaneous fundamentals have limited explanatory
power for the monthly change in the real WTI price.

Residual diagnostics for this OLS specification reveal:

\begin{itemize}
  \item strong serial correlation, as indicated by Ljung--Box statistics with
  $p$-values well below 1\%;
  \item pronounced heteroskedasticity according to Breusch--Pagan tests
  ($p \approx 0.0000$);
  \item clear deviations from normality, with heavy tails relative to the
  Gaussian benchmark.
\end{itemize}

Rolling-window estimates of the coefficients further display substantial
instability across subsamples, especially around major structural episodes
such as the early-2000s boom and the post-2010 shale expansion. Together,
these diagnostics suggest that the static OLS framework is inadequate for
capturing the dynamics of oil prices and motivates the move to a multivariate,
dynamic VAR/SVAR approach.

\section*{A.4 Distributional Properties of OLS Residuals}

The residuals $u_t$ from the baseline regression provide an initial window on
the distributional features of unexplained oil-price movements. Empirical
density plots, histograms and normal Q--Q plots point to marked departures
from Gaussianity:

\begin{itemize}
    \item the residual distribution exhibits mild negative skewness, reflecting
    occasional large downward adjustments in the real WTI price;
    \item excess kurtosis indicates a leptokurtic shape, with more probability
    mass in the centre and in the tails than a normal distribution;
    \item extreme observations are clustered around well-known market events
    (price collapses and sharp spikes), rather than being isolated outliers.
\end{itemize}

These features justify the use of bootstrap methods in subsequent stages
(impulse-response confidence bands, scenario generation) and provide
preliminary evidence that non-Gaussian shocks may be structurally relevant in
the oil market. In particular, the heavy tails and asymmetric behaviour of
$u_t$ are consistent with the presence of large, infrequent structural
disturbances, which are modelled explicitly in the SVAR framework developed in
Chapters~\ref{ch:empirical} and~\ref{ch:stress}.

\clearpage

\chapter{VAR--SVAR Framework and Shock Dependence}

\section*{B.1 Reduced-Form VAR Specification}

The reduced-form VAR is estimated on the vector of stationary
transformations
\[
y_t =
\begin{bmatrix}
\text{Production}^{DL}_t \\
\text{OECD\_cycle}_t \\
\text{WTI}^{DL}_t \\
\text{Inventories}^{DL}_t
\end{bmatrix},
\]
where $DL$ denotes $\Delta\log$ transformations and
\texttt{OECD\_cycle} is the cyclical component of the OECD activity index.
Each component of $y_t$ is standardised to zero mean and unit variance before
estimation, so that the resulting coefficients and shocks are directly
comparable across variables.

The VAR($p$) in companion form is
\[
y_t = A_1 y_{t-1} + \dots + A_p y_{t-p} + u_t,
\quad
u_t \sim (0,\Sigma_u),
\]
with the ordering
\[
(\text{Production}, \; \text{OECD}, \; \text{WTI}, \; \text{Inventories}).
\]

Lag-length selection is performed over $p=1,\dots,15$ using a combination of
information criteria, residual autocorrelation tests and stability checks on
the companion matrix eigenvalues. All criteria point to relatively long
dynamics, and residual portmanteau tests reject low-order specifications. A
lag length of $p=12$ is ultimately selected as a compromise between capturing
the long adjustment horizons typical of global oil markets and preserving a
reasonable number of effective observations.

The script \texttt{var\_main.m} implements this procedure and stores the
estimated coefficient matrices and residuals, which serve as the basis for the
subsequent structural analysis.

\section*{B.2 Benchmark Cholesky Identification}

As a preliminary step, a recursive (Cholesky) identification is applied to the
reduced-form residuals $u_t$. The covariance matrix $\Sigma_u$ is factorised as
$\Sigma_u = P P'$ and the orthogonalised innovations
$\tilde{\varepsilon}_t = P^{-1} u_t$ are interpreted as shocks ordered according
to $(\text{Production}, \text{OECD}, \text{WTI}, \text{Inventories})$.

The associated Cholesky impulse responses provide a diagnostic benchmark:

\begin{itemize}
  \item aggregate-demand disturbances (innovations to OECD activity) account
  for a large fraction of the forecast error variance of the real WTI price at
  medium horizons;
  \item shocks to crude oil production generate modest and short-lived price
  responses;
  \item inventory innovations produce noisy and moderately persistent effects.
\end{itemize}

These patterns are broadly in line with the structural interpretation proposed
in the literature, but the recursive scheme is inherently sensitive to the
chosen ordering and cannot be given a fully structural meaning. It is therefore
used only as a starting point for assessing the plausibility of the data and
guiding the design of sign restrictions.

\section*{B.3 Structural Identification via Sign Restrictions}

Structural shocks are identified using sign restrictions in the spirit of
\textcite{KilianMurphy2014}. Let
\[
u_t = B \varepsilon_t,
\]
where $B$ is the contemporaneous impact matrix and
$\varepsilon_t = (\varepsilon_t^{s},\varepsilon_t^{d},\varepsilon_t^{p},\varepsilon_t^{\text{other}})'$
collects the structural disturbances: flow supply ($s$), aggregate demand
($d$), precautionary (or storage) demand ($p$) and a residual “other” shock.

Starting from a particular decomposition $B_0$ such that $\Sigma_u = B_0 B_0'$,
a large number of candidate impact matrices are generated via random orthogonal
rotations $Q$:
\[
B = B_0 Q, \qquad Q'Q = I.
\]
For each candidate $B$, the corresponding impulse responses are computed and
retained only if they satisfy the following sign restrictions over a short
horizon (typically $h=1,\dots,12$):

\begin{itemize}
  \item \textbf{Flow supply shock} ($\varepsilon^s$): Production $\downarrow$,
  WTI $\uparrow$;
  \item \textbf{Aggregate-demand shock} ($\varepsilon^d$): OECD $\uparrow$,
  Production $\uparrow$, WTI $\uparrow$;
  \item \textbf{Precautionary/storage shock} ($\varepsilon^p$): WTI $\uparrow$,
  Inventories $\uparrow$.
\end{itemize}

Out of 100{,}000 random orthogonal rotations drawn in
\texttt{svar\_sign\_restrictions.m}, only a handful (six) satisfy all the
restrictions simultaneously. This low acceptance rate indicates that the
restrictions are informative and tightly pin down the economically relevant
decomposition of shocks. All structural results reported in the main text are
computed as medians across the accepted draws, with percentile bands used to
capture the remaining identification uncertainty.

\section*{B.4 Structural Impulse Responses and Variance Decomposition}

The structural impulse responses obtained from the accepted $B$ matrices reveal
a clear pattern:

\begin{itemize}
  \item aggregate-demand shocks produce the strongest and most persistent
  increases in the real WTI price, together with sustained increases in
  production and economic activity;
  \item flow supply shocks generate moderate but economically meaningful price
  responses that decay relatively quickly, with production falling on impact
  and gradually recovering;
  \item precautionary shocks display hump-shaped short-run dynamics in the oil
  price and inventories, consistent with temporary inventory accumulation driven
  by expectations of future scarcity.
\end{itemize}

The corresponding forecast error variance decomposition (reported in
Chapter~\ref{ch:empirical}) shows that demand shocks dominate the medium-run
variance of the real WTI price, while precautionary shocks contribute
disproportionately to very short-run volatility. Supply shocks play a more
limited but non-negligible role.

\section*{B.5 Marginal Distributions of Structural Shocks}

To analyse the distributional properties of structural shocks, the accepted
draws of $(\varepsilon_t^{s},\varepsilon_t^{d},\varepsilon_t^{p})$ are
assembled and standardised. The script \texttt{fit\_marginal\_shocks.m} then
fits several parametric families to each marginal by maximum likelihood,
including Gaussian, Logistic and Student-\emph{t} distributions.

Across all three shocks, Gaussian and Logistic specifications systematically
underestimate the empirical tail thickness. Student-\emph{t} marginals with
finite degrees of freedom provide a markedly better fit, capturing both the
leptokurtosis and, for demand and precautionary shocks, pronounced skewness.
These findings reinforce the evidence from pre-VAR diagnostics that
heavy-tailed shocks are a salient feature of oil-market dynamics and motivate
their use in the copula-based dependence analysis.

\section*{B.6 Copula-Based Dependence}

Dependence among structural shocks is studied using bivariate copulas fitted to
pairs $(\varepsilon_t^{i},\varepsilon_t^{j})$ after transforming each marginal
to the unit interval via the fitted Student-\emph{t} distributions. The
\texttt{copula\_fit\_shocks.m} script compares several families---Gaussian,
Student-\emph{t}, Clayton, Gumbel and Frank---using pseudo-likelihood criteria.

The main qualitative findings are:

\begin{itemize}
  \item a Student-\emph{t} copula with low degrees of freedom
  ($\nu \approx 3.4$) best describes the joint behaviour of aggregate-demand
  and precautionary shocks, implying strong upper-tail dependence: large
  positive demand shocks tend to coincide with large positive precautionary
  shocks;
  \item dependence between flow supply and precautionary shocks is weaker and
  more symmetric, with a Frank copula providing the best fit;
  \item dependence between flow supply and aggregate-demand shocks is moderate
  and close to elliptical, and can be captured reasonably well by either a
  Gaussian or a mild Student-\emph{t} copula.
\end{itemize}

These patterns suggest that episodes of strong global demand are often
accompanied by inventory-accumulation behaviour and expectation-driven
pressures, thereby amplifying oil price spikes. The copula analysis is used
here in a purely \emph{diagnostic} fashion: it characterises how shocks co-move
in the historical sample and motivates the construction of joint stress
scenarios, but it does not yet feed directly into the scenario generator of
Chapter~\ref{ch:stress}.

\section*{B.7 Conditional Tail Probabilities}

Using the fitted copulas, the script
\texttt{copula\_prob\_cond\_shocks.m} computes conditional probabilities of
joint extreme events. In particular, interest centres on the probability that
one shock exceeds a high quantile conditional on another shock being large:

\begin{itemize}
  \item the probability that the precautionary shock exceeds its 90th
  percentile given that the aggregate-demand shock is above its 90th
  percentile is found to be in the range of 35--45\%, consistent with strong
  upper-tail dependence between the two;
  \item in contrast, the probability that the precautionary shock is extreme
  conditional on a large negative flow supply shock remains below 10\%,
  reflecting the weaker and more symmetric dependence between supply and
  precautionary disturbances.
\end{itemize}

These conditional tail probabilities highlight that extreme demand episodes are
the most likely environment in which large precautionary shocks materialise,
providing a structural rationale for focusing stress scenarios on joint demand
and precautionary disturbances. A full integration of copula-based simulation
into the scenario generator is left to future research and outlined in
Chapter~\ref{ch:conclusions}.

\section*{B.8 Additional VAR Diagnostic: Residual Distribution}

Figure~\ref{fig:var-residuals-app} reports the histogram of standardised
residuals for the WTI equation in the VAR(12), overlaid with a standard normal
density. The heavy tails and mild asymmetry relative to the Gaussian benchmark
are consistent with the evidence discussed in Appendix~A and motivate the use
of bootstrap methods and heavy–tailed marginals in the copula analysis.

\begin{figure}[H]
    \centering
    \includegraphics[width=0.9\textwidth]{fig_var_residuals.pdf} % usa lo stesso file che avevi
    \caption{Standardised residuals of the WTI equation in the VAR(12) versus a standard normal density.}
    \label{fig:var-residuals-app}
\end{figure}


\clearpage

\chapter{ITA Airways Fuel Demand Estimation}
\label{app:ita_fuel}

\section*{C.1 Purpose and Context}

The hedging simulations developed in the main analysis require a realistic estimate of ITA Airways’ annual jet-fuel consumption. Since the airline does not publicly disclose its physical fuel-use figures, the value must be inferred indirectly using operational data, fleet characteristics and standard aviation fuel-burn benchmarks.

This appendix documents the procedure followed to obtain a transparent and conservative estimate suitable for modelling fuel-price exposure. The goal is not to reconstruct exact accounting values, but to generate a consistent operational estimate applicable to the stress-testing and hedging framework.

\section*{C.2 Underlying Operational Data}

The estimation relies on publicly available information for the year 2023:

\begin{itemize}[leftmargin=*]
    \item \textbf{Scheduled flights:} approximately 124,000 across domestic, intra-European and long-haul routes.
    \item \textbf{Passengers:} roughly 15 million carried, with an average load factor of about 79\%.
    \item \textbf{Fleet:} 96 Airbus aircraft, including a growing share of new-generation aircraft (A220, A320neo, A330neo, A350).
    \item \textbf{Network structure:} operations distributed across:
    \begin{itemize}[leftmargin=*]
        \item Domestic and short-haul European routes (\(< 2.5\) hours),
        \item Medium-haul routes (2.5--5 hours),
        \item Long-haul intercontinental routes (North America, South America, Middle East, India).
    \end{itemize}
\end{itemize}

\section*{C.3 Fuel-Burn Benchmarks and Adjustments}

Average per-flight fuel-burn benchmarks are sourced from industry technical reports and aircraft flight-performance documentation:

\begin{itemize}[leftmargin=*]
    \item Narrow-body aircraft (short/medium haul): \(2.5\)–\(3.0\) tonnes/flight.
    \item Wide-body aircraft (long haul): \(40\)–\(55\) tonnes/flight.
\end{itemize}

To account for real-world operational inefficiencies (congestion, taxi time, step climbs, airspace restrictions, payload variability), a \(\,+10\%\) adjustment is applied.

The following conversion factor is used throughout:

\[
1~\text{tonne Jet-A1} \approx 7.9~\text{barrels}.
\]

\section*{C.4 Scenario-Based Estimation Method}

Due to incomplete route-level public data, fuel consumption is modelled using a scenario-based aggregation:

\[
Q_{\text{jet}} = N_{\text{SR}}\, q_{\text{SR}} + N_{\text{LH}}\, q_{\text{LH}},
\]

where \(N_{\text{SR}}, N_{\text{LH}}\) denote the number of short-/medium-haul and long-haul flights, and \(q_{\text{SR}}, q_{\text{LH}}\) represent corresponding fuel-burn averages.

Instead of fixing values at a granular level, three internally consistent scenarios are constructed to capture uncertainty in:

\begin{itemize}[leftmargin=*]
    \item mix of long-haul versus short-/medium-haul flying,
    \item share of new-generation versus legacy aircraft,
    \item seasonal variation and utilisation rates,
    \item operational efficiency and route constraints.
\end{itemize}

\subsection*{C.4.1 Scenario Results}

\begin{table}[H]
\centering
\caption{Estimated 2023 fuel consumption under alternative scenarios.}
\label{tab:ita_scenarios}
\renewcommand{\arraystretch}{1.25}
\begin{tabularx}{\textwidth}{lccX}
\toprule
\textbf{Scenario} & \textbf{Fuel (Mt)} & \textbf{Fuel (million barrels)} & \textbf{Notes} \\
\midrule
Minimum & $\approx 0.40$ & $\approx 3.2$ & High efficiency, shorter route structure and high share of new-generation aircraft. \\
Central & $0.44$--$0.45$ & $3.5$--$3.6$ & Reflects observed 2023 utilisation, fleet composition and route distribution. \\
Maximum & $0.49$--$0.50$ & $3.9$--$4.0$ & Higher long-haul share, lower efficiency and greater legacy aircraft operation. \\
\bottomrule
\end{tabularx}
\end{table}


\section*{C.5 Cross-Validation Against Comparable Airlines}

To validate plausibility, the estimated range is benchmarked against
\emph{approximate 2022 jet-fuel use reconstructed from public environmental
and sustainability disclosures} of airlines with similar fleet size and
network structure:\footnote{Approximate values obtained by combining
CO$_2$-emissions and fuel-efficiency indicators reported in the 2022
sustainability/annual reports of Finnair, TAP Air Portugal and LOT Polish
Airlines; see, for example, \textcite{Finnair2022,TAP2022,PGLDecarb2030}.}

\begin{itemize}[leftmargin=*]
    \item Finnair (2022): $\sim 0.79$ Mt,
    \item TAP Air Portugal (2022): $\sim 0.58$ Mt,
    \item LOT Polish Airlines (2022): $\sim 0.45$ Mt.
\end{itemize}

Given ITA’s long-haul expansion in 2023 and the partial transition toward
newer aircraft, the estimated $0.44$–$0.50$ Mt range is consistent with
these comparators.


\section*{C.6 Financial Consistency Check}

To ensure internal consistency with typical airline economics, fuel cost implications are evaluated using a reference wholesale price range:

\[
\$700 \leq P_{\text{Jet Fuel}} \leq \$1,050 \text{ per tonne}.
\]

This implies:

\[
280\text{ M€} \leq \text{Annual Fuel Cost} \leq 525\text{ M€},
\]

which aligns with the 20–35\% cost share commonly observed among European network carriers.

\section*{C.7 Final Value Adopted in the Model}

For the simulations, a single representative value is required. Based on scenario inference and validation tests, the following value is adopted:

\[
Q_{\text{jet, used}} = 3.5 \times 10^{6}~\text{barrels/year}.
\]

This quantity provides a conservative but realistic approximation of ITA Airways’ annual exposure to jet-fuel price risk.



   % Ringraziamenti 
    \chapter*{Ringraziamenti}
    Desidero innanzitutto ringraziare il mio relatore, il \textit{Ch.mo} Dott.\ Marco Papi, per la disponibilità, la chiarezza delle indicazioni e il rigore metodologico con cui mi ha guidato nello sviluppo di questo lavoro. La possibilità di confrontarmi con lui sui temi dell’econometria dei mercati energetici e del risk management mi ha permesso di scoprire un interesse profondo per queste tematiche, che considero ormai centrale nelle mie prospettive di crescita accademica e professionale. Gli sono particolarmente grato non solo per l’opportunità di approfondire questi argomenti, che hanno contribuito a ridefinire l’orientamento delle mie scelte future, ma anche dal punto di vista personale: mi ha accolto in un momento difficile e mi ha aiutato, con discrezione e continuità, a ritrovare equilibrio e motivazione. A lui devo una parte essenziale di questo percorso.

    Un ringraziamento va anche ai colleghi di corso con cui ho condiviso non solo esami e progetti, ma anche fatiche, preoccupazioni e soddisfazioni, e che hanno contribuito in modo significativo a chiarire concetti e a mettere alla prova le mie idee.

    Infine, un ringraziamento particolare va alla mia famiglia, per il sostegno costante, la pazienza e la fiducia riposta in me lungo tutto il percorso universitario. La loro presenza silenziosa ma continua è stata la condizione che ha reso possibile questo traguardo.
    
    Ringrazio inoltre tutte le persone che non compaiono in questo elenco, ma che sanno di avere il mio affetto e che desidero sinceramente ringraziare per essermi state accanto.


\end{document}
