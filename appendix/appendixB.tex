\chapter{VAR--SVAR Framework and Shock Dependence}

\section*{B.1 Reduced-Form VAR Specification}

The reduced-form VAR is estimated on the vector of stationary
transformations
\[
y_t =
\begin{bmatrix}
\text{Production}^{DL}_t \\
\text{OECD\_cycle}_t \\
\text{WTI}^{DL}_t \\
\text{Inventories}^{DL}_t
\end{bmatrix},
\]
where $DL$ denotes $\Delta\log$ transformations and
\texttt{OECD\_cycle} is the cyclical component of the OECD activity index.
Each component of $y_t$ is standardised to zero mean and unit variance before
estimation, so that the resulting coefficients and shocks are directly
comparable across variables.

The VAR($p$) in companion form is
\[
y_t = A_1 y_{t-1} + \dots + A_p y_{t-p} + u_t,
\quad
u_t \sim (0,\Sigma_u),
\]
with the ordering
\[
(\text{Production}, \; \text{OECD}, \; \text{WTI}, \; \text{Inventories}).
\]

Lag-length selection is performed over $p=1,\dots,15$ using a combination of
information criteria, residual autocorrelation tests and stability checks on
the companion matrix eigenvalues. All criteria point to relatively long
dynamics, and residual portmanteau tests reject low-order specifications. A
lag length of $p=12$ is ultimately selected as a compromise between capturing
the long adjustment horizons typical of global oil markets and preserving a
reasonable number of effective observations.

The script \texttt{var\_main.m} implements this procedure and stores the
estimated coefficient matrices and residuals, which serve as the basis for the
subsequent structural analysis.

\section*{B.2 Benchmark Cholesky Identification}

As a preliminary step, a recursive (Cholesky) identification is applied to the
reduced-form residuals $u_t$. The covariance matrix $\Sigma_u$ is factorised as
$\Sigma_u = P P'$ and the orthogonalised innovations
$\tilde{\varepsilon}_t = P^{-1} u_t$ are interpreted as shocks ordered according
to $(\text{Production}, \text{OECD}, \text{WTI}, \text{Inventories})$.

The associated Cholesky impulse responses provide a diagnostic benchmark:

\begin{itemize}
  \item aggregate-demand disturbances (innovations to OECD activity) account
  for a large fraction of the forecast error variance of the real WTI price at
  medium horizons;
  \item shocks to crude oil production generate modest and short-lived price
  responses;
  \item inventory innovations produce noisy and moderately persistent effects.
\end{itemize}

These patterns are broadly in line with the structural interpretation proposed
in the literature, but the recursive scheme is inherently sensitive to the
chosen ordering and cannot be given a fully structural meaning. It is therefore
used only as a starting point for assessing the plausibility of the data and
guiding the design of sign restrictions.

\section*{B.3 Structural Identification via Sign Restrictions}

Structural shocks are identified using sign restrictions in the spirit of
\textcite{KilianMurphy2014}. Let
\[
u_t = B \varepsilon_t,
\]
where $B$ is the contemporaneous impact matrix and
$\varepsilon_t = (\varepsilon_t^{s},\varepsilon_t^{d},\varepsilon_t^{p},\varepsilon_t^{\text{other}})'$
collects the structural disturbances: flow supply ($s$), aggregate demand
($d$), precautionary (or storage) demand ($p$) and a residual “other” shock.

Starting from a particular decomposition $B_0$ such that $\Sigma_u = B_0 B_0'$,
a large number of candidate impact matrices are generated via random orthogonal
rotations $Q$:
\[
B = B_0 Q, \qquad Q'Q = I.
\]
For each candidate $B$, the corresponding impulse responses are computed and
retained only if they satisfy the following sign restrictions over a short
horizon (typically $h=1,\dots,12$):

\begin{itemize}
  \item \textbf{Flow supply shock} ($\varepsilon^s$): Production $\downarrow$,
  WTI $\uparrow$;
  \item \textbf{Aggregate-demand shock} ($\varepsilon^d$): OECD $\uparrow$,
  Production $\uparrow$, WTI $\uparrow$;
  \item \textbf{Precautionary/storage shock} ($\varepsilon^p$): WTI $\uparrow$,
  Inventories $\uparrow$.
\end{itemize}

Out of 100{,}000 random orthogonal rotations drawn in
\texttt{svar\_sign\_restrictions.m}, only a handful (six) satisfy all the
restrictions simultaneously. This low acceptance rate indicates that the
restrictions are informative and tightly pin down the economically relevant
decomposition of shocks. All structural results reported in the main text are
computed as medians across the accepted draws, with percentile bands used to
capture the remaining identification uncertainty.

\section*{B.4 Structural Impulse Responses and Variance Decomposition}

The structural impulse responses obtained from the accepted $B$ matrices reveal
a clear pattern:

\begin{itemize}
  \item aggregate-demand shocks produce the strongest and most persistent
  increases in the real WTI price, together with sustained increases in
  production and economic activity;
  \item flow supply shocks generate moderate but economically meaningful price
  responses that decay relatively quickly, with production falling on impact
  and gradually recovering;
  \item precautionary shocks display hump-shaped short-run dynamics in the oil
  price and inventories, consistent with temporary inventory accumulation driven
  by expectations of future scarcity.
\end{itemize}

The corresponding forecast error variance decomposition (reported in
Chapter~\ref{ch:empirical}) shows that demand shocks dominate the medium-run
variance of the real WTI price, while precautionary shocks contribute
disproportionately to very short-run volatility. Supply shocks play a more
limited but non-negligible role.

\section*{B.5 Marginal Distributions of Structural Shocks}

To analyse the distributional properties of structural shocks, the accepted
draws of $(\varepsilon_t^{s},\varepsilon_t^{d},\varepsilon_t^{p})$ are
assembled and standardised. The script \texttt{fit\_marginal\_shocks.m} then
fits several parametric families to each marginal by maximum likelihood,
including Gaussian, Logistic and Student-\emph{t} distributions.

Across all three shocks, Gaussian and Logistic specifications systematically
underestimate the empirical tail thickness. Student-\emph{t} marginals with
finite degrees of freedom provide a markedly better fit, capturing both the
leptokurtosis and, for demand and precautionary shocks, pronounced skewness.
These findings reinforce the evidence from pre-VAR diagnostics that
heavy-tailed shocks are a salient feature of oil-market dynamics and motivate
their use in the copula-based dependence analysis.

\section*{B.6 Copula-Based Dependence}

Dependence among structural shocks is studied using bivariate copulas fitted to
pairs $(\varepsilon_t^{i},\varepsilon_t^{j})$ after transforming each marginal
to the unit interval via the fitted Student-\emph{t} distributions. The
\texttt{copula\_fit\_shocks.m} script compares several families---Gaussian,
Student-\emph{t}, Clayton, Gumbel and Frank---using pseudo-likelihood criteria.

The main qualitative findings are:

\begin{itemize}
  \item a Student-\emph{t} copula with low degrees of freedom
  ($\nu \approx 3.4$) best describes the joint behaviour of aggregate-demand
  and precautionary shocks, implying strong upper-tail dependence: large
  positive demand shocks tend to coincide with large positive precautionary
  shocks;
  \item dependence between flow supply and precautionary shocks is weaker and
  more symmetric, with a Frank copula providing the best fit;
  \item dependence between flow supply and aggregate-demand shocks is moderate
  and close to elliptical, and can be captured reasonably well by either a
  Gaussian or a mild Student-\emph{t} copula.
\end{itemize}

These patterns suggest that episodes of strong global demand are often
accompanied by inventory-accumulation behaviour and expectation-driven
pressures, thereby amplifying oil price spikes. The copula analysis is used
here in a purely \emph{diagnostic} fashion: it characterises how shocks co-move
in the historical sample and motivates the construction of joint stress
scenarios, but it does not yet feed directly into the scenario generator of
Chapter~\ref{ch:stress}.

\section*{B.7 Conditional Tail Probabilities}

Using the fitted copulas, the script
\texttt{copula\_prob\_cond\_shocks.m} computes conditional probabilities of
joint extreme events. In particular, interest centres on the probability that
one shock exceeds a high quantile conditional on another shock being large:

\begin{itemize}
  \item the probability that the precautionary shock exceeds its 90th
  percentile given that the aggregate-demand shock is above its 90th
  percentile is found to be in the range of 35--45\%, consistent with strong
  upper-tail dependence between the two;
  \item in contrast, the probability that the precautionary shock is extreme
  conditional on a large negative flow supply shock remains below 10\%,
  reflecting the weaker and more symmetric dependence between supply and
  precautionary disturbances.
\end{itemize}

These conditional tail probabilities highlight that extreme demand episodes are
the most likely environment in which large precautionary shocks materialise,
providing a structural rationale for focusing stress scenarios on joint demand
and precautionary disturbances. A full integration of copula-based simulation
into the scenario generator is left to future research and outlined in
Chapter~\ref{ch:conclusions}.

\section*{B.8 Additional VAR Diagnostic: Residual Distribution}

Figure~\ref{fig:var-residuals-app} reports the histogram of standardised
residuals for the WTI equation in the VAR(12), overlaid with a standard normal
density. The heavy tails and mild asymmetry relative to the Gaussian benchmark
are consistent with the evidence discussed in Appendix~A and motivate the use
of bootstrap methods and heavy–tailed marginals in the copula analysis.

\begin{figure}[H]
    \centering
    \includegraphics[width=0.9\textwidth]{fig_var_residuals.pdf} % usa lo stesso file che avevi
    \caption{Standardised residuals of the WTI equation in the VAR(12) versus a standard normal density.}
    \label{fig:var-residuals-app}
\end{figure}


\clearpage
