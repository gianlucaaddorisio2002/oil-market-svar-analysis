\chapter{Data Construction and Pre-VAR Diagnostics}

\section*{A.1 Data Sources, Frequency Alignment and Storage}

The empirical analysis relies on monthly data spanning 1990--2024 for four
key variables: global crude oil production, OECD real activity (aggregate
industrial index), petroleum inventories and the real WTI price. Raw series
are gathered from the sources listed in Table~\ref{tab:data_sources} in
Chapter~\ref{ch:data-methods} and then harmonised to a common monthly calendar.

All series are first converted to end-of-month observations and aligned on a
shared time index. When a series is released at a different frequency or with
occasional missing values, the following conventions are applied:

\begin{itemize}
    \item quarterly or higher-frequency observations are aggregated or
    averaged to the monthly frequency, preserving the timing conventions of
    the original release;
    \item isolated missing entries are linearly interpolated only when this is
    necessary to avoid spurious gaps within otherwise continuous stretches of
    data; no series is forward- or backward-filled over extended periods;
    \item outlier values clearly attributable to data errors are replaced by
    the median of a narrow neighbourhood, with all adjustments documented in
    the \texttt{build\_oil\_dataset.m} script.
\end{itemize}

After alignment, three data structures are created and stored in
\texttt{clean\_data.mat}:

\begin{itemize}
    \item \texttt{All}: timetable of raw levels or log-levels (real WTI price,
    crude oil production, OECD activity index, petroleum inventories);
    \item \texttt{All\_d}: timetable of stationary transformations
    (first differences or $\Delta \log$ transforms), subsequently
    standardised to zero mean and unit variance;
    \item \texttt{ALL\_VAR}: numeric matrix used as input for the VAR
    estimation, with columns ordered as Production, OECD, WTI and Inventories
    and rows corresponding to common monthly dates.
\end{itemize}

The OECD activity index is chosen as the global demand proxy because, within
the sample, it displays a more stable dynamic relationship with the real WTI
price than the IGREA index, while avoiding some of the measurement revisions
associated with shipping-based indicators
\parencite{Kilian2009,Kilian2019}.

\section*{A.2 Transformations and Stationarity Checks}

The construction of \texttt{All\_d} follows the standard practice of rendering
the series approximately covariance-stationary before specifying a VAR.

\begin{itemize}
    \item Crude oil production, the real WTI price and inventories are
    transformed using $\Delta \log$ differences, which correspond
    approximately to monthly percentage changes.
    \item The OECD activity index is decomposed into a low-frequency trend and
    a cyclical component; the latter (denoted \texttt{OECD\_cycle}) is used in
    the VAR to capture global business-cycle fluctuations around the long-run
    level.
    \item Each stationary transformation is then standardised to zero mean and
    unit variance, so that the corresponding VAR coefficients and impulse
    responses are expressed in comparable units.
\end{itemize}

Augmented Dickey--Fuller (ADF) tests are applied both to the original levels
and to the transformed series. In levels, the null of a unit root cannot be
rejected for any of the four variables at conventional significance levels,
consistent with the large literature documenting non-stationarity in oil
prices, production, inventories and activity indicators. In contrast, all
transformed series in \texttt{All\_d} exhibit ADF $p$-values well below the
5\% threshold, supporting the use of the VAR in differences and cycle form.

To complement formal tests, pre-VAR diagnostics include visual inspection of
time-series plots, autocorrelation functions and ADF $p$-value charts for each
variable. These confirm that the transformed series fluctuate around a stable
mean with no obvious residual trends or explosive behaviour.

\section*{A.3 Baseline Static Regression and Diagnostics}

Before moving to a multivariate dynamic specification, a benchmark static
regression is estimated to assess whether contemporaneous linear relationships
between the transformed fundamentals and the oil price can explain a
meaningful share of its variation. The regression is specified as:
\[
WTI_{t}^{DL}
= \beta_0
+ \beta_1 \, \text{Production}_{t}^{DL}
+ \beta_2 \, \text{OECD}_{t}^{DL}
+ \beta_3 \, \text{Inventories}_{t}^{DL}
+ u_t,
\]
where the superscript $DL$ denotes $\Delta \log$-transformed variables.
Although this formulation already removes deterministic trends and stabilises
variances, the model attains only a very modest adjusted $R^2$ (approximately
0.11), indicating that contemporaneous fundamentals have limited explanatory
power for the monthly change in the real WTI price.

Residual diagnostics for this OLS specification reveal:

\begin{itemize}
  \item strong serial correlation, as indicated by Ljung--Box statistics with
  $p$-values well below 1\%;
  \item pronounced heteroskedasticity according to Breusch--Pagan tests
  ($p \approx 0.0000$);
  \item clear deviations from normality, with heavy tails relative to the
  Gaussian benchmark.
\end{itemize}

Rolling-window estimates of the coefficients further display substantial
instability across subsamples, especially around major structural episodes
such as the early-2000s boom and the post-2010 shale expansion. Together,
these diagnostics suggest that the static OLS framework is inadequate for
capturing the dynamics of oil prices and motivates the move to a multivariate,
dynamic VAR/SVAR approach.

\section*{A.4 Distributional Properties of OLS Residuals}

The residuals $u_t$ from the baseline regression provide an initial window on
the distributional features of unexplained oil-price movements. Empirical
density plots, histograms and normal Q--Q plots point to marked departures
from Gaussianity:

\begin{itemize}
    \item the residual distribution exhibits mild negative skewness, reflecting
    occasional large downward adjustments in the real WTI price;
    \item excess kurtosis indicates a leptokurtic shape, with more probability
    mass in the centre and in the tails than a normal distribution;
    \item extreme observations are clustered around well-known market events
    (price collapses and sharp spikes), rather than being isolated outliers.
\end{itemize}

These features justify the use of bootstrap methods in subsequent stages
(impulse-response confidence bands, scenario generation) and provide
preliminary evidence that non-Gaussian shocks may be structurally relevant in
the oil market. In particular, the heavy tails and asymmetric behaviour of
$u_t$ are consistent with the presence of large, infrequent structural
disturbances, which are modelled explicitly in the SVAR framework developed in
Chapters~\ref{ch:empirical} and~\ref{ch:stress}.

\clearpage
