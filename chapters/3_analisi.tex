\chapter{Empirical Analysis of Oil Market Dynamics}
\label{ch:empirical}

\section{Overview}

This chapter presents the empirical analysis of the oil market based on the multivariate dynamic framework developed in Chapter~\ref{ch:data-methods}. Building on the harmonised monthly dataset and the VAR(12) specification constructed through \texttt{build\_oil\_dataset.m} and \texttt{var\_main.m}, the objective is to identify and quantify the structural disturbances driving the real price of oil.

The analysis follows the structural VAR (SVAR) approach of \textcite{Kilian2009} and \textcite{KilianMurphy2014}, using sign and elasticity restrictions to identify three economically meaningful shocks: a flow supply shock, an aggregate demand shock, and a precautionary demand shock. Impulse responses, forecast error variance decompositions (FEVDs) and historical decompositions are used to characterise the transmission of these shocks. The final section maps structural shocks into WTI price trajectories, laying the groundwork for the scenario-based stress tests in Chapter~\ref{ch:stress}.

In addition to the VAR/SVAR analysis reported in this chapter, Appendix~B
provides an exploratory study of the marginal distributions and copula-based
dependence structure of the identified structural shocks.

\section{Reduced-Form VAR Results}
\label{sec:var-results}

\subsection{Estimation and Diagnostics}

The reduced-form VAR(12) is estimated on the stationary vector
\[
x_t =
\begin{bmatrix}
\Delta \log \text{Prod}_t \\
\text{OECD\_cycle}_t \\
\Delta \log \text{WTI}_t \\
\Delta \log \text{Inv}_t
\end{bmatrix},
\]
whose components correspond to the standardised transformations
\texttt{Production\_DL}, \texttt{OECD\_cycle}, \texttt{WTI\_real\_DL} and
\texttt{Inventories\_DL} in \texttt{All\_d}.

Residual diagnostics for the VAR(12) indicate (Table~\ref{tab:lag_selection},
Table~\ref{tab:var-fit}, Figure~\ref{fig:var-stability} and
Appendix~B}):

\begin{itemize}
    \item no significant residual autocorrelation, as Ljung--Box
    portmanteau $p$-values are well above conventional thresholds;
    \item all companion-matrix eigenvalues strictly inside the unit circle,
    confirming dynamic stability;
    \item mild departures from normality (Jarque--Bera tests) and evidence
    of heavy tails in the residual distribution of the WTI equation;
    \item overall, sufficient goodness-of-fit and stability for subsequent
    structural identification.
\end{itemize}

\begin{table}[H]
\centering
\caption{Reduced-form VAR(12) equation fit statistics.}
\label{tab:var-fit}
\renewcommand{\arraystretch}{1.1}
\begin{tabular}{lcccc}
\toprule
\textbf{Equation} & \textbf{Number of Parameters} & \textbf{Std.\ Error} & $\mathbf{R^2}$ & \textbf{Adj.\ $R^2$} \\
\midrule
Production\_DL  & 49 & 0.9358 & 0.2507 & 0.1468 \\
OCSE\_DL        & 49 & 0.7255 & 0.5493 & 0.4868 \\
WTI\_real\_DL   & 49 & 0.9620 & 0.1928 & 0.0808 \\
Inventories\_DL & 49 & 0.8205 & 0.3978 & 0.3143 \\
\bottomrule
\end{tabular}
\end{table}

\begin{figure}[h!]
    \centering
    \includegraphics[width=0.9\textwidth]{fig_var_stability.pdf}
    \caption{Companion matrix eigenvalues for the VAR(12). All lie inside the unit circle.}
    \label{fig:var-stability}
\end{figure}

To quantify the in-sample forecasting performance of the reduced-form model,
Table~\ref{tab:varfit} reports standard accuracy metrics for real WTI, based
on the script \texttt{var\_fit\_vs\_actual\_WTI.m}.

\begin{table}[H]
    \centering
    \caption{Accuracy metrics for VAR predicted vs.\ actual real WTI.}
    \label{tab:varfit}
    \begin{tabular}{lc}
        \toprule
        Metric & Value \\
        \midrule
        RMSE   & 12.28 \\
        MAE    & 10.66 \\
        MAPE   & 61.34\% \\
        \bottomrule
    \end{tabular}
\end{table}



\section{Structural Identification via Sign Restrictions}
\label{sec:svar-ident}

\subsection{Economic Restrictions}

Following \textcite{Kilian2009} and \textcite{KilianMurphy2014}, sign restrictions are imposed over horizons $h=0,1,2$:

\begin{itemize}
    \item \textbf{Flow supply shock:} production $\downarrow$, inventories $\downarrow$, real WTI $\uparrow$, OECD -- limited reaction.
    \item \textbf{Aggregate demand shock:} OECD $\uparrow$, production $\uparrow$, real WTI $\uparrow$, inventories $\uparrow$.
    \item \textbf{Precautionary demand shock:} inventories $\uparrow$, real WTI $\uparrow$, OECD $\approx$ unchanged.
\end{itemize}

Table~\ref{tab:sign-restrictions} summarises the full set of restrictions used in the baseline specification.
\begin{table}[H]
\centering
\caption{Sign restrictions used for structural identification (responses on impact).}
\label{tab:sign-restrictions}
\renewcommand{\arraystretch}{1.1}
\begin{tabular}{lccc}
\toprule
\textbf{Variable} & \textbf{Supply Shock} & \textbf{Aggregate Demand Shock} & \textbf{Precautionary Shock} \\
\midrule
Production   & $-$ & $+$ & $0$ \\
OECD\_cycle  & $0$ & $+$ & $0$ \\
Inventories  & $-$ & $+$ & $+$ \\
Real WTI     & $+$ & $+$ & $+$ \\
\bottomrule
\end{tabular}
\end{table}



\subsection{Elasticity Bounds}

Elasticity constraints follow \textcite{KilianMurphy2014}:

\[
\varepsilon^s_p \in [-0.05, 0], \qquad \varepsilon^d_p \in [-0.2, -0.01].
\]

These enforce plausible short-run responses of supply and demand to price movements.

\subsection{Implementation}

The routine \texttt{svar\_sign\_restrictions.m} implements the
rotation-based procedure of Rubio-Ramirez, Waggoner and Zha~\cite{RubioRamirez2010}.
Starting from the reduced-form covariance matrix $\Sigma_u$, the algorithm
computes its Cholesky factor $P$, draws random orthonormal matrices $Q$ from
the Haar distribution, and constructs candidate impact matrices $B = P Q$.
A draw is retained only if the corresponding impulse responses satisfy all
sign restrictions over the horizon $h = 0,\dots,H$; otherwise it is discarded.
In the baseline specification, accepted draws represent about 1--3\% of all
candidates, in line with Kilian and Murphy~\cite{KilianMurphy2014}.


\section{Impulse Response Functions}
\label{sec:irfs}

Because the VAR is estimated on stationary log-differences (and on the
detrended OECD activity index), the raw impulse responses describe the effect
of each structural shock on monthly growth rates. For economic interpretation,
impulse responses for the levels are obtained by cumulating the responses of
log-differences over the horizon. The figures reported in this section refer to
these cumulated responses and can therefore be read as percentage deviations of
the levels from their baseline paths.

Impulse responses (IRFs) are computed over a 36-month horizon using bootstrap
percentile bands. For each shock, we report the median and the 16th--84th
percentiles across accepted structural decompositions.

Table~\ref{tab:shock-stats} summarises the basic distributional properties of
the identified shocks, confirming that they are approximately mean-zero and
exhibit non-Gaussian higher moments.

\begin{table}[H]
\centering
\caption{Descriptive statistics of identified structural shocks.}
\label{tab:shock-stats}
\renewcommand{\arraystretch}{1.1}
\begin{tabular}{lcccc}
\toprule
\textbf{Shock} & \textbf{Mean} & \textbf{Std.\ Dev.} & \textbf{Skewness} & \textbf{Kurtosis} \\
\midrule
Supply           & 0.0000 & 1.0013 & 0.0973   & 3.6076  \\
Aggregate Demand & 0.0000 & 1.0013 & $-0.5293$ & 4.7428 \\
Precautionary    & 0.0000 & 1.0013 & 1.7476   & 16.5490 \\
\bottomrule
\end{tabular}
\end{table}


\subsection{Flow Supply Shock}

A negative flow supply shock --- interpreted as an unanticipated shortfall in
U.S.\ crude oil production --- generates the following pattern in the cumulated
responses:

\begin{itemize}
    \item an immediate rise in the real WTI price, peaking after roughly
          3--4 months before gradually declining;
    \item a decline in crude oil production on impact, with only partial
          recovery over the subsequent year;
    \item an inventory drawdown consistent with the use of stocks to smooth
          the temporary supply shortfall;
    \item a muted and statistically imprecise reaction in OECD activity.
\end{itemize}

This configuration is qualitatively consistent with the supply-driven oil price
episodes documented by \textcite{Kilian2009}.

\begin{figure}[h!]
    \centering
    \includegraphics[width=1\textwidth]{fig_irf_supply.png}
    \caption{Impulse responses to a flow supply shock.}
    \label{fig:irf-supply}
\end{figure}
\subsection{Aggregate Demand Shock}

An aggregate demand expansion, capturing shifts in global economic activity,
produces:

\begin{itemize}
    \item a strong and persistent rise in OECD activity;
    \item an increase in crude oil production and inventories, reflecting the
          response of supply and stock-building to stronger demand;
    \item a sustained increase in the real WTI price that remains positive for
          roughly 12--18 months.
\end{itemize}


\begin{figure}[H]
    \centering
    \includegraphics[width=1\textwidth]{figu.png}
    \caption{Impulse responses to an aggregate demand shock.}
    \label{fig:irf-demand}
\end{figure}

\subsection{Precautionary Demand Shock}

A precautionary demand shock --- interpreted as a revision in expectations of
future scarcity --- yields:

\begin{itemize}
    \item an immediate and sizeable increase in inventories, as market
          participants build precautionary stocks;
    \item a short-lived but sharp increase in the real WTI price, which peaks
          within a few months and then dissipates;
    \item negligible and statistically weak effects on OECD activity and crude
          oil production.
\end{itemize}


\begin{figure}[H]
    \centering
    \includegraphics[width=1\textwidth]{figuta36.png}
    \caption{Impulse responses to a precautionary demand shock.}
    \label{fig:irf-inventory}
\end{figure}


\section{Forecast Error Variance Decomposition}
\label{sec:fevd}

The FEVD quantifies the importance of each shock in explaining forecast error variance at different horizons. Results are consistent with \textcite{Kilian2009,BaumeisterHamilton2019}:

\begin{itemize}
    \item supply shocks: limited contribution at short horizons;
    \item aggregate demand shocks: dominant at 6--18 months;
    \item precautionary shocks: relevant for short-run volatility.
\end{itemize}

\subsection{Multipanel FEVD Overview}

To provide a compact view of the contribution of each structural shock to the
variability of the real WTI price, Figure~\ref{fig:fevd-svar} reports the
SVAR-based forecast error variance decomposition over horizons up to 24 months.
Aggregate demand shocks account for roughly 45--50\% of the forecast error
variance at horizons between 6 and 18 months, while flow supply and
precautionary shocks each explain about 10\% in the long run.

\begin{figure}[h!]
    \centering
    \includegraphics[width=\textwidth]{svar_wti.png}
    \caption{SVAR-based forecast error variance decomposition of the real WTI
    price by structural shock (flow supply, aggregate demand, precautionary).}
    \label{fig:fevd-svar}
\end{figure}

The identification scheme and the main properties of the structural shocks are
summarised in Appendix~B.


\section{Historical Decomposition}
\label{sec:hist-decomp}

Historical decomposition assigns each observed movement in real WTI to one of the structural shocks. Using \texttt{extract\_structural\_shocks.m}, contributions are reconstructed across the full sample.

The decomposition indicates:

\begin{itemize}
    \item 2003--2008 oil price surge: predominantly aggregate demand;
    \item 2008 collapse: fall in aggregate demand + unwinding of precautionary positions;
    \item 2014--2016 decline: persistent positive supply shocks (shale expansion).
\end{itemize}

\section{Mapping Structural Shocks to Price Trajectories}
\label{sec:shock-mapping}

To generate the scenario-based projections in Chapter~\ref{ch:stress},
structural shocks must be translated into paths for the level of the real WTI
price. Since the VAR is estimated on stationary log-differences, the impulse
responses are first cumulated to obtain the effect of each shock on the log
level of the real price.

Let $\theta_h^{(j)}$ denote the \emph{cumulated} impulse response of log real
WTI at horizon $h$ to a unit structural shock $j \in \{s,d,p\}$ (flow supply,
aggregate demand, precautionary). For a shock of magnitude $\varepsilon_j$, the
deviation of log real WTI from its baseline path at horizon $h$ is

\[
    \Delta \log \text{WTI}_{t+h}
    \;=\; \varepsilon_j \, \theta_h^{(j)}.
\]

Given a baseline real price level $P_0$, the corresponding price path in levels
is obtained by exponentiating these deviations:

\[
    P_{t+h}^{(j)}
    \;=\;
    P_0 \exp\!\big( \varepsilon_j \, \theta_h^{(j)} \big).
\]

This mapping is implemented in \texttt{extract\_wti\_irf\_from\_var.m}, which
reads the SVAR impulse responses, cumulates the log-difference responses and
returns scenario-specific trajectories for the real WTI price. These paths are
then used in Chapter~\ref{ch:stress} to construct the baseline, supply-shock
and demand-shock scenarios for the stress-testing exercise.
