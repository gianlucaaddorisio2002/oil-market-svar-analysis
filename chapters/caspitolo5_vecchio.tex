\chapter{Conclusion and Final Remarks}
\label{chap:conclusion}

The present study analysed the dynamics of the real oil price through a structural VAR model based on the distinction between supply shocks, aggregate demand shocks, and precautionary (oil-specific) shocks, following the established methodological approach in the literature (Kilian, 2009; Kilian and Murphy, 2014).

The analysis was conducted through four main components: a review of the theoretical foundations (Chapter~\ref{chap:theory}), selection of time series and methodological design (Chapter~\ref{chap:data_methods}), empirical estimation and results (Chapter~\ref{chap:empirical}), and critical interpretation with practical implications (Chapter~\ref{chap:discussion}).


\section{Summary of Key Findings}

Three core empirical findings emerge from the analysis:

\begin{enumerate}
    \item Supply shocks have a limited and transitory impact on the real oil price.
    \item Aggregate demand shocks contribute to price fluctuations, but less persistently than expected in pre-2000 models.
    \item Precautionary shocks represent the primary driver of real oil price dynamics in the analysed sample, with persistent and statistically relevant effects.
\end{enumerate}

These results confirm the structural transformation of the oil market over time, highlighting a gradual shift from fundamental-driven dynamics toward mechanisms shaped by expectations, anticipations, and uncertainty.


\section{Contribution of the Study}

The main contribution of this work lies in its ability to frame recent oil-market behaviour within a recognised methodological structure and verify its stability in the U.S. context. Specifically:

\begin{itemize}
    \item the choice of WTI as a benchmark was validated through robustness checks against Brent, confirming model consistency and higher stability;
    \item the structural representation of the market enables an economic interpretation of results, going beyond purely statistical descriptions;
    \item integrating REA as a proxy for global demand confirms the relevance of macro-industrial activity in price formation, while not being the sole determining factor.
\end{itemize}


\section{Limitations}

While providing robust evidence, the analysis presents certain limitations:

\begin{itemize}
    \item The structure of the model assumes linearity and stationarity, while recent literature suggests the presence of non-linearities and structural breaks.
    \item The large residual share in the FEVD indicates that relevant variables — particularly financial and geopolitical indicators — are not included in the model.
    \item The use of monthly data: a higher-frequency dataset may capture faster-moving dynamics typical of modern energy markets.
\end{itemize}

These limitations do not undermine the validity of the findings, but they delineate the interpretive scope of the study.


\section{Future Research Directions}

The evidence obtained suggests several avenues for further development:

\begin{enumerate}
    \item Extending the model to non-linear frameworks (MS-VAR, Threshold VAR).
    \item Including financial variables such as VIX, futures positioning, and implied volatility measures.
    \item Running a multicountry comparison across WTI–Brent–Dubai–OPEC Reference Basket benchmarks.
    \item Performing rolling-window analysis to test temporal stability of shock responses.
\end{enumerate}


\section{Final Remarks}

Overall, the results confirm the complex, expectation-driven, and only partially observable nature of the contemporary oil market. The study highlights that oil is not merely a physical commodity, but a financial asset sensitive to perceptions, risks, and global expectations.

The analysis presented here aims to contribute to a more structured understanding of this phenomenon while providing a methodological basis for future extensions.
