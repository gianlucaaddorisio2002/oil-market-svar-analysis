\chapter{Introduction}

\section{Motivation and Background}

    Understanding the behaviour of crude oil prices is critical for macroeconomic
    analysis, financial stability and industrial decision-making. Oil remains the world's most
    strategic commodity, influencing inflation, business cycles, transportation networks and
    geopolitical dynamics. Because crude oil serves both as a physical input and as a financial asset,
    its price reflects a complex interaction between real fundamentals, expectations, inventories,
    speculative pressures and policy events. As argued by \textcite{BarskyKilian2002, BarskyKilian2004},
    macroeconomic outcomes associated with oil price movements cannot be understood through
    a simplistic focus on physical supply disruptions alone.

    A central insight of the modern literature is that oil price fluctuations predominantly reflect
    global aggregate demand conditions and changes in precautionary inventories rather than
    supply-side disturbances. In the structural decomposition of \textcite{Kilian2009}, shocks to global
    real economic activity account for the majority of medium-run oil price variation, while
    precautionary demand shocks --- captured through inventory dynamics --- generate sharp price
    spikes during episodes of uncertainty about future availability. More recent contributions
    \parencite{KilianMurphy2014, BaumeisterHamilton2019} confirm these findings and refine the
    understanding of speculative motives and inventory behaviour.

    Despite substantial progress in modelling, a persistent misconception in public and policy discussions is that oil price spikes are mainly supply-driven. Structural evidence, however, contradicts this view: most large post-1990 price surges were demand-led rather than the result of physical shortages \cite{KilianZhou2020}. Historical accounts further reinforce this interpretation, showing that many so-called “supply crises” were in fact driven by shifts in global demand, precautionary behaviour, or broader macroeconomic conditions \cite{Carollo2010}.
     However, the
    empirical evidence overwhelmingly shows that supply shocks are relatively rare, generally small
    in magnitude and have limited persistence. The short-run price elasticity of crude oil supply is
    extremely low, reflecting physical and technological constraints in the extraction sector.
    Consequently, even small shifts in global economic activity or revisions in expectations can
    produce outsized price movements.

    The complexity of the oil market extends beyond macroeconomic theory. For industries
    that rely heavily on refined petroleum products --- particularly aviation --- fluctuations in oil and
    jet fuel prices translate into substantial financial risk. Fuel expenses typically represent between
    20\% and 35\% of airline operating costs \parencite{IATA_FuelCosts2024}. This exposure
    makes airlines highly vulnerable to extreme price movements arising from macroeconomic or
    geopolitical shocks. For this reason, the aviation sector requires robust risk management tools
    capable of mapping crude oil disturbances into jet fuel prices and quantifying the impact on
    cash-flow and budgeting decisions. 
    \newline
    
    Modern risk management increasingly relies on scenario-based methods that integrate
    economic fundamentals. Firms can no longer depend on simple forecasting tools or static
    projections, as such models fail to capture the structural origins of volatility. Instead, firms look
    towards econometric frameworks capable of isolating the underlying drivers of price
    movements and generating both probabilistic scenarios and adversarial stress tests
    \parencite{AndersonKelloggSalant2018, SockinXiong2015}. Structural VAR (SVAR) models, in particular, offer
    a theoretically grounded method for disentangling supply, demand and inventory shocks, while
    copula-based approaches allow researchers to characterise nonlinear dependence and
    tail-risk interactions across shocks. 
\\
\begin{figure}[H]
    \centering
    \includegraphics[width=0.99\textwidth]{fig_stylised_oil_market.png}
    \caption{Stylised Facts of the Oil Market, 1990--2024: Real WTI Price, OECD Activity, Oil Production, Oil Inventories. (\textit{Source}: own elaboration on EIA, FRED and OECD data)}
    \label{fig:stylisedfacts}
\end{figure}

Figure~\ref{fig:stylisedfacts} presents a visual overview of the main variables used throughout
the thesis. The co-movement between global activity and crude oil prices is striking, especially
during episodes such as the 2003--2008 expansion and the COVID-19 contraction. Production
adjusts slowly, while inventories exhibit spikes during periods of heightened uncertainty. These
stylised facts motivate the adoption of a multivariate dynamic model capable of accounting
for these interactions.

\section{Problem Statement}

Despite extensive research, economists and practitioners face several persistent challenges in
modelling and interpreting oil price dynamics.

First, the oil market is inherently multivariate. Real oil prices respond to production, global
business cycles, inventory adjustments and financial market conditions. Univariate models, including autoregressive specifications and static regressions, cannot reproduce these joint dynamics. Formal diagnostic tests confirm the inadequacy of such models: regressions of the
real price of oil on production, activity and inventories exhibit strong serial correlation in the
residuals \parencite{LjungBox1978}, heteroscedasticity \parencite{BreuschPagan1979} and clear departures from
normality \parencite{JarqueBera1987, Massey1951}. Coefficient instability across subsamples further
suggests that relationships are time-varying and regime-dependent \parencite{Brown1975, BaiPerron2003}.

Second, reduced-form VARs, while effective in capturing dynamic interactions, do not provide economically meaningful interpretations of shocks unless a structural identification strategy is imposed. As emphasised by \textcite{Sims1980} and formalised in the oil-market context by \textcite{Kilian2009}, theory-based restrictions are essential for distinguishing supply, aggregate-demand and precautionary-demand innovations. Building on these principles, \textcite{RubioRamirez2010} show how identification schemes can be derived in a fully Bayesian framework. Without economically motivated restrictions, impulse responses cannot be interpreted in terms of underlying structural mechanisms.


Third, extreme events play a central role in the oil market. Structural shocks display
heavy-tailed distributions, meaning that rare but impactful events occur with higher probability
than implied by Gaussian models. Modelling these extremes requires flexible distributional
assumptions and tools capable of capturing tail dependence, such as copulas \parencite{Nelsen2006,
Patton2012}.

Finally, translating structural oil shocks into industrial risk metrics requires bridging two
domains: macroeconometric modelling and corporate finance. Firms need tools that not only
identify the origins of price movements, but also translate them into actionable scenarios,
stress tests and hedging implications. This is particularly relevant for airlines, where fuel-price
risk affects fleet planning, pricing strategies and financial performance.

\section{Theoretical Framework}

The theoretical backbone of this thesis is the structural VAR framework originally proposed by
\textcite{Kilian2009} and later refined by \textcite{KilianMurphy2014}. In this framework, the global oil market
is driven by three fundamental shocks:

\begin{enumerate}
    \item \textbf{Flow supply shocks} --- unexpected changes in global crude oil production.
    \item \textbf{Aggregate demand shocks} --- shifts in global real economic activity, reflecting industrial
    demand for commodities.
    \item \textbf{Precautionary or inventory demand shocks} --- changes in expectations about future oil
    availability, inferred through adjustments in inventories.
\end{enumerate}

Flow supply shocks typically have limited and short-lived effects on oil prices due to the
inelasticity of near-term production. By contrast, aggregate demand shocks produce large and
persistent price movements, reflecting the procyclical nature of industrial commodity markets.
Precautionary demand shocks capture periods of heightened uncertainty, during which firms and
speculators increase inventory holdings in anticipation of potential future shortages
\cite{Kilian2009,BaumeisterHamilton2019}.


\begin{figure}[t!]
\centering
\begin{tikzpicture}[
    >=Latex,
    box/.style={
        rectangle,
        draw,
        rounded corners,
        minimum height=1.1cm,
        minimum width=4.8cm,
        align=center,
        thick,
        fill=gray!5
    },
    arrow/.style={->, very thick}
]

% Colonna shock a sinistra
\node[box] (supply)    at (0, 2.0) {Flow Supply Shock};
\node[box] (demand)    at (0, 0.0) {Aggregate Demand Shock};
\node[box] (inventory) at (0,-2.0) {Precautionary Shock};

% VAR al centro
\node[box, minimum width=5.0cm] (var) at (6,0) {Structural VAR System};

% Prezzo a destra
\node[box, minimum width=4.8cm] (price) at (11,0) {Real Price of Oil};

% Frecce shock -> VAR
\draw[arrow] (supply.east)    to[bend left=15]  (var.north west);
\draw[arrow] (demand.east)    --                (var.west);
\draw[arrow] (inventory.east) to[bend right=15] (var.south west);

% Freccia VAR -> prezzo
\draw[arrow] (var.east) -- (price.west);

\end{tikzpicture}
\caption{Conceptual structure of the oil market SVAR model.}
\label{fig:varconcept}
\end{figure}




Figure~\ref{fig:varconcept} provides a schematic representation of the structural model. Each
shock affects oil prices through distinct channels, though real-world dynamics often involve
interactions between shocks. The structural VAR allows researchers to quantify these effects
through impulse response functions (IRFs) and forecast error variance decompositions (FEVD).

Beyond the VAR, this thesis incorporates distributional analysis of structural shocks, building
on the observation that heavy tails and nonlinear dependence are key features of global
commodity markets. Understanding tail dependencies is crucial for designing robust stress
tests and risk scenarios.

\section{Research Objectives}

This thesis pursues four overarching objectives:

\begin{enumerate}
    \item \textbf{To construct a harmonised and comprehensive monthly dataset (1990--2024)} including
    real crude oil prices, global crude oil production, OECD industrial activity and U.S. petroleum
    and product inventories, using official sources such as the EIA and the Federal Reserve
    \parencite{WTI_FRED, Inventories_EIA, OilProduction_EIA}.
    
    \item \textbf{To estimate and validate a reduced-form VAR model} that captures the dynamic
    interactions between these variables, supported by extensive diagnostic testing
    \parencite{Lutkepohl2005}.
    
    \item \textbf{To identify structural shocks using sign restrictions and elasticity bounds}, following the
    methodology of \textcite{KilianMurphy2014}, and to characterise their impulse responses and
    variance contributions.
    
    \item \textbf{To translate structural oil shocks into industrial risk metrics}, including Monte Carlo
    scenario generation, Hormuz-type stress testing and hedging implications for a commercial
    airline exposed to jet fuel price fluctuations.
\end{enumerate}

\section{Methodological Contribution}

This thesis makes several methodological contributions to the empirical analysis of oil markets
and their industrial applications:

\begin{itemize}
    \item \textbf{Data construction and proxy evaluation.}  
    It provides a harmonised dataset covering more than three decades of global oil market
    activity. A key contribution is the evaluation of alternative proxies for global real activity,
    comparing OECD-based indicators to the \textit{REA} index \parencite{IGREA_DallasFed, Kilian2019}.
    
    \item \textbf{Structural identification.}  
    The thesis implements a robust identification strategy based on sign and elasticity
    restrictions, ensuring economically meaningful structural shocks consistent with the oil
    market literature \parencite{KilianMurphy2014, BaumeisterHamilton2019}.
    
    \item \textbf{Distributional characterisation of shocks.}  
    It documents heavy-tailed marginal distributions and copula-based dependence structures
    across structural shocks, revealing substantial tail dependence between demand and
    precautionary disturbances \parencite{Nelsen2006, Patton2012}.
    
    \item \textbf{Stress-testing framework.}  
    The thesis develops a scenario generation and stress-testing framework grounded in
    structural econometrics and applicable to industrial decision-making, particularly the
    aviation sector.
    
    \item \textbf{Application to fuel risk management.}  
    The thesis bridges macroeconomic modelling with corporate risk management by mapping
    structural oil shocks into jet fuel price paths and assessing the impact on an airline's fuel
    cost exposure.
\end{itemize}

\begin{table}[t!]
\centering
\caption{Key structural mechanisms in oil markets.}
\label{tab:mechanisms}
\setlength{\tabcolsep}{6pt}
\renewcommand{\arraystretch}{1.25}
\begin{tabular}{p{3.2cm}p{7.2cm}p{3.4cm}}
\toprule
\textbf{Mechanism} & \textbf{Description} & \textbf{References} \\
\midrule
Flow supply        & Slow adjustment due to extraction and investment constraints & Kilian \& Murphy (2014) \\
Aggregate demand   & Driven by the global business cycle and industrial activity & Barsky \& Kilian (2002, 2004) \\
Inventory demand   & Reflects expectations about future availability             & Alquist, Bhattarai \& Coibion (2019) \\
Speculation        & Amplifies short-run volatility                              & Sockin \& Xiong (2015) \\
\bottomrule
\end{tabular}
\end{table}


\section{Structure of the Thesis}

The thesis is organised into five chapters:

\begin{itemize}
    \item \textbf{Chapter 2} introduces the dataset, details preprocessing steps and presents the
    econometric methodology, including stationarity analysis and the specification of the
    reduced-form VAR model.
    
    \item \textbf{Chapter 3} presents the estimation results of the VAR model and the identification of
    structural shocks. It reports impulse responses, variance decompositions and the
    transmission of shocks to oil prices.
    
    \item \textbf{Chapter 4} examines the empirical distributions of structural shocks, estimates copula
    models to capture nonlinear dependence and applies the structural results to scenario
    generation, stress testing and hedging.
    
    \item \textbf{Chapter 5} concludes by summarising the main findings, discussing limitations and
    proposing avenues for future research.
\end{itemize}

These considerations motivate the multivariate modelling framework employed in the remainder of this thesis.


