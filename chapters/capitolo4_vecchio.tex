\chapter{Discussion and Practical Implications}
\label{chap:discussion}

This chapter discusses and interprets the econometric results presented in Chapter~\ref{chap:empirical}, linking them to the theoretical framework outlined in Chapters~\ref{chap:theory} and~\ref{chap:data_methods}. 
The objective is not to restate the results, but to provide a coherent economic interpretation, compare the empirical evidence with established findings in the literature, and draw operational implications for economic agents potentially affected, with particular attention to transport and logistics companies.


\section{Interpretation of the Structural Evidence}
\label{sec:interpretation}

The SVAR estimation results highlight three main findings: (i) supply shocks have a moderate and short-lived impact, (ii) aggregate demand shocks produce more visible but non-persistent effects, and (iii) precautionary shocks emerge as the dominant force in the dynamics of the real oil price.

These results are consistent with the theoretical expectation that the oil market is characterized by short-run supply rigidity and a strong anticipatory component in expectations. In particular, the persistence of the response to precautionary shocks suggests that the market does not react exclusively to fundamentals, but also incorporates information, uncertainty, and sentiment, as emphasized by Kilian (2009) and Kilian and Murphy (2014).

A second relevant aspect is the structure of the Forecast Error Variance Decomposition, according to which the share explained by fundamentals is limited relative to the unobserved residual component. This implies that non-modelled shocks — such as financial dynamics, monetary policy, geopolitical tensions, or logistical constraints — play a substantial role in determining the oil price. In other words, crude oil appears to function not only as a physical commodity, but also as a financial asset.


\section{Comparison with Existing Literature}
\label{sec:literature_comparison}

The empirical evidence obtained aligns with the modern oil-market literature, which has progressively reassessed the role of supply shocks, historically considered central in interpretations from the 1970s–1990s.

Compared with the findings of Kilian (2009) and Kilian and Murphy (2014), the empirical picture confirms two key points:

\begin{itemize}
    \item precautionary demand (oil-specific demand) is a significant driver of price volatility;
    \item aggregate demand shocks influence prices only during periods in which global growth deviates substantially from expectations.
\end{itemize}

At the same time, the large residual share of variance represents a partial deviation from the literature, suggesting that in the specific period analysed, the speculative dimension and geopolitical uncertainty played a comparatively more pronounced role than industrial cycles.


\section{Implications for Transport and Logistics Firms}
\label{sec:implications}

From an applied perspective, the SVAR estimation results suggest that fuel-cost dynamics for transport and logistics firms cannot be managed by assuming a simple and deterministic relationship between macroeconomic fundamentals and oil prices.

Since precautionary shocks are dominant and persistent, firms cannot rely exclusively on traditional indicators (production, industrial demand, macro indices) for hedging strategies, but must instead integrate:

\begin{itemize}
    \item forward-looking indicators;
    \item measures of global uncertainty;
    \item signals of tension related to inventories and industrial logistics.
\end{itemize}

In practice, the most relevant finding is that the oil price reacts to expectations before fundamentals. This implies that instruments such as futures, energy swaps, or flexible indexation contracts may reduce vulnerability to unexpected shocks.


\section{Strengths and Limitations of the Analysis}
\label{sec:limits}

The present model provides a coherent and methodologically solid structural representation of oil market dynamics. However, several limitations must be acknowledged:

\begin{itemize}
    \item the high residual component indicates that relevant variables are not included in the model;
    \item the model assumes a linear and time-invariant structure, whereas preliminary evidence in Chapter~\ref{chap:theory} indicated possible non-linearities and regime switching;
    \item the SVAR captures average relationships over time but not their structural evolution.
\end{itemize}

These limitations do not invalidate the results but clearly define the scope of the analysis.


\section{Directions for Further Research}
\label{sec:future_work}

Based on the findings, three possible future extensions are identified:

\begin{itemize}
    \item implementation of non-linear models (Markov-Switching VAR);
    \item integration of financial variables (implied volatility, futures positioning, dollar index);
    \item multi-benchmark comparison with Brent, Dubai Fateh, and a Weighted Crude Index.
\end{itemize}

Such extensions would allow a more accurate understanding of the role of expectations and the financial dimension of oil — elements that are central to contemporary oil market dynamics.
ca