\chapter{Dispensa}
\section{Dalla Regressione Lineare Statica al Modello VAR Dinamico}

In una prima fase dell’analisi è stato stimato un modello lineare statico di tipo OLS con variabile dipendente la variazione del prezzo del petrolio (WTI) e tre regressori fondamentali: produzione di petrolio, attività economica globale (REA) e scorte. Il modello, stimato tramite il comando \texttt{fitlm} di MATLAB, ha restituito risultati coerenti con quanto atteso dalla letteratura sulle serie energetiche: bassa capacità esplicativa, scarsa significatività dei coefficienti e violazioni sistematiche delle assunzioni classiche dell’OLS.

\subsection{Limiti del modello statico}
La diagnostica condotta sui residui ha mostrato:
\begin{itemize}
    \item \textbf{autocorrelazione seriale} (Ljung--Box sempre significativo);
    \item \textbf{eteroschedasticità} (Breusch--Pagan con $p=0$);
    \item \textbf{non normalità nelle code} (deviazioni marcate nel QQ-plot e presenza di heavy tails);
    \item \textbf{R\textsuperscript{2} estremamente basso} e \textbf{coefficiente di determinazione corretto quasi nullo};
    \item \textbf{instabilità parametrica nel tempo}, confermata dall’analisi rolling.
\end{itemize}

Tali evidenze indicano che il modello statico non riesce a catturare la dinamica del processo generatore dei dati, caratterizzato da forte dipendenza temporale, non linearità, shock e comportamenti asimmetrici nelle code. L’OLS si rivela quindi inadeguato per descrivere la relazione tra variabili macroeconomiche e prezzo del petrolio.

\subsection{Motivazione per il passaggio al modello VAR}
Per superare i limiti del modello statico, è stato adottato un modello VAR (Vector Autoregression), che consente di:
\begin{itemize}
    \item modellare simultaneamente le interazioni dinamiche tra più variabili endogene;
    \item catturare la propagazione degli shock nel tempo;
    \item incorporare memoria tramite valori ritardati (lag);
    \item produrre strumenti di analisi dinamica quali Impulse Response Functions (IRF) e Forecast Error Variance Decomposition (FEVD).
\end{itemize}

A differenza dell’OLS, che impone una struttura unidirezionale e contemporanea, il VAR consente a ciascuna variabile di influenzare e di essere influenzata dalle altre nel tempo. La specificazione stimata è un modello VAR(p), selezionato sulla base di criteri di stabilità e diagnosi dei residui, con $p=7$ ritardi mensili. Questo approccio segue la tradizione della letteratura sui mercati petroliferi (Kilian, 2009; Kilian and Murphy, 2014), dove l’utilizzo di un numero elevato di lag è standard per garantire residui non autocorrelati.

\subsection{Cosa si conserva e cosa cambia rispetto all’OLS}
L’analisi OLS preliminare fornisce informazioni utili, ma limitate:
\begin{itemize}
    \item si conferma che la domanda globale (REA) mostra il legame più consistente con il WTI;
    \item produzione e scorte risultano debolmente informative in forma contemporanea;
    \item la struttura delle code e la non linearità richiedono modelli più flessibili.
\end{itemize}

Il modello VAR introduce invece nuovi elementi fondamentali:
\begin{itemize}
    \item \textbf{dinamica temporale} tramite valori ritardati;
    \item \textbf{relazioni bi-direzionali} tra tutte le variabili;
    \item possibilità di identificare \textbf{shock strutturali} (supply, demand, precautionary);
    \item \textbf{analisi IRF e FEVD} per studiare come gli shock si propagano e quale quota della varianza del WTI essi spiegano;
    \item base per la \textbf{Historical Decomposition} e per applicazioni di risk management.
\end{itemize}

\subsection{Struttura del modello VAR stimato}
Il modello stimato è un VAR(7) sulle serie stazionarie \textit{Production\_DL}, \textit{REA\_DL}, \textit{WTI\_real\_DL} e \textit{Inventories\_DL}, definito come:
\[
    Y_t = A_1 Y_{t-1} + A_2 Y_{t-2} + \dots + A_7 Y_{t-7} + u_t,
\]
dove $Y_t$ contiene le quattro variabili del sistema. Il modello risulta stabile e i residui non presentano autocorrelazione, garantendo la validità econometrica delle IRF e delle decomposizioni successive.

\subsection{Obiettivi dell’analisi VAR}
L’adozione del modello VAR consente di:
\begin{enumerate}
    \item valutare la risposta dinamica delle variabili agli shock fondamentali;
    \item quantificare l’importanza relativa di ciascuno shock (FEVD);
    \item identificare la natura dei movimenti del prezzo del petrolio (supply, demand, precautionary);
    \item fornire la base per applicazioni di \textbf{hedging}, \textbf{stress test} e \textbf{scenario analysis} per imprese esposte al rischio energetico.
\end{enumerate}

In sintesi, il passaggio dal modello OLS al modello VAR rappresenta un’evoluzione necessaria e metodologicamente fondata, che permette di descrivere in modo realistico il comportamento dinamico del mercato petrolifero.

\section{Dal modello OLS statico al VAR dinamico e al VAR strutturale}

In una prima fase dell'analisi è stato stimato un modello lineare statico (OLS) che mette in relazione la variazione del prezzo reale del petrolio con tre fondamentali osservabili del mercato:

\begin{equation}
\Delta \log(\text{WTI}_t) 
= \alpha 
+ \beta_1 \Delta \log(\text{Production}_t)
+ \beta_2 \Delta \text{REA}_t
+ \beta_3 \Delta \log(\text{Inventories}_t)
+ \varepsilon_t.
\end{equation}

I risultati empirici di questa regressione mostrano:
\begin{itemize}
    \item un coefficiente di determinazione $R^2$ molto basso e $Adj.\,R^2$ prossimo a zero;
    \item un numero limitato di coefficienti statisticamente significativi;
    \item residui affetti da autocorrelazione (test di Ljung--Box), eteroschedasticità (Breusch--Pagan) e deviazioni dalla normalità nelle code;
    \item evidenza, tramite analisi rolling, di instabilità del legame nel tempo.
\end{itemize}

In sintesi, il modello OLS statico ha un contenuto informativo limitato: le variabili fondamentali considerate (produzione, attività economica, scorte) spiegano solo una frazione minima della variabilità di breve periodo del prezzo del petrolio, e le ipotesi classiche dell'OLS risultano sistematicamente violate. Ciò suggerisce che la dinamica del mercato petrolifero non possa essere descritta in termini di relazioni contemporanee e statiche, ma richieda un modello esplicitamente dinamico.

\subsection{Adozione di un modello VAR dinamico}

Per superare i limiti del modello statico è stato stimato un modello VAR sulle serie stazionarie in differenze logaritmiche:

\begin{equation}
Y_t = 
\begin{bmatrix}
\Delta \log(\text{Production}_t) \\
\Delta \text{REA}_t \\
\Delta \log(\text{WTI}_t) \\
\Delta \log(\text{Inventories}_t)
\end{bmatrix}
=
A_1 Y_{t-1} + A_2 Y_{t-2} + \dots + A_p Y_{t-p} + u_t,
\end{equation}

dove $u_t$ è il vettore degli errori ridotti-forma. Tramite una procedura incrementale si è considerato un ordine $p$ da 1 a 8, verificando per ciascun modello:
\begin{enumerate}
    \item la \emph{stabilità} (tutti gli autovalori della companion matrix all'interno del cerchio unitario);
    \item l'\emph{assenza di autocorrelazione residua} (test di Ljung--Box sui residui di ciascuna equazione).
\end{enumerate}

Tutti i modelli risultano stabili, ma l'autocorrelazione residua viene eliminata solo a partire da $p=7$. Il VAR(7) rappresenta quindi il primo modello econometricamente valido (stabile e con residui assimilabili a rumore bianco), ed è adottato come specificazione finale. Questo risultato conferma che il sistema petrolifero presenta una \emph{memoria lunga}: sono necessari almeno sette ritardi mensili per catturare in modo adeguato la dinamica congiunta di produzione, attività economica, prezzo e scorte.

\subsection{IRF e FEVD del VAR(7): cosa aggiunge rispetto all'OLS}

Una volta stimato il VAR(7), è possibile analizzare:
\begin{itemize}
    \item le \emph{Impulse Response Functions} (IRF), che descrivono la risposta dinamica di ciascuna variabile a uno shock unitario nelle altre;
    \item la \emph{Forecast Error Variance Decomposition} (FEVD), che quantifica la quota di varianza dell'errore di previsione attribuibile a ciascun shock.
\end{itemize}

L'identificazione iniziale avviene tramite una decomposizione di Cholesky della matrice di covarianza degli errori, assumendo l'ordine economico
\[
\text{Production} \rightarrow \text{REA} \rightarrow \text{WTI} \rightarrow \text{Inventories},
\]
coerente con l'idea che la produzione sia più esogena nel breve periodo, seguita dall'attività economica, dal prezzo e infine dalle scorte.

Le IRF ortogonalizzate mostrano che:
\begin{itemize}
    \item uno shock nella produzione (interpretabile, in ridotto-forma, come perturbazione lato offerta) genera una risposta del prezzo del WTI piccola, oscillante e di breve durata;
    \item uno shock nella REA (domanda aggregata) induce una reazione relativamente più ampia del prezzo, concentrata nei primi mesi, ma non particolarmente persistente;
    \item uno shock positivo nelle scorte commerciali (\emph{Inventories}) tende, in ridotto-forma, ad associare un calo temporaneo del prezzo, coerente con un aumento percepito della disponibilità fisica, ma non identificabile di per sé come \emph{shock di domanda precauzionale}.
\end{itemize}

La FEVD del WTI conferma che gli shock fondamentali spiegano solo una quota ridotta della variabilità del prezzo:
\begin{itemize}
    \item shock associati alla produzione di petrolio spiegano circa il $2.8\%$ della varianza del WTI a 30 mesi;
    \item shock legati all'attività economica (domanda aggregata) spiegano circa il $5.9\%$;
    \item shock sulle scorte spiegano circa il $9.8\%$;
    \item il residuo della varianza (oltre l'80\%) è attribuibile a innovazioni idiosincratiche del prezzo stesso.
\end{itemize}

Questi risultati sono coerenti con l'evidenza del modello OLS: anche in un contesto dinamico multivariato, i fondamentali inclusi nel modello spiegano solo una frazione limitata dei movimenti del prezzo. Tuttavia, rispetto all'OLS, il VAR consente di:
\begin{itemize}
    \item modellare esplicitamente la \emph{propagazione temporale} degli shock;
    \item distinguere tra orizzonti di breve e medio periodo;
    \item decomporre la varianza del prezzo tra le diverse fonti di innovazione.
\end{itemize}

\subsection{Perché il VAR ridotto-forma non è ancora sufficiente}

Il VAR stimato finora è un modello \emph{di forma ridotta}: gli shock $u_t$ sono innovazioni statistiche, ortogonalizzate tramite Cholesky, ma non hanno ancora un significato economico univoco. In particolare:
\begin{itemize}
    \item uno shock nella variabile ``scorte'' non coincide automaticamente con uno shock di \emph{domanda precauzionale}; può riflettere variazioni stagionali, sorprese negli stoccaggi, decisioni dell'OPEC o altre componenti miste;
    \item uno shock nella produzione non è necessariamente un puro shock di offerta esogena, ma può inglobare risposte endogene del lato produttivo a shock di domanda;
    \item la decomposizione Cholesky dipende in modo forte dall'ordine delle variabili e impone una struttura triangolare che raramente coincide con la struttura economica vera.
\end{itemize}

In assenza di ulteriori restrizioni, gli shock ridotti-forma non permettono di distinguere in modo pulito tra:
\begin{enumerate}
    \item \textbf{shock di offerta} (esogeni alla produzione fisica di petrolio);
    \item \textbf{shock di domanda aggregata} (legati al ciclo economico globale);
    \item \textbf{shock di domanda precauzionale} (variazioni nelle aspettative di scarsità futura che si manifestano tramite movimenti congiunti di prezzo e scorte).
\end{enumerate}

Per esempio, nella formulazione di \textcite{KilianMurphy2014}, uno shock di domanda precauzionale è definito come uno shock che, nel breve periodo, fa salire il prezzo del petrolio e scendere le scorte commerciali, in quanto gli operatori accumulano petrolio per motivi precauzionali. Un semplice shock positivo nelle scorte osservate non soddisfa questa firma: al contrario, un aumento delle scorte commerciali tende a segnalare abbondanza fisica e quindi a ridurre il prezzo. Ne consegue che l'IRF ridotto-forma ``Inventories $\rightarrow$ WTI'' non può essere interpretata direttamente come risposta a uno shock di domanda precauzionale.

In altri termini, il VAR ridotto-forma fornisce una descrizione corretta della dinamica statistica, ma non identifica ancora gli \emph{shock strutturali} sottostanti.

\subsection{Verso un VAR strutturale con restrizioni di segno}

Per passare dagli shock ridotti-forma $u_t$ agli shock strutturali $\varepsilon_t$ è necessario introdurre una matrice di impatto $B$ tale che:
\begin{equation}
u_t = B \varepsilon_t, \qquad
E[\varepsilon_t \varepsilon_t'] = I.
\end{equation}
La scelta di $B$ non è unica: molte matrici diverse generano la stessa matrice di covarianza dei residui, $\Sigma_u = E[u_t u_t'] = B B'$. La decomposizione di Cholesky fornisce una particolare $B$, ma impone una struttura rigida (triangolare) e fortemente dipendente dall'ordinamento delle variabili.

L'approccio proposto da \textcite{Kilian2009} e \textcite{KilianMurphy2014} consiste nell'utilizzare un insieme di \emph{restrizioni di segno} sulle IRF per identificare gli shock strutturali. In pratica:
\begin{itemize}
    \item si generano molte possibili matrici $B$ compatibili con $\Sigma_u$ (ad esempio rotazioni casuali della fattorizzazione di Cholesky);
    \item per ciascuna di esse si calcolano le IRF strutturali e si verifica se soddisfano un insieme di segni attesi nei primi periodi (es.\,0--3 mesi);
    \item si conservano solo le decomposizioni che rispettano tutte le restrizioni di segno, ottenendo così una famiglia di IRF coerenti con la teoria economica.
\end{itemize}

Esempi di restrizioni di segno possono essere:
\begin{itemize}
    \item \textbf{shock di offerta}: produzione in diminuzione, prezzo del WTI in aumento, scorte in diminuzione nel breve periodo;
    \item \textbf{shock di domanda aggregata}: attività economica (REA) e prezzo del WTI entrambe in aumento nel breve periodo;
    \item \textbf{shock di domanda precauzionale}: prezzo del WTI in aumento e scorte commerciali in diminuzione, con effetti limitati sulla produzione e sull'attività reale nel brevissimo termine.
\end{itemize}

L'uso di restrizioni di segno consente di:
\begin{enumerate}
    \item separare gli shock osservati nelle equazioni ridotte-forma in tre categorie economicamente interpretabili;
    \item ottenere IRF strutturali e FEVD che abbiano un significato microfondato;
    \item collegare la dinamica del prezzo del petrolio a scenari di rischio ben definiti (shock di offerta, domanda aggregata, domanda precauzionale) utili per analisi di \emph{stress testing} e strategie di \emph{hedging}.
\end{enumerate}

In conclusione, il passaggio OLS $\rightarrow$ VAR(7) permette di modellare in modo consistente la dinamica del sistema e di valutare il contributo complessivo dei fondamentali alla volatilità del prezzo del petrolio. Tuttavia, solo l'estensione a un VAR strutturale con restrizioni di segno consente di identificare in maniera credibile i tre shock fondamentali dell'economia petrolifera e di trasformare i risultati del modello in strumenti operativi per la gestione del rischio.

\section{Dal VAR ridotto-forma al VAR strutturale con restrizioni di segno}

Una volta stimato un modello VAR ridotto-forma, è possibile analizzare la dinamica congiunta delle variabili tramite le funzioni di risposta agli impulsi (IRF) e la scomposizione della varianza dell'errore di previsione (FEVD). Tuttavia, gli shock del VAR ridotto-forma non hanno un significato economico univoco: rappresentano semplici innovazioni statistiche, non interpretabili direttamente come shock di offerta, di domanda aggregata o di domanda precauzionale.

Per ottenere una decomposizione economica degli shock è necessario passare a un modello VAR \textit{strutturale} (SVAR), introducendo una matrice di impatto $B$ tale che:
\[
u_t = B \varepsilon_t, \qquad E[\varepsilon_t \varepsilon_t'] = I,
\]
dove $u_t$ sono i residui ridotto-forma e $\varepsilon_t$ rappresentano gli shock strutturali. Poiché la matrice $B$ non è identificata univocamente dalla matrice di covarianza $E[u_t u_t']$, è necessario introdurre ulteriori restrizioni.

Seguendo l’approccio proposto da \textcite{Kilian2009} e \textcite{KilianMurphy2014}, si adottano \emph{restrizioni di segno} sulle IRF nei primi istanti di risposta. L’idea è semplice: ogni shock strutturale deve produrre nel brevissimo periodo un insieme di risposte con segni coerenti con la teoria economica.

Nel caso in esame, le restrizioni adottate sono:

\begin{itemize}
    \item \textbf{Shock di offerta (Supply shock)}: la produzione diminuisce e il prezzo del petrolio aumenta nell'immediato;
    \item \textbf{Shock di domanda aggregata}: l'attività economica (REA) aumenta e il prezzo del petrolio aumenta nell'immediato;
    \item \textbf{Shock di domanda precauzionale}: le scorte commerciali diminuiscono e il prezzo del petrolio aumenta nell'immediato.
\end{itemize}

Per identificare gli shock strutturali si generano migliaia di matrici ortogonali casuali $Q$ (rotazioni), applicate alla fattorizzazione di Cholesky del VAR. Solo le rotazioni che soddisfano simultaneamente tutte le restrizioni di segno vengono accettate.

L’insieme delle IRF accettate viene quindi sintetizzato tramite la mediana (IRF strutturali) e bande di confidenza percentile (IRF\_low e IRF\_high). Questo processo permette di isolare i tre shock fondamentali del mercato petrolifero e di ottenere IRF e FEVD con interpretazione economica coerente, superando i limiti identificativi del VAR ridotto-forma.

\section{Risultati del modello SVAR a restrizioni di segno}

Una volta stimato il modello VAR ridotto-forma e verificata la debole interpretabilità
economica degli shock identificati tramite decomposizione di Cholesky, si procede alla
stima di un modello SVAR con restrizioni di segno in stile \textcite{Kilian2009} e
\textcite{KilianMurphy2014}. Tale passaggio permette di identificare in modo credibile gli
shock strutturali fondamentali del mercato del petrolio: shock di offerta, shock di domanda
aggregata e shock di domanda precauzionale.

Le restrizioni di segno imposte all'impatto sono le seguenti:

\begin{itemize}
    \item \textbf{Shock di offerta (supply shock)}: la produzione diminuisce e il prezzo del petrolio aumenta ($\Delta\text{Production}<0$, $\Delta\text{WTI}>0$).
    \item \textbf{Shock di domanda aggregata}: l'attività economica aumenta e il prezzo del petrolio aumenta ($\Delta\text{REA}>0$, $\Delta\text{WTI}>0$).
    \item \textbf{Shock di domanda precauzionale}: le scorte commerciali diminuiscono e il prezzo del petrolio aumenta ($\Delta\text{Inventories}<0$, $\Delta\text{WTI}>0$).
\end{itemize}

La procedura di identificazione consiste nel generare rotazioni ortogonali casuali della
fattorizzazione di Cholesky e nel selezionare esclusivamente le matrici di impatto
compatibili con le firme dei segni sopra indicate. L’insieme delle IRF accettate viene
quindi sintetizzato tramite la mediana, ottenendo così le IRF strutturali.

\subsection{Impulse Response Functions strutturali}

Le IRF strutturali del prezzo del petrolio (WTI) ai tre shock fondamentali presentano
caratteristiche coerenti con la teoria economica e con l’evidenza empirica più recente.

\paragraph{Shock di offerta.}
Lo shock di offerta (diminuzione esogena della produzione di petrolio) genera una
risposta immediata e molto ampia del prezzo del petrolio. L’impatto all’impatto è
positivo e significativo, con un effetto di breve periodo che si smorza progressivamente
entro circa cinque mesi. Tale dinamica è coerente con episodi di shock geopolitici,
interruzioni inattese della produzione o interventi restrittivi dell’OPEC. La forma
dell’IRF riproduce fedelmente la dinamica presentata da \textcite{Kilian2009}.

\paragraph{Shock di domanda aggregata.}
Uno shock di domanda aggregata (crescita dell'attività reale globale) produce un aumento
del prezzo del petrolio più contenuto rispetto allo shock di offerta. L’impulso iniziale è
positivo ma modesto, seguito da un temporaneo rimbalzo negativo e da una rapida
riconvergenza verso lo zero. Questo risultato suggerisce che, negli anni recenti, la
domanda globale esercita un ruolo meno dominante nei movimenti del prezzo del petrolio,
in linea con l’evidenza post-2014 che sottolinea un crescente peso degli shock
specifici dell’industria petrolifera rispetto ai cicli economici globali.

\paragraph{Shock di domanda precauzionale.}
Lo shock di domanda precauzionale, definito come una variazione nelle aspettative di
scarsità futura che induce gli operatori ad accumulare scorte oggi, genera un aumento
iniziale del prezzo del petrolio, sebbene di entità moderata. L’effetto si attenua rapidamente,
diventando leggermente negativo nei mesi successivi per poi convergere verso lo zero.
L’impatto relativamente debole di questo shock riflette la minore rilevanza, negli anni
recenti, delle aspettative di scarsità futura rispetto al periodo 1970--2005 analizzato
dalla letteratura classica.

\subsection{Forecast Error Variance Decomposition strutturale}

La scomposizione strutturale della varianza dell’errore di previsione del prezzo del
petrolio fornisce una chiara misura dell’importanza relativa dei tre shock identificati.
A un orizzonte di 24 mesi, i risultati principali sono:

\begin{itemize}
    \item lo \textbf{shock di offerta} spiega circa il \textbf{50.8\%} della variabilità del WTI;
    \item lo \textbf{shock di domanda aggregata} spiega circa il \textbf{15.0\%};
    \item lo \textbf{shock di domanda precauzionale} spiega circa l’\textbf{8.1\%};
    \item la restante quota è attribuibile a innovazioni proprie del prezzo del petrolio
          e a componenti non spiegate dalle tre categorie di shock.
\end{itemize}

Questi valori rappresentano una marcata differenza rispetto ai risultati ottenuti dal VAR
ridotto-forma. In particolare, mentre il modello ridotto-forma attribuiva oltre l’80\% della
variabilità del WTI a innovazioni proprie del prezzo, lo SVAR a restrizioni di segno
redistribuisce gran parte della varianza verso gli shock strutturali, in particolare verso lo
shock di offerta.

\subsection{Discussione}

Il modello SVAR rivela che la volatilità del prezzo del petrolio è dominata dagli shock di
offerta. Gli shock di domanda aggregata svolgono un ruolo secondario, e gli shock di
domanda precauzionale hanno un impatto limitato ma comunque non trascurabile.
Questa evidenza è completamente coerente con la letteratura più recente e con la
finanziarizzazione dei mercati petroliferi.

Nel complesso, il modello SVAR consente di identificare correttamente le tre fonti
strutturali di shock nel mercato del petrolio, superando le limitazioni del VAR
ridotto-forma e fornendo una base quantitativa affidabile per la costruzione di scenari di
rischio e per lo sviluppo di strategie di hedging per gli operatori esposti alle fluttuazioni
del prezzo del petrolio.
